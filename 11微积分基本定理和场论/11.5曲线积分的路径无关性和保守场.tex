\section{曲线积分的路径无关性和保守场}

\begin{theorem}
设$D$为平面上的单连通区域,$\boldsymbol{f}\left( \boldsymbol{p} \right) =\left( P\,\,Q \right) ^T$是定义在二维平面上的向量值函数,且$P,Q$在$D$上有一阶连续偏导数,则下面4个命题等价:
\begin{enumerate}[label=\Roman*.]
    \item $D$内封闭曲线$L$上有$\oint_L{\boldsymbol{f}^T\boldsymbol{dl}}=0$;
    \item $\oint_L{\boldsymbol{f}^T\boldsymbol{dl}}=0$在$D$内与路径无关;
    \item $D$中必有数量值函数$g\left( \boldsymbol{p} \right) $,满足$\frac{\partial g}{\partial x}=P,\frac{\partial g}{\partial y}=Q$,即有全微分$dg=Pdx+Qdy$;
    \item $D$内每一点满足$\frac{\partial Q}{\partial x}=\frac{\partial P}{\partial y}$。
\end{enumerate}
\end{theorem}

以上:
\begin{itemize}
    \item 命题I和IV的等价性由Green公式体现;
    \item 命题III和IV综合即为二阶混偏相等定理。
\end{itemize}

\begin{theorem}
假设三维空间中有一单连通区域,$\boldsymbol{f}\left( \boldsymbol{p} \right) =\left( P\,\,Q\,\,R \right) ^T$是定义在该区域上的向量值函数,且$P,Q,R$在该区域上有一阶连续偏导数,则下面4个命题等价:
\begin{enumerate}[label=\Roman*.]
    \item 区域内任意封闭曲线$L$上有$\oint_L{\boldsymbol{f}^T\boldsymbol{dl}}=0$;
    \item $\oint_L{\boldsymbol{f}^T\boldsymbol{dl}}=0$在区域内与路径无关;
    \item 区域内必有数量值函数$g\left( \boldsymbol{p} \right) $有全微分,且满足$dg=Pdx+Qdy+Rdz$;
    \item 区域内每点满足$\frac{\partial R}{\partial y}=\frac{\partial Q}{\partial z},\frac{\partial P}{\partial z}=\frac{\partial R}{\partial x},\frac{\partial Q}{\partial x}=\frac{\partial P}{\partial y}$,即
    \[
    \left| \begin{matrix}
    	\mathbf{i}&		\mathbf{j}&		\mathbf{k}\\
    	\frac{\partial}{\partial x}&		\frac{\partial}{\partial y}&		\frac{\partial}{\partial z}\\
    	P&		Q&		R\\
    \end{matrix} \right|=0
    \]
\end{enumerate}
\end{theorem}

\begin{definition}
如果一个矢量场中,在任何区域内沿任何曲线的第二类曲线积分与路径无关,只与曲线的起点和终点有关,称该矢量场为{\bf 保守场}。
如果一个矢量场是保守场,则该场的旋度为0,$\nabla \times \boldsymbol{f}=0$,同时该场必然是一个标量场的梯度,即$\nabla g=\boldsymbol{f}$,$g$称为$\boldsymbol{f}$的{\bf 势函数},$\boldsymbol{f}$也称为{\bf 有势场}。
\end{definition}

也即这4个概念等价:保守场$\Leftrightarrow $无旋场$\Leftrightarrow $梯度场$\Leftrightarrow $有势场。




