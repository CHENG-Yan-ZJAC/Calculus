\section{斯托克斯Stokes公式和旋度}

本节介绍Stokes公式。

本节要点:
\begin{itemize}
    \item 掌握Stokes公式的概念;
    \item 理解Stokes公式的物理意义;
    \item 掌握旋度的概念;
    \item 理解旋度的物理意义。
\end{itemize}

%============================================================
\subsection{Stokes公式的概念}

\begin{definition}[Stokes公式]
假设三维空间中有分片光滑曲面$S$,其边界$L$分段光滑,两者方向符合右手定则,即右手四指表示曲线$L$方向,大拇指的指向和曲面$S$的方向一致,$\boldsymbol{f}\left( \boldsymbol{p} \right) =\left( P\,\,Q\,\,R \right) ^T$是定义在三维空间上的向量值函数,且$P,Q,R$在$S$上有一阶连续偏导数,则有:
\begin{align*}
&\oint\limits_L{\boldsymbol{f}^T\boldsymbol{dl}}=\oint\limits_L{Pdx+Qdy+Rdz}= \\
&\iint\limits_S{\left( \frac{\partial R}{\partial y}-\frac{\partial Q}{\partial z} \right) dydz+\left( \frac{\partial P}{\partial z}-\frac{\partial R}{\partial x} \right) dzdx+\left( \frac{\partial Q}{\partial x}-\frac{\partial P}{\partial y} \right) dxdy}
\end{align*}
称为{\bf Stokes公式}。
\end{definition}

Green公式可视为Stokes的二维特例,之后一般情况下将不作讨论。

%============================================================
\subsection{Stokes公式的物理意义}

Stokes公式左边代表一个矢量场沿一个闭合曲线一周产生的环流量:
\begin{itemize}
    \item $\boldsymbol{f}^T\boldsymbol{dl}$:环流微量,表示沿封闭曲线方向产生的微作用力;
    \item $\oint_L$:环曲线一周的累积效果;
    \item $\oint_L{\boldsymbol{f}^T\boldsymbol{dl}}=0$:表示绕该闭合曲线一周不产生效果;
    \item $\oint_L{\boldsymbol{f}^T\boldsymbol{dl}}>0$:表示绕该闭合曲线一周会产生某种“正”效果,可以理解为“加速”效果;
    \item $\oint_L{\boldsymbol{f}^T\boldsymbol{dl}}<0$:表示绕该闭合曲线一周会产生某种“负”效果,可以理解为“减速”效果。
\end{itemize}

\begin{tcolorbox}
Stokes公式右边的物理意义暂不考虑。到这里,只需要理解“环流量”,更深入的物理意义在给出旋度这个概念后再分析。
\end{tcolorbox}

%============================================================
\subsection{旋度的概念}

如果将环无限缩小,就是在该点的自旋一周产生的环流效果。

\begin{definition}[旋度]
若三维空间中有矢量值函数$\boldsymbol{f}\left( \boldsymbol{p} \right) $在有向闭合曲线$L$内的积分$\oint_L{\boldsymbol{f}^T\boldsymbol{dl}}$,设$\lambda =\max \left\{ \left\| \boldsymbol{p}_0-\boldsymbol{p} \right\| \right\} $表示该曲线围成的曲面上最远点到目标点$\boldsymbol{p}_0$的距离,设$S$为包围的曲面的面积,若当$\lambda \rightarrow 0$时,极限$\underset{\lambda \rightarrow 0}{\lim}\frac{\oint_L{\boldsymbol{f}\left( \boldsymbol{p}_0 \right) ^T\boldsymbol{dl}}}{S}$存在,则称该极限为{\bf $\boldsymbol{f}\left( \boldsymbol{p} \right) $在$\boldsymbol{p}_0$点的旋度},记为$\mathbf{rot}\boldsymbol{f}$,即:
\[
\mathbf{rot}\boldsymbol{f}=\underset{\lambda \rightarrow 0}{\lim}\frac{\oint\limits_L{\boldsymbol{f}\left( \boldsymbol{p}_0 \right) ^T\boldsymbol{dl}}}{S}
\]
\end{definition}

根据Stokes公式得到旋度计算公式:
\begin{align*}
\mathbf{rot}\boldsymbol{f}&=\underset{L\rightarrow M}{\lim}\frac{\iint\limits_S{\left( \frac{\partial R}{\partial y}-\frac{\partial Q}{\partial z} \right) dydz+\left( \frac{\partial P}{\partial z}-\frac{\partial R}{\partial x} \right) dzdx+\left( \frac{\partial Q}{\partial x}-\frac{\partial P}{\partial y} \right) dxdy}}{S} \\
&=\underset{L\rightarrow M}{\lim}\frac{1}{S}\iint\limits_S{\left( \begin{array}{c}
	\frac{\partial R}{\partial y}-\frac{\partial Q}{\partial z}\\
	\frac{\partial P}{\partial z}-\frac{\partial R}{\partial x}\\
	\frac{\partial Q}{\partial x}-\frac{\partial P}{\partial y}\\
\end{array} \right) ^T\left( \begin{array}{c}
	dydz\\
	dzdx\\
	dxdy\\
\end{array} \right)} \\
&=\underset{L\rightarrow M}{\lim}\frac{1}{S}\iint\limits_S{\left( \begin{array}{c}
	\frac{\partial R}{\partial y}-\frac{\partial Q}{\partial z}\\
	\frac{\partial P}{\partial z}-\frac{\partial R}{\partial x}\\
	\frac{\partial Q}{\partial x}-\frac{\partial P}{\partial y}\\
\end{array} \right) ^T\boldsymbol{ds}} \\
&=\left( \begin{array}{c}
	\frac{\partial R}{\partial y}-\frac{\partial Q}{\partial z}\\
	\frac{\partial P}{\partial z}-\frac{\partial R}{\partial x}\\
	\frac{\partial Q}{\partial x}-\frac{\partial P}{\partial y}\\
\end{array} \right) =\left| \begin{matrix}
	\mathbf{i}&		\mathbf{j}&		\mathbf{k}\\
	\frac{\partial}{\partial x}&		\frac{\partial}{\partial y}&		\frac{\partial}{\partial z}\\
	P&		Q&		R\\
\end{matrix} \right|
\end{align*}
运用旋度公式,Stokes公式可以表示为:
\[
\oint\limits_L{\boldsymbol{f}^T\boldsymbol{dl}}=\iint\limits_S{\left( \mathbf{rot}\boldsymbol{f} \right) ^T\boldsymbol{ds}}
\]

注意:
\begin{itemize}
    \item 旋度只能针对矢量场,数量场没有旋度这个概念;
    \item 旋度描述了一矢量场在某点的旋转程度(可以理解为角速度);
    \item 该旋转程度可以分解对应到各个坐标轴的旋转程度,所以旋度的结果是一个矢量场;
    \item 一个矢量场的旋度可以处处为0,表示该场没有任何旋转,即该矢量场是个{\bf 无旋场},如点电荷产生的电场。
\end{itemize}

\begin{tcolorbox}
矢量场可以没有旋度,如静止的点电荷产生的电场,电力线“直直地”发散出去,空间任意一点“左右两侧”的场强方向和大小一致,不会使得该点处的电荷发生转动。
再如,一个均匀的、直的河流,水流“笔直地”向前流,没有旋转,但如果河道弯曲,流水沿河道方向有拐弯,产生了涡流,在该涡流处的叶子会发生“转动”。
所以,旋度的“旋”,表示该点处“左右”两侧流速“相反”,会使该点“旋转”。
\end{tcolorbox}

%============================================================
\subsection{旋度的量纲和物理意义}

结合物理单位,假设流速场$\boldsymbol{v}$:

\begin{table}[h]
\centering
\begin{tabular}{lll}
    \toprule
    表达式 & 量纲\\
    \midrule
    $\boldsymbol{v}$ & $\mathrm{m}\cdot \mathrm{s}^{-1}$\\
    $\oint_L{\boldsymbol{v}^T\boldsymbol{dl}}$ & $\mathrm{m}^2\cdot \mathrm{s}^{-1}$\\
    $\mathbf{rot}\boldsymbol{v}=\underset{\lambda \rightarrow 0}{\lim}\frac{\oint_L{\boldsymbol{v}^T\boldsymbol{dl}}}{S}$ & $\mathrm{s}^{-1}$\\
    \bottomrule
\end{tabular}
\end{table}

旋度表示空间里一点的自转角速度。
如果一个曲面上每点自转角速度累加起来不为零,则整个曲面也会转起来,整个曲面的自旋可以通过环流量度量。

\begin{tcolorbox}
Stokes公式的物理意义在于,其描述的是一个矢量场,确切地讲是描述一个矢量场在一个曲面上造成的自旋程度。
公式右边表示曲面上各个点自旋程度的累积,左边则是这种自旋在边界造成的环流量。
\end{tcolorbox}

%============================================================
\subsection{旋度和微分算子}

根据旋度公式和微分算子$\nabla =\left( \frac{\partial}{\partial x}\,\,\frac{\partial}{\partial y}\,\,\frac{\partial}{\partial z} \right) ^T$,可以做纯数学上的抽象:
\begin{align*}
&\because \begin{cases}
	\mathbf{rot}\boldsymbol{f}=\left| \begin{matrix}
	\mathbf{i}&		\mathbf{j}&		\mathbf{k}\\
	\frac{\partial}{\partial x}&		\frac{\partial}{\partial y}&		\frac{\partial}{\partial z}\\
	P&		Q&		R\\
\end{matrix} \right|\\
	\nabla \times \boldsymbol{f}=\left( \begin{array}{c}
	\frac{\partial}{\partial x}\\
	\frac{\partial}{\partial y}\\
	\frac{\partial}{\partial z}\\
\end{array} \right) \times \left( \begin{array}{c}
	P\\
	Q\\
	R\\
\end{array} \right) =\left| \begin{matrix}
	\mathbf{i}&		\mathbf{j}&		\mathbf{k}\\
	\frac{\partial}{\partial x}&		\frac{\partial}{\partial y}&		\frac{\partial}{\partial z}\\
	P&		Q&		R\\
\end{matrix} \right|\\
\end{cases} \\
&\therefore \mathbf{rot}\boldsymbol{f}=\nabla \times \boldsymbol{f}
\end{align*}
即旋度可以记为微分算子和矢量场的外积,于是Stokes公式可以写为:
\[
\oint\limits_L{\boldsymbol{f}^T\boldsymbol{dl}}=\iint\limits_S{\left( \nabla \times \boldsymbol{f} \right) ^T\boldsymbol{ds}}
\]




