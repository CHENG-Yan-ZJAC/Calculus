\section{度、微分算子和拉普拉斯算子}

梯度作用于标量场,得到其关于变化率的一个矢量场:
\[
\mathbf{grad}f=\nabla f=\left( \frac{\partial f}{\partial x} \quad \frac{\partial f}{\partial y} \quad \frac{\partial f}{\partial z} \right) ^T
\]
散度作用于矢量场,得到其关于流出汇入程度的一个标量场:
\[
\mathrm{div}\boldsymbol{f}=\nabla ^T\boldsymbol{f}=\frac{\partial P}{\partial x}+\frac{\partial Q}{\partial y}+\frac{\partial R}{\partial z}
\]
旋度作用于矢量场,得到其关于旋转速度的一个矢量场:
\[
\mathbf{rot}\boldsymbol{f}=\nabla \times \boldsymbol{f}=\left| \begin{matrix}
	\mathbf{i}&		\mathbf{j}&		\mathbf{k}\\
	\frac{\partial}{\partial x}&		\frac{\partial}{\partial y}&		\frac{\partial}{\partial z}\\
	P&		Q&		R\\
\end{matrix} \right|=\left( \begin{array}{c}
	\frac{\partial R}{\partial y}-\frac{\partial Q}{\partial z}\\
	\frac{\partial P}{\partial z}-\frac{\partial R}{\partial x}\\
	\frac{\partial Q}{\partial x}-\frac{\partial P}{\partial y}\\
\end{array} \right)
\]
特别地,标量场的梯度可以再次作散度,记作$\nabla ^2f$,即:
\[
\nabla ^2f=\nabla ^T\left( \nabla f \right) =\left( \begin{array}{c}
	\frac{\partial}{\partial x}\\
	\frac{\partial}{\partial y}\\
	\frac{\partial}{\partial z}\\
\end{array} \right) ^T\left( \begin{array}{c}
	\frac{\partial f}{\partial x}\\
	\frac{\partial f}{\partial y}\\
	\frac{\partial f}{\partial z}\\
\end{array} \right) =\frac{\partial ^2f}{\partial x^2}+\frac{\partial ^2f}{\partial y^2}+\frac{\partial ^2f}{\partial z^2}
\]
我们称$\nabla ^2$为{\bf 拉普拉斯算子},记作$\Delta $,即:
\[
\Delta :=\nabla ^2=\frac{\partial ^2}{\partial x^2}+\frac{\partial ^2}{\partial y^2}+\frac{\partial ^2}{\partial z^2}
\]

拉普拉斯算子作用于标量场,首先得到变化率的矢量场,再得到该矢量场的汇入流出情况。
最终,拉普拉斯算子等价于得到一个标量场的最高点和最低点。




