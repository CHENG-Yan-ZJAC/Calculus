\section{高斯Gauss公式和散度}

本节介绍Gauss公式。

本节要点:
\begin{itemize}
    \item 掌握Gauss公式的概念;
    \item 理解Gauss公式的物理意义;
    \item 掌握散度的概念;
    \item 理解散度的物理意义。
\end{itemize}

%============================================================
\subsection{Gauss公式的概念}

\begin{definition}[Gauss公式]
假设三维空间中有单连通区域$V$,其外侧边界$S$分片光滑,$\boldsymbol{f}\left( \boldsymbol{p} \right) =\left( P\,\,Q\,\,R \right) ^T$是定义在三维空间上的向量值函数,且$P,Q,R$在$V$上有一阶连续偏导数,则有:
\begin{align*}
&\oiint\limits_S{\boldsymbol{f}^T\boldsymbol{ds}}=\oiint\limits_S{Pdydz+Qdzdx+Rdxdy}= \\
&\iiint\limits_V{\left( \frac{\partial P}{\partial x}+\frac{\partial Q}{\partial y}+\frac{\partial R}{\partial z} \right) dv}
\end{align*}
上述公式称为{\bf Gauss公式}。
\end{definition}

%============================================================
\subsection{Gauss公式的物理意义}

Gauss公式左边代表一个矢量场对于一个封闭曲面的总通量:
\begin{itemize}
    \item $\boldsymbol{f}^T\boldsymbol{ds}$:通量微元,或者说是矢量场在曲面微元形成的通量;
    \item $\oiint_S$:对整个封闭曲面的总累积量;
    \item $\oiint_S{\boldsymbol{f}^T\boldsymbol{ds}}=0$:通量为0,表示流入流出该空间的量相抵消;
    \item $\oiint_S{\boldsymbol{f}^T\boldsymbol{ds}}>0$:表示有东西流出该区域;
    \item $\oiint_S{\boldsymbol{f}^T\boldsymbol{ds}}<0$:表示有东西流入该区域。
\end{itemize}

Gauss公式右边代表这个封闭曲面中源或者汇:
\begin{itemize}
    \item $\left( \frac{\partial P}{\partial x}+\frac{\partial Q}{\partial y}+\frac{\partial R}{\partial z} \right) dv$:该点处的源或者汇;
    \item $\iiint_V$:整个区域中的所有的源或汇;
    \item $\iiint_V{\left( \frac{\partial P}{\partial x}+\frac{\partial Q}{\partial y}+\frac{\partial R}{\partial z} \right) dv}=0$:表示区域内既无源也无汇;
    \item $\iiint_V{\left( \frac{\partial P}{\partial x}+\frac{\partial Q}{\partial y}+\frac{\partial R}{\partial z} \right) dv}>0$:表示区域内有源;
    \item $\iiint_V{\left( \frac{\partial P}{\partial x}+\frac{\partial Q}{\partial y}+\frac{\partial R}{\partial z} \right) dv}<0$:表示区域内有汇。
\end{itemize}

\begin{tcolorbox}
Gauss公式的物理意义显而易见,整个公式描述一个矢量场。
左边从其对区域边界的作用效果的角度描述,即矢量场在边界造成的通量。
右边从其在区域内的总的发散或汇聚的量的角度描述。
\end{tcolorbox}

%============================================================
\subsection{散度的概念}

如果将曲面无限缩小,就是该点的发散或汇聚的量。

\begin{definition}[散度]
若三维空间中有矢量值函数$\boldsymbol{f}\left( \boldsymbol{p} \right) $在有向封闭曲面$S$内的积分$\oiint_S{\boldsymbol{f}^T\boldsymbol{ds}}$,设$\lambda =\max \left\{ \left\| \boldsymbol{p}_0-\boldsymbol{p} \right\| \right\} $表示该曲面的包围体中最远点到目标点$\boldsymbol{p}_0$的距离,设$V$为包围体的体积,若当$\lambda \rightarrow 0$时,极限$\underset{\lambda \rightarrow 0}{\lim}\frac{\oiint_S{\boldsymbol{f}\left( \boldsymbol{p}_0 \right) ^T\boldsymbol{ds}}}{V}$存在,则称该极限为{\bf $\boldsymbol{f}\left( \boldsymbol{p} \right) $在$\boldsymbol{p}_0$点的散度},记为$\mathrm{div}\boldsymbol{f}$,即:
\[
\mathrm{div}\boldsymbol{f}=\underset{\lambda \rightarrow 0}{\lim}\frac{\oiint\limits_S{\boldsymbol{f}^T\boldsymbol{ds}}}{V}
\]
\end{definition}

根据Gauss公式可以得到散度的计算公式:
\[
\mathrm{div}\boldsymbol{f}=\underset{V\rightarrow \boldsymbol{p}_0}{{\lim}}\frac{1}{V}\iiint\limits_V{\left( \frac{\partial P}{\partial x}+\frac{\partial Q}{\partial y}+\frac{\partial R}{\partial z} \right) dv}=\frac{\partial P}{\partial x}+\frac{\partial Q}{\partial y}+\frac{\partial R}{\partial z}
\]
反过来,运用散度公式,Gauss公式可以表示为:
\[
\oiint\limits_S{\boldsymbol{f}^T\boldsymbol{ds}}=\iiint\limits_V{\mathrm{div}\boldsymbol{f}\cdot dv}
\]
注意:
\begin{itemize}
    \item 散度只能针对矢量场,数量场没有“流”,所以也就没有散度这个概念;
    \item 散度是该点流出汇入的量的体密度,根据Gauss公式,是该点沿各坐标轴的流出汇入程度的代数和;
    \item 一个矢量场的散度是一个数量场,可以处处为0,表示该场没有流出点和汇入点,即该矢量场是个{\bf 无源场},如磁场。
\end{itemize}

\begin{tcolorbox}
Gauss公式的物理意义更显而易见,一封闭曲面包围下的通量等于该包围下所有点的散度和。
\end{tcolorbox}

~

\begin{example}
设有矢量场$\boldsymbol{f}\left( \boldsymbol{p} \right) =\left( xy^2 \quad ye^z \quad x\ln \left( 1+z^2 \right) \right) $,求该向量场的散度场,并求在点$\left( 1,1,0 \right) $处的散度。
\end{example}

解:
\begin{align*}
&\mathrm{div}\boldsymbol{f}=\frac{\partial P}{\partial x}+\frac{\partial Q}{\partial y}+\frac{\partial R}{\partial z}=y^2+e^z+x\frac{2z}{1+z^2} \\
&\left. \mathrm{div}\boldsymbol{f} \right|_{\left( 1,1,0 \right)}=1+1+1\cdot \frac{0}{1}=2
\end{align*}

%============================================================
\subsection{散度的量纲和物理意义}

结合物理量的单位的理解散度的物理意义。
假设有流速场$\boldsymbol{v}$,量纲为$\mathrm{m}\cdot \mathrm{s}^{-1}$,表示该点的流速:

\begin{table}[h]
\centering
\begin{tabular}{lll}
    \toprule
    表达式 & 量纲\\
    \midrule
    $\boldsymbol{v}$ & $\mathrm{m}\cdot \mathrm{s}^{-1}$\\
    $\oiint_S{\boldsymbol{v}^T\boldsymbol{ds}}$ & $\mathrm{m}^3\cdot \mathrm{s}^{-1}$\\
    $\mathrm{div}\boldsymbol{v}=\underset{\lambda \rightarrow 0}{\lim}\frac{\oiint_S{\boldsymbol{v}^T\boldsymbol{ds}}}{V}$ & $\mathrm{s}^{-1}$\\
    \bottomrule
\end{tabular}
\end{table}

$\mathrm{div}\boldsymbol{v}$的量纲没有$\mathrm{m}$,是$\mathrm{s}^{-1}$,说明散度作为空间上的一个“点”,本身不承载尺度信息,只反映了该点的流速的程度。
如果我们做一个体积分再加上一个时间积分,就是一段时间内流出或流入一个体的总流量。
\[
\int_{t_1}^{t_2}{\left( \iiint\limits_V{\mathrm{div}\boldsymbol{v}\cdot dv} \right) dt}
\]

%============================================================
\subsection{散度和微分算子}

在第七章中介绍了向量微分算子$\nabla =\left( \frac{\partial}{\partial x}\,\,\frac{\partial}{\partial y}\,\,\frac{\partial}{\partial z} \right) ^T$。
根据散度公式和微分算子,可以做纯数学上的抽象:
\begin{align*}
&\because \begin{cases}
	\mathrm{div}\boldsymbol{f}=\frac{\partial P}{\partial x}+\frac{\partial Q}{\partial y}+\frac{\partial R}{\partial z}\\
	\nabla ^T\boldsymbol{f}=\left( \begin{array}{c}
	\frac{\partial}{\partial x}\\
	\frac{\partial}{\partial y}\\
	\frac{\partial}{\partial z}\\
\end{array} \right) ^T\left( \begin{array}{c}
	P\\
	Q\\
	R\\
\end{array} \right) =\frac{\partial P}{\partial x}+\frac{\partial Q}{\partial y}+\frac{\partial R}{\partial z}\\
\end{cases} \\
&\therefore \mathrm{div}\boldsymbol{f}=\nabla ^T\boldsymbol{f}
\end{align*}
即散度可以记为微分算子和矢量场的内积,于是Gauss公式还可以写为:
\[
\oiint\limits_S{\boldsymbol{f}^T\boldsymbol{ds}}=\iiint\limits_V{\left( \nabla ^T\boldsymbol{f} \right) dv}
\]




