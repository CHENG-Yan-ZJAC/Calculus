\section{微积分基本定理}

本节在纯数学的抽象上讨论Gauss公式和Stokes公式的意义。
首先引入微分的外积、外微分形式和外微分算子,将两个公式进行外微分化,得到微积分基本定理的统一公式。
最后在外微分的角度重新考察一下三个度。

本节要点:
\begin{itemize}
    \item 了解微分外积的概念;
    \item 了解外微分形式的概念;
    \item 了解统一公式。
\end{itemize}

%============================================================
\subsection{微分的外积}

在一元积分中,有方向的概念,即$\int_a^b{fdx}=-\int_b^a{fdx}$。
在第二类曲线积分和第二类曲面积分中,也有方向的概念。
第二类曲线积分中的方向是曲线的切线的方向,第二类曲面积分中的方向是曲面的切平面的法方向。
但公式中的如$dxdy$本身不承担方向的正负性,即$dxdy=dydx$。
方向及其正反性都由切线和切平面决定。
讨论微积分基本定理、挖掘三个公式的共同点时,需要把这个方向的“正反性”让$dxdy$承担,即我们需要定义一个概念,或者说一个运算规则,使得$dx\bigcirc dy=-dy\bigcirc dx$。

\begin{definition}[微分外积]
规定一种微分的乘法运算,记为$\land $,使得:
\begin{itemize}
    \item $dx\land dy=-dy\land dx$
    \item $dx\land dx=0$
\end{itemize}
满足这两条的微分乘法称为{\bf 微分的外积}。同时称$dxdy$为{\bf 微分的内积}。
\end{definition}

\begin{tcolorbox}
微分的内积和外积,可以类比于矢量的内积和外积。
\end{tcolorbox}

%============================================================
\subsection{外微分形式}

\begin{definition}[外微分形式]
设$P,Q,R$均为定义在三维空间的三元数量值函数,且在三维空间的某区域$V$内可微,我们定义如下概念。

称它们各自对三个坐标轴微分的积的代数和为{\bf $V$上的一个一阶外微分形式},记为$\omega $,即:
\[
\omega =Pdx+Qdy+Rdz
\]
$P,Q,R$称为这个{\bf 一阶外微分形式$\omega $的系数}。

称它们各自对三个坐标平面的外积微分的积的代数和为{\bf $V$上的一个二阶外微分形式},同样可以记为$\omega $,即:
\[
\omega =Pdy\land dz+Qdz\land dx+Rdx\land dy
\]
$P,Q,R$称为这个{\bf 二阶外微分形式$\omega $的系数}。

设$f$在三维空间$V$内可微,则称它对$V$上的外积形式的体积微元为{\bf $V$上的一个三阶外微分形式},记为$\omega $,即:
\[
\omega =fdx\land dy\land dz
\]
$f$称为这个{\bf 三阶外微分形式$\omega $的系数}。

特别地,我们称这个可微函数$f$本身为{\bf $V$上的一个零阶外微分形式}。

\end{definition}

%============================================================
\subsection{外微分算子}

对外微分形式$\omega $定义外微分算子$d$。

对于零阶外微分形式,即函数$f$本身,外微分算子为$f$的全微分:
\[
d\omega =df=\frac{\partial f}{\partial x}dx+\frac{\partial f}{\partial y}dy+\frac{\partial f}{\partial z}dz
\]

对于一阶外微分形式$\omega =Pdx+Qdy+Rdz$,外微分算子为:
\begin{align*}
&d\omega =dP\land dx+dQ\land dy+dR\land dz \\
&\because \left\{ \begin{aligned}
	dP\land dx&=\left( \frac{\partial P}{\partial x}dx+\frac{\partial P}{\partial y}dy+\frac{\partial P}{\partial z}dz \right) \land dx\\
	&=\frac{\partial P}{\partial y}dy\land dx+\frac{\partial P}{\partial z}dz\land dx\\
	dQ\land dy&=\left( \frac{\partial Q}{\partial x}dx+\frac{\partial Q}{\partial y}dy+\frac{\partial Q}{\partial z}dz \right) \land dy\\
	&=\frac{\partial Q}{\partial x}dx\land dy+\frac{\partial Q}{\partial z}dz\land dy\\
	dR\land dz&=\left( \frac{\partial R}{\partial x}dx+\frac{\partial R}{\partial y}dy+\frac{\partial R}{\partial z}dz \right) \land dz\\
	&=\frac{\partial R}{\partial x}dx\land dz+\frac{\partial R}{\partial y}dy\land dz\\
\end{aligned} \right. \\
&\therefore d\omega =\left( \frac{\partial R}{\partial y}-\frac{\partial Q}{\partial z} \right) dy\land dz+\left( \frac{\partial P}{\partial z}-\frac{\partial R}{\partial x} \right) dz\land dx \\
& \qquad \qquad +\left( \frac{\partial Q}{\partial x}-\frac{\partial P}{\partial y} \right) dx\land dy
\end{align*}

对于二阶外微分形式$\omega =Pdy\land dz+Qdz\land dx+Rdx\land dy$,外微分算子为:
\begin{align*}
&d\omega =dP\land dy\land dz+dQ\land dz\land dx+dR\land dx\land dy \\
&\because \left\{ \begin{aligned}
	dP\land dy\land dz&=\left( \frac{\partial P}{\partial x}dx+\frac{\partial P}{\partial y}dy+\frac{\partial P}{\partial z}dz \right) \land dy\land dz\\
	&=\frac{\partial P}{\partial y}\land dx\land dy\land dz\\
	dQ\land dz\land dx&=\left( \frac{\partial Q}{\partial x}dx+\frac{\partial Q}{\partial y}dy+\frac{\partial Q}{\partial z}dz \right) \land dz\land dx\\
	&=\frac{\partial Q}{\partial y}dx\land dy\land dz\\
	dR\land dx\land dy&=\left( \frac{\partial R}{\partial x}dx+\frac{\partial R}{\partial y}dy+\frac{\partial R}{\partial z}dz \right) \land dx\land dy\\
	&=\frac{\partial R}{\partial z}dx\land dy\land dz\\
\end{aligned} \right. \\
&\therefore d\omega =\left( \frac{\partial P}{\partial x}+\frac{\partial Q}{\partial y}+\frac{\partial R}{\partial z} \right) dx\land dy\land dz
\end{align*}

对于三阶外微分形式$\omega =fdx\land dy\land dz$,外微分算子为:
\begin{align*}
&d\omega =dfdx\land dy\land dz \\
&\because df=\frac{\partial f}{\partial x}dx+\frac{\partial f}{\partial y}dy+\frac{\partial f}{\partial z}dz \\
&\therefore d\omega =\left( \frac{\partial f}{\partial x}dx+\frac{\partial f}{\partial y}dy+\frac{\partial f}{\partial z}dz \right) \land dx\land dy\land dz=0
\end{align*}

%============================================================
\subsection{统一公式}

将Gauss公式和Stokes公式写成外微分形式:
\begin{align*}
&\oiint\limits_S{Pdy\land dz+Qdz\land dx+Rdx\land dy} \\
& \qquad \qquad \qquad \qquad =\iiint\limits_V{\left( \frac{\partial P}{\partial x}+\frac{\partial Q}{\partial y}+\frac{\partial R}{\partial z} \right) dx\land dy\land dz} \\
&\oint\limits_L{Pdx+Qdy+Rdz}=\iint\limits_S{\left( \frac{\partial R}{\partial y}-\frac{\partial Q}{\partial z} \right) dy\land dz+\left( \frac{\partial P}{\partial z}-\frac{\partial R}{\partial x} \right) dz\land dx} \\
& \qquad \qquad \qquad \qquad +\left( \frac{\partial Q}{\partial x}-\frac{\partial P}{\partial y} \right) dx\land dy
\end{align*}
进一步采用外微分算子可以写成:
\begin{align*}
&\oiint\limits_S{\omega}=\iiint\limits_V{d\omega} \\
&\oint\limits_L{\omega}=\iint\limits_S{d\omega}
\end{align*}
Gauss公式和Stokes公式有共同点,左边是对外微分形式在区域边界的积分,右边是对其外微分算子在该区域的积分。

\begin{theorem}[微积分基本定理的统一公式]
设某$n$维空间内有一封闭区域$\varOmega $,$\varOmega $上有$n$阶外微分形式$\omega $满足其系数为定义在$\varOmega $上的可微数量值函数,$\omega $对应有外微分算子$d\omega $,有:
\[
\oint\limits_{d\varOmega}{\omega}=\iint\limits_{\varOmega}{d\omega}
\]
\end{theorem}

统一公式说明,如果空间中有一个光滑场,则其在对考察区域的边界的作用(即方程的左边),等于该场的源在该区域内的累积(即方程的右边)。
从纯粹公式来看,并不刻意区分场和源的先后问题。
从右向左看,说明源导致一个场;但从左向右看,场必然汇到一个源。
考虑一个球体,电磁场流入球体表面,则必然在内部形成一个负电荷。




