\section{本章小结}

本章以二元函数为例讨论多元微积分的基础概念。
极限和连续的概念在二元函数均有所扩展。

要注意,极限在二元函数中比在一元函数中复杂很多。
一元函数中,极限存在的充分必要条件是左右极限都存在且相等。
二元函数中可以说类似,但要求的是各个方向的极限都存在且相等,这使得判断复杂很多,也困难很多。
连续的数学意义在于计算初等函数的极限,即,只要一个二元函数是“初等函数”,则在其定义域内任何点的极限都存在,都为该点的函数值。
注意,$f\left( \boldsymbol{p} \right) =\frac{xy}{x^2+y^2}$这种不是二元初等函数!

形而上来讲,我们开始了对高维空间几何体的“光滑”的考察。
本章是第一步,不断。




