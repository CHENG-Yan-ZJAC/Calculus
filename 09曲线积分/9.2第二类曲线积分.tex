\section{第二类曲线积分}

本节讨论第二类曲线积分。
所谓“第二类曲线积分”是向量值函数对曲线的积分。

本节要点:
\begin{itemize}
    \item 掌握第二类曲线积分的概念;
    \item 掌握第二类曲线积分的计算,重点掌握积分表达式的内积计算。
\end{itemize}

%============================================================
\subsection{第二类曲线积分的概念}

\begin{definition}[第二类曲线积分]
若三维空间中有曲线$L$,$\mathbf{n}\left( \boldsymbol{p} \right) $为曲线上任一点处的单位切向量,$\boldsymbol{f}\left( \boldsymbol{p} \right) $为定义在该空间上的三元向量值函数,若内积$\boldsymbol{f}\left( \boldsymbol{p} \right) ^T\mathbf{n}\left( \boldsymbol{p} \right) $在$L$上的第一类曲线积分存在,则称此积分值为{\bf $\boldsymbol{f}\left( \boldsymbol{p} \right) $在$L$上的第二类曲线积分},记为$\int_L{\left[ \boldsymbol{f}\left( \boldsymbol{p} \right) ^T\mathbf{n}\left( \boldsymbol{p} \right) \right] \cdot dl}$,由于$\mathbf{n}\left( \boldsymbol{p} \right) dl=\boldsymbol{dl}$为弧微分在{\it xyz}坐标的投影向量,所以也可以写为$\int_L{\boldsymbol{f}\left( \boldsymbol{p} \right) ^T\boldsymbol{dl}}$,即:
\begin{align*}
&\int\limits_L{\left[ \boldsymbol{f}\left( \boldsymbol{p} \right) ^T\mathbf{n}\left( \boldsymbol{p} \right) \right] \cdot dl}=\int\limits_L{\boldsymbol{f}\left( \boldsymbol{p} \right) ^T\boldsymbol{dl}} \\
&:=\underset{\lambda \rightarrow 0}{\lim}\sum_{i=1}^n{\left[ \boldsymbol{f}\left( \xi _i,\eta _i,\zeta _i \right) ^T\mathbf{n}\left( \xi _i,\eta _i,\zeta _i \right) \cdot \Delta l_i \right]}
\end{align*}
其中:
\begin{itemize}
    \item $\boldsymbol{f}\left( \boldsymbol{p} \right) $:{\bf 被积函数};
    \item $\mathbf{n}\left( \boldsymbol{p} \right) $:曲线上任一点处的{\bf 单位切向量},也即切线的方向余弦;
    \item $\boldsymbol{dl}=\mathbf{n}\left( \boldsymbol{p} \right) dl$:{\bf 有向弧微分};
    \item $L,t\in \left[ t_1,t_2 \right] $:{\bf 积分曲线}。
\end{itemize}
特别地,当$L$为封闭曲线时,记为$\oint_L{\boldsymbol{f}\left( \boldsymbol{p} \right) ^T\boldsymbol{dl}}$。
\end{definition}

这里要注意,第二类曲线积分是有方向性的,$t_1\rightarrow t_2$的方向是切线的方向,如果取反方向,则写为:
\[
\int\limits_{L^-}{\boldsymbol{f}\left( \boldsymbol{p} \right) ^T\boldsymbol{dl}}
\]

第二类曲线积分的性质:
\begin{itemize}
    \item 线性性:
    \[
    \int\limits_L{\left( a\boldsymbol{f}+b\boldsymbol{g} \right) ^T\boldsymbol{dl}}=a\int\limits_L{\boldsymbol{f}^T\boldsymbol{dl}}+b\int\limits_L{\boldsymbol{g}^T\boldsymbol{dl}}
    \]
    \item 区域可加性:
    \[
    \int\limits_L{\boldsymbol{f}^T\boldsymbol{dl}}=\int\limits_{L_1}{\boldsymbol{f}^T\boldsymbol{dl}}+\int\limits_{L_2}{\boldsymbol{f}^T\boldsymbol{dl}}
    \]
    \item 有向性:
    \[
    \int\limits_{L^-}{\boldsymbol{f}^T\boldsymbol{dl}}=-\int\limits_L{\boldsymbol{f}^T\boldsymbol{dl}}
    \]
\end{itemize}

%============================================================
\subsection{第二类曲线积分的计算}

\begin{theorem}[第二类曲线积分的计算公式]
曲线在某点的切向量:
\[
\boldsymbol{n}\left( \boldsymbol{p}_0 \right) =\left. \left( x'\,\,y'\,\,z' \right) ^T \right|_{t=t_0}=\left( x'\left( t_0 \right) \,\,y'\left( t_0 \right) \,\,z'\left( t_0 \right) \right) ^T
\]
其方向余弦为:
\[
\mathbf{n}\left( \boldsymbol{p}_0 \right) =\frac{\boldsymbol{n}\left( \boldsymbol{p}_0 \right)}{\left\| \boldsymbol{n}\left( \boldsymbol{p}_0 \right) \right\|} =\frac{\left( x'\left( t_0 \right) \,\,y'\left( t_0 \right) \,\,z'\left( t_0 \right) \right) ^T}{\sqrt{\left( x'\left( t_0 \right) \right) ^2+\left( y'\left( t_0 \right) \right) ^2+\left( z'\left( t_0 \right) \right) ^2}}
\]
在第一类曲线积分中,我们知道弧微分$dl=\sqrt{\left( x' \right) ^2+\left( y' \right) ^2+\left( z' \right) ^2}\cdot dt$,结合以上,有向弧微分有:
\begin{align*}
\boldsymbol{dl}&=\mathbf{n}\left( \boldsymbol{p} \right) dl \frac{\left( x'\,\,y'\,\,z' \right) ^T}{\sqrt{\left( x' \right) ^2+\left( y' \right) ^2+\left( z' \right) ^2}} \left( \sqrt{\left( x' \right) ^2+\left( y' \right) ^2+\left( z' \right) ^2}\cdot dt \right) \\
&=\left( x'\,\,y'\,\,z' \right) ^Tdt
\end{align*}
%于是$\boldsymbol{f}\left( \boldsymbol{p} \right) $在$L$上的第二类曲线积分的计算有:
于是$\boldsymbol{f}\left( \boldsymbol{p} \right) =\left( P\left( x,y,z \right) \,\,Q\left( x,y,z \right) \,\,R\left( x,y,z \right) \right) ^T$在$L$上的第二类曲线积分的计算有:
\[
\int\limits_L{\boldsymbol{f}\left( \boldsymbol{p} \right) ^T\boldsymbol{dl}}=\int\limits_L{\left( \begin{array}{c}
	P\\
	Q\\
	R\\
\end{array} \right) ^T\left( \begin{array}{c}
	x'\\
	y'\\
	z'\\
\end{array} \right) dt} =\int_{t_1}^{t_2}{\left( P\cdot x'+Q\cdot y'+R\cdot z' \right) \cdot dt}
\]
\end{theorem}

由于有向弧微分还可以写为$\boldsymbol{dl}=\left( x'\,\,y'\,\,z' \right) ^Tdt=\left( dx\,\,dy\,\,dz \right) ^T$,所以第二类曲线积分还可以写为:
\[
\int\limits_L{\boldsymbol{f}\left( \boldsymbol{p} \right) ^T\boldsymbol{dl}}=\int\limits_L{\left( \begin{array}{c}
	P\\
	Q\\
	R\\
\end{array} \right) ^T\left( \begin{array}{c}
	dx\\
	dy\\
	dz\\
\end{array} \right)}=\int\limits_L{P\cdot dx+Q\cdot dy+R\cdot dz}
\]
这个写法有深刻的物理意义,表明$\boldsymbol{f}\left( \boldsymbol{p} \right) $在$L$上的积分可以分解为$\boldsymbol{f}$的三个坐标分量($P,Q,R$)分别对坐标轴({\it x}、{\it y}和{\it z})积分的代数和。




