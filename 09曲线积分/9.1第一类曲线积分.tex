\section{第一类曲线积分}

本节讨论第一类曲线积分。
所谓“第一类曲线积分”是数量值函数对曲线的积分。

本节要点:
\begin{itemize}
    \item 掌握第一类曲线积分的概念;
    \item 掌握第一类曲线积分的计算,重点掌握弧微分的计算。
\end{itemize}

%============================================================
\subsection{第一类曲线积分的概念}

\begin{definition}[第一类曲线积分]
若三维空间中有曲线$L$,$f\left( \boldsymbol{p} \right) ,\boldsymbol{p}\in \mathbb{R} ^3$为定义在该空间上的三元数量值函数,若下式极限存在,则称极限值为{\bf $f\left( \boldsymbol{p} \right) $在$L$上积分称为第一类曲线积分},记为$\int_L{f\left( \boldsymbol{p} \right) dl}$,即:
\[
\int\limits_L{f\left( \boldsymbol{p} \right) dl}:=\underset{\lambda \rightarrow 0}{\lim}\sum_{i=1}^n{\left[ f\left( \xi _i,\eta _i,\zeta _i \right) \cdot \Delta l_i \right]}
\]
其中:
\begin{itemize}
    \item $f\left( \boldsymbol{p} \right) $:{\bf 被积函数};
    \item $dl$:{\bf 弧微分};
    \item $L$:{\bf 积分曲线}。
\end{itemize}
\end{definition}

\begin{tcolorbox}
实际定义参考“教材\cite{book1}”,这里是一个简写,总体来讲就是将曲线分段,每段上取一个值和这一小段弧长乘积$f\left( \xi _i,\eta _i,\zeta _i \right) \cdot \Delta l_i$,如果和式有极限,则称为$f\left( \boldsymbol{p} \right) $在$L$上有第一类曲线积分。
大致思路和一元函数积分一样。
\end{tcolorbox}

第一类曲线积分的性质:
\begin{itemize}
    \item 线性性:
    \[
    \int\limits_L{\left( af+bg \right) dl}=a\int\limits_L{fdl}+b\int\limits_L{gdl}
    \]
    \item 区域可加性:
    \[
    \int\limits_L{fdl}=\int\limits_{L_1}{fdl}+\int\limits_{L_2}{fdl}
    \]
\end{itemize}

同样,可定义$n$维空间上的数量值函数$f\left( \boldsymbol{p} \right) ,\boldsymbol{p}\in \mathbb{R} ^n$对空间内曲线$L$上积分为$f\left( \boldsymbol{p} \right) $在$n$维空间曲线$L$上的第一类曲线积分,记为$\int_L{f\left( \boldsymbol{p} \right) dl}$。

%============================================================
\subsection{第一类曲线积分的计算}

\begin{theorem}[第一类曲线积分的计算公式]
设三维空间上有曲线$L$,则曲线可以写为:
\[
L:\begin{cases}
	x=x\left( t \right)\\
	y=y\left( t \right)\\
	z=z\left( t \right)\\
\end{cases}
\]
弧微分的表达式:
\begin{align*}
dl&=\sqrt{\left( dx \right) ^2+\left( dy \right) ^2+\left( dz \right) ^2}=\sqrt{\left( x'dt \right) ^2+\left( y'dt \right) ^2+\left( z'dt \right) ^2} \\
&=\sqrt{\left( x' \right) ^2+\left( y' \right) ^2+\left( z' \right) ^2}\cdot dt
\end{align*}
若$f\left( \boldsymbol{p} \right) $在曲线$L$的$\left[ t_1,t_2 \right] $上连续,其中$x\left( t \right) ,y\left( t \right) ,z\left( t \right) $在$\left[ t_1,t_2 \right] $上具有一阶连续导数,则第一类曲线积分可计算为:
\[
\int\limits_L{f\left( \boldsymbol{p} \right) \cdot dl}=\int\limits_L{f\left( x,y,z \right) \cdot \sqrt{\left( x' \right) ^2+\left( y' \right) ^2+\left( z' \right) ^2}\cdot dt}
\]
\end{theorem}




