\section{多元数量值函数积分的概念}

本节讨论多元数量积分的概念、性质、分类。

本节要点:
\begin{itemize}
    \item 掌握数量积分的概念;
    \item 掌握数量积分的性质。
\end{itemize}

%============================================================
\subsection{多元数量值函数积分的概念}

\begin{definition}[数量积分]
假设$\varOmega $为一可度量的有界的封闭几何体,函数$f$是定义在$\varOmega $上的一个有界的数量值函数,将$\varOmega $任意划分为$n$个小部分,并用$\Delta \varOmega _i$表示每一个小部分的度量,在每一个小部分上任取一点$M_i$,作乘积后再作和式:
\[
\sum_{i=1}^n{\left[ f\left( M_i \right) \cdot \Delta \varOmega _i \right]}
\]
记$\lambda =\max \left\{ diameter \ of \ \Delta \varOmega _i \right\} $,如果不论对$\varOmega $如何划分,也不论点$M_i$在$\varOmega $中如何选取,若当$\lambda \rightarrow 0$时上述和式有极限且相等,则称{\bf 函数$f$在$\varOmega $上可积},此极限值称为{\bf 多元数量值函数$f$在几何体$\varOmega $上的积分},记作$\int_{\varOmega}{f\left( M \right) d\varOmega}$,即:
\[
\int\limits_{\varOmega}{f\left( M \right) d\varOmega}:=\underset{\lambda \rightarrow 0}{\lim}\sum_{i=1}^n{\left[ f\left( M_i \right) \cdot \Delta \varOmega _i \right]}
\]
其中:
\begin{itemize}
    \item $\varOmega $:{\bf 积分区域},可以是二维、三维、甚至$n$维,并没有规定维度;
    \item $f$:{\bf 被积函数},是几何体$\varOmega $相应的维度的数量值函数,如果$\varOmega $是二维,则$f$是二元函数,如果$\varOmega $是三维,则$f$是三元函数。
\end{itemize}
\end{definition}

数学上,数量积分依旧是“加权和”的概念,$d\varOmega $对应加权$f\left( M \right) $后,在$\varOmega $范围内的和。

\begin{theorem}[存在性定理]
如果函数$f$在有界闭几何体$\varOmega $上是连续的,则$f$在$\varOmega $上必可积。
\end{theorem}

%============================================================
\subsection{多元数量值函数积分的性质和定理}

线性性:
\[
\int\limits_{\varOmega}{\left[ af\left( M \right) +bg\left( M \right) \right] d\varOmega}=a\int\limits_{\varOmega}{f\left( M \right) d\varOmega}+b\int\limits_{\varOmega}{g\left( M \right) d\varOmega}
\]

区域可加性:
\[
\int\limits_{\varOmega}{f\left( M \right) d\varOmega}=\int\limits_{\varOmega _1}{f\left( M \right) d\varOmega}+\int\limits_{\varOmega _2}{f\left( M \right) d\varOmega}
\]

不等式性质:
\[
f\left( M \right) \leqslant g\left( M \right) \quad \Rightarrow \quad \int\limits_{\varOmega}{f\left( M \right) d\varOmega}\leqslant \int\limits_{\varOmega}{g\left( M \right) d\varOmega}
\]

~

\begin{theorem}[估值定理]
设$m,M$分别是$f$在闭几何体$\varOmega $上的最小值和最大值,则:
\[
m\int\limits_{\varOmega}{d\varOmega}\leqslant \int\limits_{\varOmega}{f\left( M \right) d\varOmega}\leqslant M\int\limits_{\varOmega}{d\varOmega}
\]
\end{theorem}

\begin{theorem}[积分中值定理]
设$f$在闭几何体$\varOmega $上连续,则存在$M_0\in \varOmega $,有:
\[
\int\limits_{\varOmega}{f\left( M \right) d\varOmega}=f\left( M_0 \right) \int\limits_{\varOmega}{d\varOmega}
\]
\end{theorem}

\begin{theorem}[对称性定理]
当积分域$\varOmega $关于$x$对称时,
\begin{itemize}
    \item 若$f$关于$x=0$为奇函数,则$\int_{\varOmega}{f\left( M \right) d\varOmega}=0$;
    \item 若$f$关于$x=0$为偶函数,则$\int_{\varOmega}{f\left( M \right) d\varOmega}=2\int_{\varOmega '}{f\left( M \right) d\varOmega}$,其中$\varOmega '$为$\varOmega $ 位于$x<0$部分。
\end{itemize}
\end{theorem}




