\section{三重积分}

本节讨论三重积分的定义、意义和计算方法。

本节要点:
\begin{itemize}
    \item 掌握三重积分的定义;
    \item 了解三重积分的数学意义和物理意义;
    \item 掌握三重积分在直角坐标下的计算方法。
\end{itemize}

%============================================================
\subsection{三重积分的概念}

\begin{definition}[三重积分]
若$f\left( \boldsymbol{p} \right) $是定义在空间区域$V$上的三元数量值函数,则将{\bf $f\left( \boldsymbol{p} \right) $在$V$上的积分为三重积分}写成:
\begin{align*}
&\iiint\limits_V{f\left( \boldsymbol{p} \right) dv} \\
&V=\left\{ \left( x,y,z \right) \middle| x\in \left[ x_1,x_2 \right] ,y\in \left[ y_1\left( x \right) ,y_2\left( x \right) \right] ,z\in \left[ z_1\left( x,y \right) ,z_2\left( x,y \right) \right] \right\}
\end{align*}
其中:
\begin{itemize}
    \item $f\left( \boldsymbol{p} \right) $:{\bf 被积函数};
    \item $d\sigma$:{\bf 积分微元};
    \item $V$:{\bf 积分区域}。
\end{itemize}
\end{definition}

%============================================================
\subsection{三重积分的几何意义和物理意义}

对比一元积分和二重积分,三重积分在几何上是对一个指定三维空间的“加权和”。
\begin{align*}
&\int_a^b{f\left( x \right) dx} \quad x\in \left[ a,b \right] \\
&\iint\limits_D{f\left( \boldsymbol{p} \right) d\sigma} \quad D=\left\{ \left( x,y \right) \middle| x\in \left[ x_1,x_2 \right] ,y\in \left[ y_1,y_2 \right] \right\} \\
&\iiint\limits_V{f\left( \boldsymbol{p} \right) dv} \quad V=\left\{ \left( x,y,z \right) \middle| x\in \left[ x_1,x_2 \right] ,y\in \left[ y_1,y_2 \right] ,z\in \left[ z_1,z_2 \right] \right\}
\end{align*}

物理上,如果这个“权”是密度,则三重积分表示这个指定三维物体的总质量。
如果这个“权”是电荷密度,则三重积分表示这个指定三维物体的总电荷量。

%============================================================
\subsection{三重积分的直角坐标计算}

三重积分的计算思路和二重一样,将微元和被积函数都一元化。
被积函数一元化的方式依然采用“切片”方法。
在直角坐标下,体积微元可以表示为$dv=dxdydz$,于是:
\begin{align*}
&\iiint\limits_V{f\left( \boldsymbol{p} \right) dv}=\iiint\limits_V{f\left( x,y,z \right) dxdydz} \\
&V=\left\{ \left( x,y,z \right) \middle| x\in \left[ x_1,x_2 \right] ,y\in \left[ y_1\left( x \right) ,y_2\left( x \right) \right] ,z\in \left[ z_1\left( x,y \right) ,z_2\left( x,y \right) \right] \right\}
\end{align*}
三重积分的求解和二重积分类似,只是多了一个“切片”。

计算步骤:
\begin{enumerate}
    \item 先将$x_0,y_0$看成常数,对$z$求从$z_1\left( x_0,y_0 \right) $到$z_2\left( x_0,y_0 \right) $的定积分
    \[
    S\left( x_0,y_0 \right) =\int_{z_1\left( x_0,y_0 \right)}^{z_2\left( x_0,y_0 \right)}{f\left( x_0,y_0,z \right) dz}
    \]
    \item 再将该函数在$D$作二重积分
    \[
    \iint\limits_D{S\left( x,y \right) d\sigma}=\iint\limits_D{\left[ \int_{z_1\left( x,y \right)}^{z_2\left( x,y \right)}{f\left( x,y,z \right) dz} \right] d\sigma}
    \]
    或常写作:
    \[
    \iiint\limits_V{f\left( x,y,z \right) dv}=\iint\limits_D{dxdy\int_{z_1\left( x,y \right)}^{z_2\left( x,y \right)}{f\left( x,y,z \right) dz}}
    \]
\end{enumerate}

通过以上方法,化为了二重积分,后续过程略。
同样,三重积分可以在柱坐标和球坐标下计算。




