\section{数量场和矢量场}

以三维空间为例:
\begin{itemize}
    \item 空间中分布着某种量,如果这个量可以用数量描述,即只有大小没有方向,则称相应的场为{\bf 数量场},记为$f\left( \boldsymbol{p} \right) $,如密度场$\rho \left( \boldsymbol{p} \right) $、温度场$T\left( \boldsymbol{p} \right) $等。数量场$f\left( \boldsymbol{p} \right) $是定义在三维空间$\mathbb{R} ^3$上的数量值函数,其值域是实数$\mathbb{R} $。
    \item 如果这个量需要用矢量描述,即既有大小也有方向,则称相应的场为{\bf 矢量场},记为$\boldsymbol{f}\left( \boldsymbol{p} \right) $,如流速场$\boldsymbol{v}\left( \boldsymbol{p} \right) $、电场$\boldsymbol{E}\left( \boldsymbol{p} \right) $等。矢量场$\boldsymbol{f}\left( \boldsymbol{p} \right) $也是定义在三维空间$\mathbb{R} ^3$上,是一个向量值函数,其值域是矢量$\mathbb{R} ^3$。
\end{itemize}
\begin{align*}
&f\left( \boldsymbol{p} \right) :\mathbb{R} ^3\mapsto \mathbb{R} \\
&\boldsymbol{f}\left( \boldsymbol{p} \right) :\mathbb{R} ^3\mapsto \mathbb{R} ^3
\end{align*}

矢量场有时也写成$\left( P\left( \boldsymbol{p} \right)\,\,Q\left( \boldsymbol{p} \right)\,\,R\left( \boldsymbol{p} \right) \right) ^T$的形式。




