\section{矢量}

本节介绍矢量及其运算法则。
矢量是量化空间的基础,更是多元函数微积分的基础。

本节要点:
\begin{itemize}
    \item 掌握矢量的定义;
    \item 掌握矢量的运算;
    \item 理解矢量内积和外积的几何意义;
    \item 理解矢量之间的关系——平行和垂直。
\end{itemize}

%============================================================
\subsection{矢量和方向余弦}

\begin{definition}[矢量]
我们称既有大小又有方向的量为{\bf 矢量},三维空间中我们采用列向量的表示方式:
\[
\boldsymbol{a}:=\left( \begin{array}{c}
	x\\
	y\\
	z\\
\end{array} \right)
\]
为排版方便,也可记为行向量的转置:
\[
\boldsymbol{a}:=\left( x\,\,y\,\,z \right) ^T
\]
\end{definition}

矢量是空间几何的代数化基础。
空间的点可以通过矢量描述,空间的面和线可以通过矢量方程描述,它们之间的关系可以通过矢量运算判断。

\begin{definition}
我们采用标量积的形式定义{\bf 矢量的模},记作$\left\| \boldsymbol{a} \right\| $,即:
\[
\left\| \boldsymbol{a} \right\| :=\sqrt{\boldsymbol{a}^T\boldsymbol{a}}=\sqrt{x^2+y^2+z^2}
\]
也称{\bf 长度}。特别地,当$\left\| \boldsymbol{a} \right\| =1$时称为{\bf 单位矢量}。

称矢量:
\[
\left( \begin{array}{c}
	\cos \alpha\\
	\cos \beta\\
	\cos \gamma\\
\end{array} \right) =\left( \begin{array}{c}
	\frac{x}{\left\| \boldsymbol{a} \right\|}\\
	\frac{y}{\left\| \boldsymbol{a} \right\|}\\
	\frac{z}{\left\| \boldsymbol{a} \right\|}\\
\end{array} \right) \qquad \alpha ,\beta ,\gamma \in \left[ 0,\pi \right]
\]
为{\bf 矢量$\boldsymbol{a}$的方向余弦},方向余弦中的三个角是矢量$\boldsymbol{a}$分别和{\it xyz}三个轴所成的角度,方向余弦本身为单位矢量。
\end{definition}

我们规定:
\begin{itemize}
    \item 零矢量:$\mathbf{0}=\left( 0\,\,0\,\,0 \right) ^T$
    \item 单位矢量:$\mathbf{a}=\frac{\boldsymbol{a}}{\left\| \boldsymbol{a} \right\|}$
    \item 数乘:$\lambda \boldsymbol{a}=\lambda \left( x\,\,y\,\,z \right) ^T=\left( \lambda x\,\,\lambda y\,\,\lambda z \right) ^T$
    \item 基矢量:$\mathbf{i}=\left( 1\,\,0\,\,0 \right) ^T \quad \mathbf{j}=\left( 0\,\,1\,\,0 \right) ^T \quad \mathbf{k}=\left( 0\,\,0\,\,1 \right) ^T$
\end{itemize}
于是任何矢量均可以用基矢量写成:
\[
\boldsymbol{a}=x\mathbf{i}+y\mathbf{j}+z\mathbf{k}
\]

\begin{tcolorbox}
注意,我们使用粗体表示单位矢量,而不是一般教材用的下标0。
因为我们将$\boldsymbol{p},\boldsymbol{p}_0$用于之后概念的定义,如$x,x_0$用于一元函数微积分中一样。
\end{tcolorbox}

%============================================================
\subsection{矢量的加法}

\begin{definition}[加法]
我们将矢量的{\bf 加法}定义为矢量中所有对应元素相加,记作$\boldsymbol{a}+\boldsymbol{b}$,即:
\begin{align*}
\boldsymbol{a}+\boldsymbol{b}:&=\left( x_{\boldsymbol{a}}\,\,y_{\boldsymbol{a}}\,\,z_{\boldsymbol{a}} \right) ^T+\left( x_{\boldsymbol{b}}\,\,y_{\boldsymbol{b}}\,\,z_{\boldsymbol{b}} \right) ^T \\
&=\left( x_{\boldsymbol{a}}+x_{\boldsymbol{b}} \quad y_{\boldsymbol{a}}+y_{\boldsymbol{b}} \quad z_{\boldsymbol{a}}+z_{\boldsymbol{b}} \right) ^T
\end{align*}
\end{definition}

注意:
\begin{itemize}
    \item 矢量相加仍为矢量;
    \item 几何上的结果是两个矢量构成的平行四边形的对角线。
\end{itemize}

运算法则:
\begin{align*}
&\boldsymbol{a}+\boldsymbol{b}=\boldsymbol{b}+\boldsymbol{a} \\
&\boldsymbol{a}+\left( \boldsymbol{b}+\boldsymbol{c} \right) =\left( \boldsymbol{a}+\boldsymbol{b} \right) +\boldsymbol{c} \\
&\lambda \left( \boldsymbol{a}+\boldsymbol{b} \right) =\lambda \boldsymbol{a}+\lambda \boldsymbol{b} \\
&\lambda \left( \mu \boldsymbol{a} \right) =\left( \lambda \mu \right) \boldsymbol{a} \\
&\left\| \boldsymbol{a}+\boldsymbol{b} \right\| \leqslant \left\| \boldsymbol{a} \right\| +\left\| \boldsymbol{b} \right\|
\end{align*}

%============================================================
\subsection{矢量的内积}

\begin{definition}[内积]
我们采用标量积定义矢量的{\bf 内积},记作$\boldsymbol{a}^T\boldsymbol{b}$,即:
\[
\boldsymbol{a}^T\boldsymbol{b}:=\left( x_{\boldsymbol{a}}\,\,y_{\boldsymbol{a}}\,\,z_{\boldsymbol{a}} \right) \left( x_{\boldsymbol{b}}\,\,y_{\boldsymbol{b}}\,\,z_{\boldsymbol{b}} \right) ^T=x_{\boldsymbol{a}}x_{\boldsymbol{b}}+y_{\boldsymbol{a}}y_{\boldsymbol{b}}+z_{\boldsymbol{a}}z_{\boldsymbol{b}}
\]
也称{\bf 数量积},也可记为$\boldsymbol{a}\cdot \boldsymbol{b}$,于是又称{\bf 点积}。
\end{definition}

\begin{tcolorbox}
有些教材使用$\boldsymbol{a}\cdot \boldsymbol{b}$表示内积,但我们用$\boldsymbol{a}^T\boldsymbol{b}$,好处是$\boldsymbol{a}^T\boldsymbol{b}\cdot dl$这样的被积式清晰无歧义。
\end{tcolorbox}

\begin{theorem}
矢量内积等于矢量的模的积再乘以它们夹角的余弦,即:
\[
\boldsymbol{a}^T\boldsymbol{b}=\left\| \boldsymbol{a} \right\| \left\| \boldsymbol{b} \right\| \cos \theta \qquad \theta \in \left[ 0,\pi \right]
\]
\end{theorem}

矢量内积常用于判断两个矢量的空间关系,相互垂直还是平行。

注意:
\begin{itemize}
    \item 矢量内积的结果为标量;
    \item 几何上,内积可以看成$\boldsymbol{a}$在$\boldsymbol{b}$的方向上的投影值$\left\| \boldsymbol{a} \right\| \cos \theta $和$\left\| \boldsymbol{b} \right\| $的乘积;
    \item 空间上,
    \begin{itemize}
        \item 两个矢量垂直$\Leftrightarrow \theta =\pi /2\Leftrightarrow \boldsymbol{a}^T\boldsymbol{b}=0$,记为$\boldsymbol{a}\bot \boldsymbol{b}$,称为两个矢量{\bf 正交},
        \item 两个矢量平行$\Leftrightarrow \theta =0\mathrm{or}\pi \Leftrightarrow \boldsymbol{a}=\lambda \boldsymbol{b}$,记为$\boldsymbol{a}\parallel \boldsymbol{b}$,称为两个矢量{\bf 平行}(或{\bf 共线})。
    \end{itemize}
    \item 根据内积公式可以得到,
    \begin{itemize}
        \item 若$\boldsymbol{a}^T\boldsymbol{b}>0$,则两个矢量方向基本相同,夹角0°~90°,
        \item 若$\boldsymbol{a}^T\boldsymbol{b}=0$,则两个矢量相互垂直,夹角为90°,
        \item 若$\boldsymbol{a}^T\boldsymbol{b}<0$,则两个矢量方向基本相反,夹角90°~180°。
    \end{itemize}
\end{itemize}

对于单位矢量特别有:
\begin{align*}
&\mathbf{i}^T\mathbf{i}=\mathbf{j}^T\mathbf{j}=\mathbf{k}^T\mathbf{k}=1 \\
&\mathbf{i}^T\mathbf{j}=\mathbf{j}^T\mathbf{k}=\mathbf{k}^T\mathbf{i}=0
\end{align*}

运算法则:
\begin{align*}
&\boldsymbol{a}^T\boldsymbol{b}=\boldsymbol{b}^T\boldsymbol{a} \\
&\boldsymbol{a}^T\left( \boldsymbol{b}+\boldsymbol{c} \right) =\boldsymbol{a}^T\boldsymbol{b}+\boldsymbol{a}^T\boldsymbol{c} \\
&\left( \lambda \boldsymbol{a} \right) ^T\left( \mu \boldsymbol{b} \right) =\left( \lambda \mu \right) \boldsymbol{a}^T\boldsymbol{b} \\
&\boldsymbol{a}^T\boldsymbol{a}=\left\| \boldsymbol{a} \right\| ^2
\end{align*}

\begin{tcolorbox}
矢量内积这个概念源于考察矢量场对边界的扩张或收缩的效果。
\end{tcolorbox}

%============================================================
\subsection{矢量的外积}

\begin{definition}[外积]
我们采用行列式定义{\bf 外积},记作$\boldsymbol{a}\times \boldsymbol{b}$,即:
\begin{align*}
\boldsymbol{a}\times \boldsymbol{b}:&=\left( x_{\boldsymbol{a}}\,\,y_{\boldsymbol{a}}\,\,z_{\boldsymbol{a}} \right) \times \left( x_{\boldsymbol{b}}\,\,y_{\boldsymbol{b}}\,\,z_{\boldsymbol{b}} \right) ^T\\
&=\left| \begin{matrix}
	\mathbf{i}&		\mathbf{j}&		\mathbf{k}\\
	x_{\boldsymbol{a}}&		y_{\boldsymbol{a}}&		z_{\boldsymbol{a}}\\
	x_{\boldsymbol{b}}&		y_{\boldsymbol{b}}&		z_{\boldsymbol{b}}\\
\end{matrix} \right|=\left( \begin{array}{c}
	y_{\boldsymbol{a}}z_{\boldsymbol{b}}-z_{\boldsymbol{a}}y_{\boldsymbol{b}}\\
	z_{\boldsymbol{a}}x_{\boldsymbol{b}}-x_{\boldsymbol{a}}z_{\boldsymbol{b}}\\
	x_{\boldsymbol{a}}y_{\boldsymbol{b}}-y_{\boldsymbol{a}}x_{\boldsymbol{b}}\\
\end{array} \right)
\end{align*}
又称{\bf 叉积}、{\bf 矢量积}、{\bf 向量积}。
\end{definition}

\begin{theorem}
矢量外积的模等于矢量的模的积再乘以它们夹角的正弦,即:
\[
\left\| \boldsymbol{a}\times \boldsymbol{b} \right\| =\left\| \boldsymbol{a} \right\| \left\| \boldsymbol{b} \right\| \sin \theta \qquad \theta \in \left[ 0,\pi \right]
\]
\end{theorem}

注意:
\begin{itemize}
    \item 矢量外积的结果还是为矢量;
    \item 外积的方向垂直于$\boldsymbol{a},\boldsymbol{b}$构成的平面,符合“右手法则”,右手握拳由$\boldsymbol{a}$转向$\boldsymbol{b}$,拇指方向为外积方向,故也称为{\bf 法向量};
    \item 外积的大小为$\boldsymbol{a},\boldsymbol{b}$构成的平行四边形面积,当$\left\| \boldsymbol{a}\times \boldsymbol{b} \right\| =0$时,表示$\boldsymbol{a},\boldsymbol{b}$平行;
    \item 矢量外积常用于构建一个平面的法向矢量,当然也可以判断两个矢量是否平行,但不常用。
\end{itemize}

\begin{tcolorbox}
矢量外积的方向的“右手法则”是直角坐标系的“右手法则”的必然结果。
\end{tcolorbox}

对于单位矢量特别有:
\begin{align*}
&\mathbf{i}\times \mathbf{i}=\mathbf{j}\times \mathbf{j}=\mathbf{k}\times \mathbf{k}=0 \\
&\mathbf{i}\times \mathbf{j}=\mathbf{k} \\
&\mathbf{j}\times \mathbf{k}=\mathbf{i} \\
&\mathbf{k}\times \mathbf{i}=\mathbf{j}
\end{align*}

运算法则:
\begin{align*}
&\boldsymbol{a}\times \boldsymbol{b}=-\boldsymbol{b}\times \boldsymbol{a} \\
&\boldsymbol{a}\times \left( \boldsymbol{b}+\boldsymbol{c} \right) =\boldsymbol{a}\times \boldsymbol{b}+\boldsymbol{a}\times \boldsymbol{c} \\
&\boldsymbol{a}\times \boldsymbol{a}=\mathbf{0}
\end{align*}

\begin{tcolorbox}
矢量外积这个概念源于考察矢量场对区域自旋的效果。
\end{tcolorbox}

%============================================================
\subsection{内积和外积的比较}

考察内积和外积,内积和外积的大小的平方和是以$\boldsymbol{a},\boldsymbol{b}$长度构成的长方形的面积的平方。
\begin{align*}
&\because \begin{cases}
	\boldsymbol{a}^T\boldsymbol{b}=\left\| \boldsymbol{a} \right\| \left\| \boldsymbol{b} \right\| \cos \theta\\
	\left\| \boldsymbol{a}\times \boldsymbol{b} \right\| =\left\| \boldsymbol{a} \right\| \left\| \boldsymbol{b} \right\| \sin \theta\\
\end{cases} \\
&\therefore \left( \boldsymbol{a}^T\boldsymbol{b} \right) ^2+\left\| \boldsymbol{a}\times \boldsymbol{b} \right\| ^2=\left\| \boldsymbol{a} \right\| ^2\left\| \boldsymbol{b} \right\| ^2
\end{align*}
该关系式说明,内积和外积是一对关系,是一个本源分裂出来的两个概念。
物理意义上,是矢量场对区域的两个作用(收扩作用和自旋作用)的反映。
矢量场对一个区域的膨胀或收缩的效应,会用到内积运算,对该区域的自旋效应,会用到外积运算。




