\section{直角坐标系}

本节介绍直角坐标系。

%============================================================
\subsection{直角坐标系的概念}

通常,三维空间的直角坐标系的方向用右手规定。
空间中的点可以用一个有序三元实数组表示,同时也可以用一个{\bf 3维列向量}(本笔记也称{\bf 矢量})表示,于是下面两个表示方法等价:
\begin{itemize}
    \item 点$\mathrm{P}\left( x,y,z \right) $,有时也可以简单写成$\mathrm{P}$;
    \item 矢量$\boldsymbol{p}=\left( x\,\,y\,\,z \right) ^T$。
\end{itemize}
其中,$x,y,z\in \mathbb{R} $表示点$\mathrm{P}$的{\it xyz}轴坐标。

我们采用欧几里得范数(也即内积引导的范数)定义两个点$\mathrm{P},\mathrm{Q}$的距离,记为$\left\| \mathrm{PQ} \right\| $,有:
\begin{align*}
\left\| \mathrm{PQ} \right\| :&=\left\| \boldsymbol{p}-\boldsymbol{q} \right\| \\
&=\sqrt{\left( \boldsymbol{p}-\boldsymbol{q} \right) ^T\left( \boldsymbol{p}-\boldsymbol{q} \right)}=\sqrt{\left( x_{\boldsymbol{p}}-x_{\boldsymbol{q}} \right) ^2+\left( y_{\boldsymbol{p}}-y_{\boldsymbol{q}} \right) ^2+\left( z_{\boldsymbol{p}}-z_{\boldsymbol{q}} \right) ^2}
\end{align*}

\begin{tcolorbox}
直角坐标系的右手方向,只是规定!规定!规定!纯人为规定!
\end{tcolorbox}




