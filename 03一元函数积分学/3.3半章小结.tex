\section{半章小结}

至此,导数、微分、定积分三者关系确立:
\begin{itemize}
    \item 导数是整个微积分的基础;
    \item 微分定理给出了微增量和导数的关系$df\left( x \right) =f'\left( x \right) \cdot dx$;
    \item 牛顿莱布尼兹公式给出了累积量和导数的关系$\int_a^b{f\left( x \right) dx}=F\left( b \right) -F\left( a \right)$。
\end{itemize}

微分和积分所表达的是一个事物的两个方面,这个事物就是光滑。
微分是从“变化”的角度看光滑,积分是从“累积”的角度看光滑。
所以,一般地,微分里的概念和定理在积分里都会有对应的表达。
也正是微分和积分的这种对立统一关系,使得在解决实际问题时才有“微元法”这个普遍意义上的数学方法。

至此,微积分的基本思想建立完成。
我们需要从“微”和“积”两个方面理解微分和积分的对立,更要从逻辑论和认识论上理解微分和积分这一对矛盾体的对立统一。
这对矛盾体统一在光滑,是光滑衍生出来的在微和积方面的两个对立。




