\section{习题}

\begin{exercise}
计算下列导数和极限:
\begin{enumerate}
    \item $\frac{d}{dx}\int_0^{x^2}{\sqrt{1+t^2}dt}$
    \item $\frac{d}{dx}\int_{\sin x}^{\cos x}{\cos \left( \pi t^2 \right) dt}$
    \item $\underset{x\rightarrow 0}{\lim}\frac{\int_0^x{\cos t^2dt}}{x}$
    \item $\underset{x\rightarrow +\infty}{\lim}\frac{\left( \int_0^x{e^{t^2}dt} \right) ^2}{\int_0^x{e^{2t^2}dt}}$
\end{enumerate}
\end{exercise}

解:

1.
\[
\frac{d}{dx}\int_0^{x^2}{\sqrt{1+t^2}dt}=\sqrt{1+\left( x^2 \right) ^2}\cdot \left( x^2 \right) '=2x\sqrt{1+x^4}
\]

2.
\begin{align*}
&\frac{d}{dx}\int_{\sin x}^{\cos x}{\cos \left( \pi t^2 \right) dt} \\
&=\frac{d}{dx}\int_{\sin x}^0{\cos \left( \pi t^2 \right) dt}+\frac{d}{dx}\int_0^{\cos x}{\cos \left( \pi t^2 \right) dt} \\
&=-\cos \left( \pi \sin ^2x \right) \cdot \cos x-\cos \left( \pi \cos ^2x \right) \cdot \sin x
\end{align*}

3.
\[
\underset{x\rightarrow 0}{\lim}\frac{\int_0^x{\cos t^2dt}}{x}=\underset{x\rightarrow 0}{\lim}\frac{\cos x^2\cdot 1}{1}=1
\]

4.
\begin{align*}
\underset{x\rightarrow +\infty}{\lim}\frac{\left( \int_0^x{e^{t^2}dt} \right) ^2}{\int_0^x{e^{2t^2}dt}}&=\underset{x\rightarrow +\infty}{\lim}\frac{2\left( \int_0^x{e^{t^2}dt} \right) \cdot e^{x^2}}{e^{2x^2}} \\
&=2\underset{x\rightarrow +\infty}{\lim}\frac{\int_0^x{e^{t^2}dt}}{e^{x^2}}=2\underset{x\rightarrow +\infty}{\lim}\frac{e^{x^2}\cdot 1}{2xe^{x^2}}=0
\end{align*}

\begin{tcolorbox}
主要是理解含积分复合函数的求导。
\end{tcolorbox}

~

\begin{exercise}
若方程$\int_0^y{e^tdt}+\int_0^x{\cos tdt}=0$确定函数$y=f\left( x \right) $,求$\frac{dy}{dx}$。
\end{exercise}

解:

大体思路是隐函数和积分复合函数的混合体,令隐函数$F\left( x,y \right) =\int_0^y{e^tdt}+\int_0^x{\cos tdt}=0$,则:
\begin{align*}
&\because \begin{cases}
	F_x=\frac{d}{dx}\left( \int_0^y{e^tdt}+\int_0^x{\cos tdt} \right) =e^y\cdot 0+\cos x\cdot 1=\cos x\\
	F_y=\frac{d}{dy}\left( \int_0^y{e^tdt}+\int_0^x{\cos tdt} \right) =e^y\cdot 1+\cos x\cdot 0=e^y\\
\end{cases} \\
&\therefore \frac{dy}{dx}=-\frac{F_x}{F_y}=-\frac{\cos y}{e^y}
\end{align*}

~

\begin{exercise}
若有一物体做直线运动,设其运动规律是$s=ct^3$,媒质阻力与速度的平方成正比,求物体由$s=0$运动至$s=b$时,克服阻力作的功。
\end{exercise}

解:

做功微元是$dW=F\cdot ds$,对路程积分,阻力$F$对路程$s$的表达式:
\[
F\left( s \right) =kv^2=k\left( \frac{ds}{dt} \right) ^2=k\left( 3ct^2 \right) ^2=9kc^2\left( \frac{s}{c} \right) ^{\frac{4}{3}}
\]
所以得到克服阻力作的功:
\[
W=\int_0^b{dW}=\int_0^b{9kc^2\left( \frac{s}{c} \right) ^{\frac{4}{3}}\cdot ds}=\frac{27}{7}kc^{\frac{2}{3}}b^{\frac{7}{3}}
\]




