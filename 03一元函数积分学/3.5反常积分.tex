\section{反常积分}

定积分的要求是积分区间是有限区间,被积函数是有界函数。
如果这两条中任何一条不满足,定积分的概念就不适用。

本节要点:
\begin{itemize}
    \item 了解反常积分;
    \item 了解伽马函数和贝塔函数。
\end{itemize}

%============================================================
\subsection{无穷区间上的反常积分}

\begin{definition}[无穷型反常积分]
设$f\left( x \right) $在无穷区间$\left[ a,+\infty \right] $上连续,则称极限$\underset{b\rightarrow +\infty}{\lim}\int_a^b{f\left( x \right) dx},b>a$为{\bf $f\left( x \right) $在无穷区间$\left[ a,+\infty \right] $上的反常积分},记为$\int_a^{+\infty}{f\left( x \right) dx}$,即:
\[
\int_a^{+\infty}{f\left( x \right) dx}:=\underset{b\rightarrow +\infty}{\lim}\int_a^b{f\left( x \right) dx} \qquad b>a
\]
如果该极限存在,称该{\bf 反常积分收敛},反之称该{\bf 反常积分发散}。

同理,设$f\left( x \right) $在无穷区间$\left[ -\infty ,b \right] $上连续,则称$\underset{a\rightarrow -\infty}{\lim}\int_a^b{f\left( x \right) dx},b>a$为{\bf $f\left( x \right) $在无穷区间$\left[ -\infty ,b \right] $上的反常积分},记为$\int_{-\infty}^b{f\left( x \right) dx}$,即:
\[
\int_{-\infty}^b{f\left( x \right) dx}:=\underset{a\rightarrow -\infty}{\lim}\int_a^b{f\left( x \right) dx} \qquad b>a
\]

综合以上,设$f\left( x \right) $在$\left[ -\infty ,+\infty \right] $上连续,定义{\bf $f\left( x \right) $在$\left[ -\infty ,+\infty \right] $上的反常积分}为:
\[
\int_{-\infty}^{+\infty}{f\left( x \right) dx}:=\int_{-\infty}^0{f\left( x \right) dx}+\int_0^{+\infty}{f\left( x \right) dx}
\]
\end{definition}

%============================================================
\subsection{无界函数的反常积分}

\begin{definition}[开区间型反常积分]
设$f\left( x \right) $在$\left[ a,b \right) $上连续,在$b$点的左邻域内无界,则称极限$\underset{\varepsilon \rightarrow 0^+}{\lim}\int_a^{b-\varepsilon}{f\left( x \right) dx}$为{\bf $f\left( x \right) $在$\left[ a,b \right) $上的反常积分}:
\[
\int_a^b{f\left( x \right) dx}:=\underset{\varepsilon \rightarrow 0^+}{\lim}\int_a^{b-\varepsilon}{f\left( x \right) dx}
\]
如果该极限存在,称该{\bf 反常积分收敛},反之称该{\bf 反常积分发散}。

同理,设$f\left( x \right) $在$\left( a,b \right] $上连续,在$a$点的右邻域内无界,则称极限$\underset{\varepsilon \rightarrow 0^+}{\lim}\int_{a+\varepsilon}^b{f\left( x \right) dx}$为{\bf $f\left( x \right) $在$\left( a,b \right] $上的反常积分}:
\[
\int_a^b{f\left( x \right) dx}:=\underset{\varepsilon \rightarrow 0^+}{\lim}\int_{a+\varepsilon}^b{f\left( x \right) dx}
\]

综合以上,设$f\left( x \right) $在$\left[ a,b \right] -\left\{ c \right\} $上连续,在$c$点的某邻域内无界,定义{\bf $f\left( x \right) $在$\left[ a,b \right] $上的反常积分}为:
\[
\int_a^b{f\left( x \right) dx}:=\int_a^c{f\left( x \right) dx}+\int_c^b{f\left( x \right) dx}
\]
\end{definition}

反常积分不是定积分,虽然采用定积分的符号,而是定积分的极限。
无穷区间的反常积分是{\it x}轴上无限,无界函数的反常积分对应几何上是{\it y}轴上的无限。

%============================================================
\subsection{伽马函数和贝塔函数}

\begin{definition}[伽马函数($\Gamma $函数)]
\[
\Gamma \left( x \right) :=\int_0^{+\infty}{e^{-t}t^{x-1}dt} \qquad x>0
\]
\end{definition}

$\Gamma $函数的性质,证明略:
\begin{itemize}
    \item $\Gamma \left( 1 \right) =1$;
    \item $\Gamma \left( x \right) =\left( x-1 \right) \Gamma \left( x-1 \right) $,特别的取整数$n$时有$\Gamma \left( n \right) =\left( n-1 \right) !$;
    \item $\Gamma \left( x \right) \Gamma \left( 1-x \right) =\frac{\pi}{\sin \pi x}$;
    \item $\underset{x\rightarrow 0^+}{\lim}\Gamma \left( x \right) =+\infty $;
    \item $\Gamma \left( \frac{1}{2} \right) =\sqrt{\pi}$。
\end{itemize}

\begin{definition}[贝塔函数($\mathrm{B}$函数)]
\[
\mathrm{B}\left( \alpha ,\beta \right) :=\int_0^1{x^{\alpha -1}\left( 1-x \right) ^{\beta -1}dx} \qquad \alpha ,\beta \in \left( 0,1 \right)
\]
\end{definition}

$\mathrm{B}$函数的性质,证明略。
\begin{itemize}
    \item $\mathrm{B}\left( 1,1 \right) =1$;
    \item $\mathrm{B}\left( \alpha ,\beta \right) =\mathrm{B}\left( \beta ,\alpha \right) $,对称性;
    \item $\mathrm{B}\left( \alpha +1,\beta +1 \right) =\frac{\alpha \beta}{\left( \alpha +\beta \right) \left( \alpha +\beta +1 \right)}\mathrm{B}\left( \alpha ,\beta \right) $,特别地,取整数$m,n$时有$\mathrm{B}\left( m,n \right) =\frac{\left( m-1 \right) !\left( n-1 \right) !}{\left( m+n-1 \right) !}$;
    \item $\mathrm{B}\left( \alpha ,\beta \right) =\frac{\Gamma \left( \alpha \right) \Gamma \left( \beta \right)}{\Gamma \left( \alpha +\beta \right)}$。
\end{itemize}

\begin{tcolorbox}
$\Gamma $函数和$\mathrm{B}$函数将以往以自然数为定义域的函数(如阶乘)扩展到了实数域,所以特别在数理统计有着广泛的应用。
\end{tcolorbox}



