\section{函数项级数}

本节介绍函数项级数。

本节要点:
\begin{itemize}
    \item 了解函数项级数的概念;
    \item 了解收敛域的概念。
\end{itemize}

%============================================================
\subsection{函数项级数的概念}

\begin{definition}[函数项级数]
若在给定的一个区间$I$上有函数列:
\[
u_1\left( x \right) ,u_2\left( x \right) ,\cdots ,u_n\left( x \right) ,\cdots
\]
则称:
\[
\sum_{n=1}^{\infty}{u_n\left( x \right)}:=u_1\left( x \right) +u_2\left( x \right) +\cdots +u_n\left( x \right) +\cdots
\]
为{\bf 定义在区间$I$上的函数项级数}。
\end{definition}

函数项级数对应的是常数项级数。
区别在于,常数项级数的一般项是一个固定的数,一般项具有一定的规律,函数项级数的一般项是函数,这些函数符合一定的规律。

\begin{definition}
设函数项级数$\sum_{n=1}^{\infty}{u_n\left( x \right)}$,对于点$x_0\in I$:
\begin{itemize}
    \item 若$\sum_{n=1}^{\infty}{u_n\left( x_0 \right)}$收敛,则称$x_0$为{\bf 该函数项级数的收敛点},全体收敛点的集合称为{\bf 函数项级数的收敛域};
    \item 若$\sum_{n=1}^{\infty}{u_n\left( x_0 \right)}$发散,则称$x_0$为{\bf 函数项级数的发散点}。
\end{itemize}
并称
\[
S\left( x \right) :=\sum_{n=1}^{\infty}{u_n\left( x \right)} \qquad x\in D
\]
为{\bf 函数项级数$\sum_{n=1}^{\infty}{u_n\left( x \right)}$的和函数},其中$D$为函数项级数的收敛域。
同时称
\[
R_n\left( x \right) :=S\left( x \right) -S_n\left( x \right) =\sum_{i=n+1}^{\infty}{u_i\left( x \right)} \qquad x\in D
\]
为{\bf 函数项级数的余项}。
\end{definition}

和函数及余项都是一个关于$x$的函数。

~

\begin{example}
求级数$\sum_{n=1}^{\infty}{x^{n-1}}$的和函数。
\end{example}

解:

这是首项1,公比$x$的等比数列的级数。
根据之前讨论,只有当$\left| x \right|<1$时,级数收敛。
所以其和函数为:
\[
S\left( x \right) =\underset{n\rightarrow \infty}{\lim}S_n\left( x \right) =\underset{n\rightarrow \infty}{\lim}\sum_{i=1}^n{x^{i-1}}=\frac{1}{1-x} \qquad x\in \left( -1,1 \right)
\]
这说明,对于函数$f\left( x \right) =\frac{1}{1-x},x\in \left( -1,1 \right) $,我们可以用数列$1+x+x^2+x^3+\cdots $去近似,而且累加的项越多,近似度越高。

%============================================================
\subsection{函数项级数的意义}

函数项级数理论的本身意义在于数学,通过级数的一致收敛性,可以判断隐函数的存在、微分方程解的存在性和唯一性等。
本笔记不做展开。

其工程意义更多地在于具象出来的两个:幂级数和三角级数。




