\section{交错级数}

本节介绍交错级数。

本节要点:
\begin{itemize}
    \item 了解交错级数的概念;
    \item 掌握交错级数敛散性的判断。
\end{itemize}

%============================================================
\subsection{交错级数的概念}

\begin{definition}[交错级数]
若$u_n>0$,则称级数
\[
\sum_{n=1}^{\infty}{\left( -1 \right) ^nu_n} \quad \text{或} \quad \sum_{n=1}^{\infty}{\left( -1 \right) ^{n-1}u_n}
\]
为{\bf 交错级数}。
\end{definition}

简而言之,交错级数就是正项级数的基础上其一般项一正一负。

%============================================================
\subsection{交错级数的性质}

\begin{theorem}[莱布尼兹判别法]
若交错级数$\sum_{n=1}^{\infty}{\left( -1 \right) ^nu_n}$满足$u_n\geqslant u_{n+1}$且$\underset{n\rightarrow \infty}{\lim}u_n=0$,则级数收敛,且其和有$S\leqslant u_1$,余项有$\left| R_n \right|\leqslant u_{n+1}$。
\end{theorem}

交错级数由于一般项的正负交替,使得收敛性加强。

%============================================================
\subsection{例}

\begin{example}
交错调和级数$\sum_{n=1}^{\infty}{\left( -1 \right) ^{n-1}\frac{1}{n}}$。
\end{example}

解:

由于
\begin{align*}
&\frac{1}{n}>\frac{1}{n+1} \\
&\underset{n\rightarrow \infty}{\lim}\frac{1}{n}=0
\end{align*}
所以级数收敛。




