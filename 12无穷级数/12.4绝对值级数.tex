\section{绝对值级数}

本节介绍绝对值级数。

本节要点:
\begin{itemize}
    \item 了解绝对值级数的概念;
    \item 掌握绝对值级数敛散性的判断;
    \item 了解绝对值级数的性质。
\end{itemize}

%============================================================
\subsection{绝对值级数的概念}

\begin{definition}[绝对值级数]
设$u_n$是任意实数,称级数$\sum_{n=1}^{\infty}{u_n}$为{\bf 任意项级数},对应地,称级数$\sum_{n=1}^{\infty}{\left| u_n \right|}$为其对应的{\bf 绝对值级数}。
\end{definition}

\begin{definition}[绝对收敛]
设任意项级数$\sum_{n=1}^{\infty}{u_n}$,若对应的绝对值级数$\sum_{n=1}^{\infty}{\left| u_n \right|}$收敛,则称{\bf 级数$\sum_{n=1}^{\infty}{u_n}$绝对收敛},若级数$\sum_{n=1}^{\infty}{u_n}$收敛,但绝对值级数$\sum_{n=1}^{\infty}{\left| u_n \right|}$却发散,则称{\bf 级数$\sum_{n=1}^{\infty}{u_n}$条件收敛}。
\end{definition}

\begin{theorem}
$\sum_{n=1}^{\infty}{\left| u_n \right|} \text{收敛} \Rightarrow \sum_{n=1}^{\infty}{u_n} \text{收敛}$。
\end{theorem}

“绝对收敛”这个概念的核心在“取绝对值”。
绝对收敛是一个要求非常高的收敛条件,应用非常广泛。

%============================================================
\subsection{绝对值级数的性质}

\begin{theorem}[可交换性]
绝对收敛级数不因改变项的位置而改变它的和。
\end{theorem}

\begin{theorem}[柯西乘积定理]
设两个级数$\sum_{n=1}^{\infty}{u_n},\sum_{n=1}^{\infty}{v_n}$均收敛,和分别为$S,T$,则它们的{\bf 柯西乘积}
\begin{align*}
\left( \sum_{n=1}^{\infty}{u_n} \right) \cdot \left( \sum_{n=1}^{\infty}{v_n} \right) &=\sum_{n=1}^{\infty}{\left( \sum_{m=1}^{\infty}{u_mv_{n-m+1}} \right)} \\
&=\sum_{n=1}^{\infty}{\left( u_1v_n+u_2v_{n-1}+\cdots \right)}
\end{align*}
绝对收敛,其和为$ST$。
\end{theorem}

有限项和,其值不随项的位置的变化而变化,通俗来讲就是符合加法交换律。
但如果是无限项和,即级数,如果是绝对收敛的级数,这个结论是对的,满足交换律。
但如果是条件收敛的级数,如果适当改变各项的位置,可能发散,也可能收敛到事先指定的任何值,即不满足交换律。
问题出在级数是无限项的和。

\begin{tcolorbox}
由于绝对收敛满足交换律,或者说与项所在的位置无关,所以在概率论中广泛使用绝对收敛这个条件。
\end{tcolorbox}




