\section{正项级数}

本节介绍正项级数。

本节要点:
\begin{itemize}
    \item 了解正项级数的概念;
    \item 掌握正项级数敛散性的判断。
\end{itemize}

%============================================================
\subsection{正项级数的概念}

\begin{definition}[正项级数]
称$u_n\geqslant 0$的级数为{\bf 正项级数}。
\end{definition}

%============================================================
\subsection{正项级数的定理}

\begin{theorem}
正项级数$\sum_{n=1}^{\infty}{u_n}$收敛$\Leftrightarrow $部分和数列$\left\{ S_n \right\} $有界,且这个界就是级数和。
\end{theorem}

\begin{theorem}[达朗贝尔判别法]
若正项级数有$\underset{n\rightarrow \infty}{\lim}\frac{u_{n+1}}{u_n}=\rho <1$,则级数收敛,若$\rho >1$,则级数发散,若$\rho =1$,则级数敛散性不定。
\end{theorem}

该定理描述的级数就是等比数列。

\begin{theorem}[柯西判别法]
若正项级数有$\underset{n\rightarrow \infty}{\lim}\sqrt[n]{u_n}=\rho <1$,则级数收敛,若$\rho >1$,则级数发散,若$\rho =1$,则级数敛散性不定。
\end{theorem}

\begin{theorem} \label{th_12_2_1}
设两个正项级数$\sum_{n=1}^{\infty}{u_n},\sum_{n=1}^{\infty}{v_n}$,若对应的一般项$u_n\leqslant v_n$,则:$\sum_{n=1}^{\infty}{v_n}$收敛$\Rightarrow \sum_{n=1}^{\infty}{u_n}$收敛,$\sum_{n=1}^{\infty}{u_n}$发散$\Rightarrow \sum_{n=1}^{\infty}{v_n}$发散。
\end{theorem}

\begin{theorem} \label{th_12_2_2}
设两个正项级数$\sum_{n=1}^{\infty}{u_n},\sum_{n=1}^{\infty}{v_n}$,若对应的一般项有极限$\underset{n\rightarrow \infty}{\lim}\frac{u_n}{v_n}=C\ne 0$且为一固定常数,则两个级数的敛散性一致,即若两级数的一般项的变化规则相同,则具有一致的敛散性。
\end{theorem}

由于正项级数不会出现“正负振荡”,所以归根结底就是考察一般项。
达朗贝尔(比值)判别法和柯西(根值)判别法是通过考察级数自身一般项判断级数的敛散性。
定理\ref{th_12_2_1}、\ref{th_12_2_2}是通过与一个已知敛散性的级数对比一般项判断原级数的敛散性。

%============================================================
\subsection{例}

\begin{example}
判断p级数$\sum_{n=1}^{\infty}{\frac{1}{n^p}}$。
\end{example}

解:

当$p\leqslant 1$时,一般项$\frac{1}{n^p}\geqslant \frac{1}{n}$,加之调和级数发散,所以原级数发散。
当$p>1$时:
\begin{align*}
&\because \frac{1}{n^p}\leqslant \frac{1}{x^p} \quad x\in \left[ n-1,n \right] \\
&\therefore \frac{1}{n^p}=\int_{n-1}^n{\frac{1}{n^p}dx}\leqslant \int_{n-1}^n{\frac{1}{x^p}dx}=\frac{1}{p-1}\left[ \frac{1}{\left( n-1 \right) ^{p-1}}-\frac{1}{n^{p-1}} \right]
\end{align*}
考察级数:
\begin{align*}
S=\sum_{n=1}^{\infty}{\frac{1}{p-1}\left[ \frac{1}{\left( n-1 \right) ^{p-1}}-\frac{1}{n^{p-1}} \right]}=\underset{n\rightarrow \infty}{\lim}\frac{1}{p-1}\left( 1-\frac{1}{n^{p-1}} \right) =\frac{1}{p-1}
\end{align*}
所以,当$p>1$时原级数收敛。
综上所述,$p$级数$p>1$时收敛,$p\leqslant 1$时发散。
更一般地,级数:
\[
\sum_{n=1}^{\infty}{\frac{1}{\left( a+bn \right) ^p}} \qquad a>0,b>0
\]
$p>1$时收敛,$p\leqslant 1$时发散。

~

\begin{example}
判断级数$\sum_{n=1}^{\infty}{\frac{1}{\sqrt{n\left( n+1 \right)}}}$。
\end{example}

解:
\begin{align*}
&\because n^2<n\left( n+1 \right) <\left( n+1 \right) ^2 \\
&\therefore \frac{1}{n+1}<\frac{1}{\sqrt{n\left( n+1 \right)}}<\frac{1}{n}
\end{align*}
虽然好于调和级数,但还是发散。
注意,该级数不能用达朗贝尔判别法:
\[
\underset{n\rightarrow \infty}{\lim}\frac{\sqrt{n\left( n+1 \right)}}{\sqrt{\left( n+1 \right) \left( n+2 \right)}}=1
\]




