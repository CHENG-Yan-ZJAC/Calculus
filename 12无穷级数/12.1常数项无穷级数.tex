\section{常数项无穷级数}

本节介绍无穷级数的基本概念和性质。

本节要点:
\begin{itemize}
    \item 掌握级数、部分和、级数和、余项的概念;
    \item 掌握级数的性质;
    \item 了解三个重要级数(等比级数、调和级数、无穷级数)的敛散性。
\end{itemize}

%============================================================
\subsection{常数项级数的概念}

\begin{definition}[常数项级数]
对于一个给定的数列$u_1,u_2,\cdots ,u_n,\cdots $,将它们的和称为{\bf 常数项无穷级数},简称为{\bf 常数项级数},记为$\sum_{n=1}^{\infty}{u_n}$,即:
\[
\sum_{n=1}^{\infty}{u_n}:=u_1+u_2+\cdots +u_n+\cdots
\]
其中,$u_n$称为常数项级数的{\bf 一般项}。
\end{definition}

对于级数,我们主要研究两个问题:1)级数是否收敛,发散的级数是没有使用价值的;2)收敛值是多少,即能不能代替目标常数。

\begin{definition}
我们称数列$u_1,u_2,\cdots ,u_n,\cdots $前$n$项之和为{\bf 级数$\sum_{n=1}^{\infty}{u_n}$的部分和},记为$S_n$,即:
\[
S_n:=\sum_{i=1}^n{u_i}
\]
如果级数$\sum_{n=1}^{\infty}{u_n}$的部分和$S_n$存在极限,则称该级数{\bf 收敛},并称该极限为级数$\sum_{n=1}^{\infty}{u_n}$的{\bf 级数和},记为$S$,即:
\[
S:=\underset{n\rightarrow \infty}{\lim}S_n
\]
反之,如果部分和没有极限,称该级数{\bf 发散}。
当级数$\sum_{n=1}^{\infty}{u_n}$收敛时,级数和与部分和之间的差称为{\bf 级数的余项},记为$R_n$,即:
\[
R_n:=S-S_n=\sum_{i=n+1}^{\infty}{u_i}
\]
\end{definition}

注意这些概念的关系。
级数$\sum_{n=1}^{\infty}{u_n}$是一个无穷数列的和,对收敛性没有强制要求。
部分和$S_n$是数列前$n$项的和。
级数和$S$是$S_n$的极限。
所以,只有收敛的级数才有级数和,部分和是用来引出或计算级数和的中间概念,实际使用中可以通过计算部分和的极限判断级数的敛散性。
余项则可以用来判断级数的敛散性。

%============================================================
\subsection{常数项级数的性质和定理}

常数项级数的性质:
\begin{itemize}
    \item 若级数$\sum_{n=1}^{\infty}{u_n}$收敛于$S$,则级数$\sum_{n=1}^{\infty}{ku_n}$收敛于$kS$,即一般项等比的两个级数的敛散性一致;
    \item 若两个级数均收敛,则对应项之和构成的级数也收敛,即
    \[
    \left. \begin{array}{r}
        \sum_{n=1}^{\infty}{u_n}=S\\
        \sum_{n=1}^{\infty}{v_n}=T\\
    \end{array} \right\} \Rightarrow \sum_{n=1}^{\infty}{\left( u_n\pm v_n \right)}=S\pm T
    \]
    \item 级数增减前有限项,其敛散性不变,但增减项后的新级数的和一般会变,即级数的余项决定其敛散性;
    \item 收敛级数按一定规律加括号后形成的新级数仍收敛,且收敛于相同值,但注意,去掉括号则不一定仍收敛;
    \item 如果加括号形成的新级数发散,则原级数也发散。
\end{itemize}

\begin{theorem} \label{th_12_1_1}
$\sum_{n=1}^{\infty}{u_n} \text{收敛} \Rightarrow \underset{n\rightarrow \infty}{\lim}u_n=0$,即收敛级数的一般项必趋于0。
\end{theorem}

注意,反之不成立!一般项的收敛必须达到一定速度才能使级数收敛。

不难总结级数收敛的判断方法:
\begin{itemize}
    \item 从定义出发,即考察$\underset{n\rightarrow \infty}{\lim}S_n$是否存在;
    \item 在已知某些级数的敛散性的基础上,将目标级数拆分,运用性质判断;
    \item 定理\ref{th_12_1_1}可用于判断级数是否发散。
\end{itemize}

%============================================================
\subsection{三个基础的级数的敛散性}

{\bf 等比级数$\sum_{n=1}^{\infty}{aq^n},a\ne 0$}

当$q=1$时,显然级数发散。
当$q=-1$时,虽然级数有界,但是在0和$a$之间震荡,级数依然发散。
当$\left| q \right|\ne 1$时,数列和有公式$S_n=a\frac{1-q^n}{1-q}$,当$\left| q \right|>1$时级数发散,当$\left| q \right|<1$时级数收敛。
\[
\sum_{n=1}^{\infty}{aq^n}=\begin{cases}
	\frac{a}{1-q} & \left| q \right|<1\\
	NA & \left| q \right|\geqslant 1\\
\end{cases}
\]

{\bf 调和级数$\sum_{n=1}^{\infty}{\frac{1}{n}}$}

使用零点麦克劳林展开,有:
\begin{align*}
&\because e^x=1+x+\frac{x^2}{2!}+...>1+x \\
&\therefore x>\ln \left( 1+x \right) \\
&\therefore \frac{1}{n}>\ln \left( 1+\frac{1}{n} \right) \\
&\therefore S_n>\sum_{i=1}^n{\ln \left( 1+\frac{1}{i} \right)}
\end{align*}
通过对数运算,将加法变成乘法:
\begin{align*}
&\because \sum_{i=1}^n{\ln \left( 1+\frac{1}{i} \right)}=\ln \left( 2\cdot \frac{3}{2}\cdot \frac{4}{3}\cdot ...\cdot \frac{n+1}{n} \right) =\ln \left( n+1 \right) \\
&\therefore \underset{n\rightarrow \infty}{\lim}S_n>\underset{n\rightarrow \infty}{\lim}\ln \left( n+1 \right) =+\infty
\end{align*}
可见,调和级数发散。

{\bf 无穷级数$\frac{1}{1\cdot 2}+\frac{1}{2\cdot 3}+\cdots +\frac{1}{n\left( n+1 \right)}+\cdots $}
\begin{align*}
&\because S_n=\sum_{i=1}^n{\frac{1}{i\left( i+1 \right)}}=\sum_{i=1}^n{\left( \frac{1}{i}-\frac{1}{i+1} \right)}=1-\frac{1}{n+1}=\frac{n}{n+1} \\
&\therefore \underset{n\rightarrow \infty}{\lim}S_n=1
\end{align*}
级数收敛于1。

~

综合来讲:
\begin{itemize}
    \item 等比级数$\sum_{n=1}^{\infty}{aq^n},a\ne 0$的敛散性看$\left| q \right|$;
    \item 调和级数$\sum_{n=1}^{\infty}{\frac{1}{n}}$发散;
    \item 无穷级数$\sum_{n=1}^{\infty}{\frac{1}{n\cdot \left( n+1 \right)}}$收敛于1。
\end{itemize}




