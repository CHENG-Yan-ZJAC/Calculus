\section{多元函数的极值}

之前在一元函数中我们用导数讨论了极值问题,这里我们扩展到二元函数,用梯度和黑塞矩阵讨论二元函数的极值问题。

本节要点:
\begin{itemize}
    \item 掌握二元函数的极值的概念;
    \item 深刻理解二元函数极值的判断方法;
    \item 了解条件极值和拉格朗日乘数法。
\end{itemize}

%============================================================
\subsection{极值的概念}

\begin{definition}[极值]
若函数$f\left( \boldsymbol{p} \right) $在点$\boldsymbol{p}_0$的某去心邻域$N\left( \boldsymbol{\hat{p}}_0,\delta \right) $内任意一点$\boldsymbol{p}$有
\[
f\left( \boldsymbol{p} \right) >f\left( \boldsymbol{p}_0 \right) \quad \text{或} \quad f\left( \boldsymbol{p} \right) <f\left( \boldsymbol{p}_0 \right)
\]
则称$f\left( \boldsymbol{p}_0 \right) $为{\bf 函数$f\left( \boldsymbol{p} \right) $的极小值}(或{\bf 极大值}),点$\boldsymbol{p}_0$为{\bf 函数的极小值点}(或{\bf 极大值点})。
我们统称极小值和极大值为{\bf 极值},极小值点和极大值点为{\bf 极值点}。
\end{definition}

在讨论函数$f\left( \boldsymbol{p} \right) $的极值问题时,我们将$f\left( \boldsymbol{p} \right) $称为{\bf 目标函数},这类问题称为无条件极值问题。
如果加了一些限定条件如$g\left( \boldsymbol{p} \right) =0$(也可以是不等式),则称为{\bf 条件极值}。

%============================================================
\subsection{极值的判断}

类似于一元函数,二元函数的极值判断也有充分条件和必要条件。

\begin{theorem}[极值的充分条件]
若函数$f\left( \boldsymbol{p} \right) $在$\boldsymbol{p}_0$处有偏导,在$\boldsymbol{p}_0$取得极值,则:
\[
f_x\left( \boldsymbol{p}_0 \right) =f_y\left( \boldsymbol{p}_0 \right) =0
\]
用矢量写为:
\[
\nabla f\left( \boldsymbol{p}_0 \right) =\mathbf{0}
\]
符合上式的$\boldsymbol{p}_0$称为{\bf 函数的驻点}。
\end{theorem}

\begin{theorem}[极值的必要条件]
若函数$f\left( \boldsymbol{p} \right) $在$\boldsymbol{p}_0$的某邻域内有二阶连续偏导,且$f_x\left( \boldsymbol{p}_0 \right) =f_y\left( \boldsymbol{p}_0 \right) =0$,记函数$f\left( \boldsymbol{p} \right) $在$\boldsymbol{p}_0$的黑塞矩阵
\[
H_f\left( \boldsymbol{p}_0 \right) =\left. \left[ \begin{matrix}
	f_{xx}&		f_{xy}\\
	f_{yx}&		f_{yy}\\
\end{matrix} \right] \right|_{\boldsymbol{p}_0}\triangleq \left[ \begin{matrix}
	A&		B\\
	B&		C\\
\end{matrix} \right]
\]
\begin{itemize}
    \item 当$H_f\left( \boldsymbol{p}_0 \right) $为正定矩阵,即$AC>B^2,A>0$,则$f\left( \boldsymbol{p}_0 \right) $为极小值;
    \item 当$H_f\left( \boldsymbol{p}_0 \right) $为负定矩阵,即$AC>B^2,A<0$,则$f\left( \boldsymbol{p}_0 \right) $为极大值;
    \item 当$H_f\left( \boldsymbol{p}_0 \right) $为不定矩阵,即$AC<B^2$,则$f\left( \boldsymbol{p}_0 \right) $不是极值。
\end{itemize}
\end{theorem}

在二元函数中,我们用梯度判断驻点,用黑塞矩阵判断极值点。
如果要判断全部极值点,还需要判断边界点和不可导点。
和一元函数对比学习!

%============================================================
\subsection{条件极值和拉格朗日乘数法}

所谓条件极值,就是$\boldsymbol{p}$中的标量元素间有相互制约关系,这点在一元函数中是不存在的。
如计算球面上点$\boldsymbol{p}\in S$和球面外的点$\boldsymbol{q}$最近的点。
这里,$S$就是约束条件,$f\left( \boldsymbol{p} \right) =\left\| \boldsymbol{p}-\boldsymbol{q} \right\| $就是目标函数。
我们要求解的就是$f\left( \boldsymbol{p} \right) =\left\| \boldsymbol{p}-\boldsymbol{q} \right\| $的最小值。

通常带约束条件的极值判断较为复杂,拉格朗日乘数法是一个通常的解决方法,具体参见“教材\cite{book1}”。
总体思路是将约束条件带入目标函数,构造拉格朗日函数$F$,用$\nabla F=\mathbf{0}$求出驻点,最后结合边界、不可导点和问题本身的物理意义判断极值。




