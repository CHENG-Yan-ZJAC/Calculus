\section{全微分}

本节讨论全微分。
全微分是多元函数微分学的核心概念,是后续概念和定理的基础,同时在工程上也是一个用于估算微增量的非常有用的概念。

本节要点:
\begin{itemize}
    \item 掌握全微分定义;
    \item 深入理解全微分定理;
    \item 掌握全微分的应用。
\end{itemize}

%============================================================
\subsection{全微分的概念}

\begin{definition}[全微分]
如果区域$D$内有函数$z=f\left( \boldsymbol{p} \right) $,若在$\boldsymbol{p}_0\in D$处全增量满足:
\[
\Delta z=A\left( \boldsymbol{p}_0 \right) \Delta x+B\left( \boldsymbol{p}_0 \right) \Delta y+o\left( \left\| \Delta \boldsymbol{p} \right\| \right)
\]
其中:
\begin{itemize}
    \item $A\left( \boldsymbol{p}_0 \right) ,B\left( \boldsymbol{p}_0 \right) $:只与$\boldsymbol{p}_0$有关、与$\Delta \boldsymbol{p}$无关的常数;
    \item $\left\| \Delta \boldsymbol{p} \right\| $:自变量全增量$\Delta \boldsymbol{p}$的长度;
    \item $o\left( \left\| \Delta \boldsymbol{p} \right\| \right) $:当$\left\| \Delta \boldsymbol{p} \right\| \rightarrow 0$时,关于$\left\| \Delta \boldsymbol{p} \right\| $的高阶无穷小量,
\end{itemize}
则称{\bf 函数$z=f\left( \boldsymbol{p} \right) $在$\boldsymbol{p}_0$处可微},其中,$A\left( \boldsymbol{p}_0 \right) \Delta x+B\left( \boldsymbol{p}_0 \right) \Delta y$称为{\bf 全微分}(也称$\Delta z$的{\bf 线性主部}),记作$dz$,即:
\[
dz:=A\left( \boldsymbol{p}_0 \right) \Delta x+B\left( \boldsymbol{p}_0 \right) \Delta y
\]
如果函数在区域$D$内每点处都可微,则称{\bf 函数$z=f\left( \boldsymbol{p} \right) $在区域$D$内可微},由于$\Delta x=dx,\Delta y=dy$,上式也可写成:
\[
dz=A\left( \boldsymbol{p} \right) dx+B\left( \boldsymbol{p} \right) dy
\]
\end{definition}

%============================================================
\subsection{全微分的代数意义}

全微分表达的意思是,如果全增量$\Delta z$可以分成:
\begin{itemize}
    \item $A\Delta x+B\Delta y$:关于自变量偏增量的一个线性组合;
    \item $o\left( \left\| \Delta \boldsymbol{p} \right\| \right) $:可以忽略的无穷小量,
\end{itemize}
则认为$A\Delta x+B\Delta y$可以代替全增量$\Delta z$,即:
\[
\Delta z=A\Delta x+B\Delta y
\]

\begin{tcolorbox}
“线性化、以直代曲”是微分的目的,无论是多元函数全微分还是一元函数的微分。
\end{tcolorbox}

%============================================================
\subsection{全微分定理}

\begin{theorem}[全微分定理]
函数$z=f\left( \boldsymbol{p} \right) $可微$\Leftrightarrow $函数$z=f\left( \boldsymbol{p} \right) $连续且两个偏导数存在,且$A\left( \boldsymbol{p} \right) =\frac{\partial z}{\partial x},B\left( \boldsymbol{p} \right) =\frac{\partial z}{\partial y}$,即:
\[
dz=\frac{\partial z}{\partial x}dx+\frac{\partial z}{\partial y}dy
\]
\end{theorem}

\begin{proof}
该定理的关键点在于计算偏导$\frac{\partial z}{\partial x}$。
从可微和偏导各自的定义出发:
\begin{align*}
&\Delta z=A\left( \boldsymbol{p}_0 \right) \Delta x+B\left( \boldsymbol{p}_0 \right) \Delta y+o\left( \left\| \Delta \boldsymbol{p} \right\| \right) \\
&\left. \frac{\partial z}{\partial x} \right|_{\boldsymbol{p}_0}=\underset{\Delta x\rightarrow 0}{\lim}\frac{\Delta _xz}{\Delta x}=\underset{\Delta x\rightarrow 0}{\lim}\frac{f\left( \boldsymbol{p}_0+\Delta _x\boldsymbol{p} \right) -f\left( \boldsymbol{p}_0 \right)}{\Delta x}
\end{align*}
计算$\frac{\partial z}{\partial x}$如下:
\begin{align*}
&\left. \frac{\partial z}{\partial x} \right|_{\boldsymbol{p}_0}=\underset{\Delta x\rightarrow 0}{\lim}\frac{\Delta _xz}{\Delta x}=\underset{\Delta x\rightarrow 0}{\lim}\frac{f\left( \boldsymbol{p}_0+\Delta _x\boldsymbol{p} \right) -f\left( \boldsymbol{p}_0 \right)}{\Delta x} \\
&=\underset{\Delta x\rightarrow 0}{\lim}\frac{f\left( \boldsymbol{p}_0+\Delta _x\boldsymbol{p} \right) -f\left( \boldsymbol{p}_0 \right) +f\left( \boldsymbol{p}_0+\Delta \boldsymbol{p} \right) -f\left( \boldsymbol{p}_0+\Delta \boldsymbol{p} \right)}{\Delta x} \\
&=\underset{\Delta x\rightarrow 0}{\lim}\frac{\Delta z+f\left( \boldsymbol{p}_0+\Delta _x\boldsymbol{p} \right) -f\left( \boldsymbol{p}_0+\Delta \boldsymbol{p} \right)}{\Delta x} \\
&=\underset{\Delta x\rightarrow 0}{\lim}\frac{A\left( \boldsymbol{p}_0 \right) \Delta x+B\left( \boldsymbol{p}_0 \right) \Delta y+o\left( \left\| \Delta \boldsymbol{p} \right\| \right) +f\left( \boldsymbol{p}_0+\Delta _x\boldsymbol{p} \right) -f\left( \boldsymbol{p}_0+\Delta \boldsymbol{p} \right)}{\Delta x} \\
&=A\left( \boldsymbol{p}_0 \right) +\underset{\Delta x\rightarrow 0}{\lim}\frac{B\left( \boldsymbol{p}_0 \right) \Delta y-\left[ f\left( \boldsymbol{p}_0+\Delta \boldsymbol{p} \right) -f\left( \boldsymbol{p}_0+\Delta _x\boldsymbol{p} \right) \right] +o\left( \left\| \Delta \boldsymbol{p} \right\| \right)}{\Delta x}
\end{align*}
其中:
\begin{itemize}
    \item $B\left( \boldsymbol{p}_0 \right) \Delta y$:自变量的偏增量$\Delta y$引起的$z$从$\boldsymbol{p}_0$开始的线性部分增量;
    \item $f\left( \boldsymbol{p}_0+\Delta \boldsymbol{p} \right) -f\left( \boldsymbol{p}_0+\Delta _x\boldsymbol{p} \right) $:自变量的偏增量$\Delta y$引起的$z$从$\boldsymbol{p}_0+\Delta _x\boldsymbol{p}$开始的线性部分增量;
    \item $o\left( \left\| \Delta \boldsymbol{p} \right\| \right) $:当$\Delta x\rightarrow 0$时可以忽略的无穷小量。
\end{itemize}
当我们考察$\frac{\partial z}{\partial x}$时,有$\Delta y=0$,所以上述部分中前两部分的值为0,于是上式变为:
\[
\left. \frac{\partial z}{\partial x} \right|_{\boldsymbol{p}_0}=A\left( \boldsymbol{p}_0 \right) +\underset{\Delta x\rightarrow 0}{\lim}\frac{o\left( \left\| \Delta \boldsymbol{p} \right\| \right)}{\Delta x}=A\left( \boldsymbol{p}_0 \right)
\]
\end{proof}

一般函数偏导会存在,但函数本身不一定连续。
连续的前提条件是极限存在。
所以讨论多元函数在某点是否可微,首先得看该函数在该点的是否有极限,再看极限值是否为该函数值,即证明连续。
\begin{align*}
&\text{多元函数:可微} \Leftrightarrow \text{偏导} + \text{连续} \\
&\text{一元函数:可微} \Leftrightarrow \text{偏导} \Rightarrow \text{连续}
\end{align*}

%============================================================
\subsection{函数可微性的讨论}

函数在某点的可微性可从两个思路出发:

一、从微分的定义入手:
\begin{enumerate}
    \item 首先证明函数在该点的偏导数存在;
    \item 再证明余项$o\left( \left\| \Delta \boldsymbol{p} \right\| \right) $为$\left\| \Delta \boldsymbol{p} \right\| $的高阶无穷小量。
\end{enumerate}

二、从全微分定理入手:
\begin{enumerate}
    \item 同上,首先证明函数在该点的偏导数存在;
    \item 再证明函数在该点连续,即证明等式$\underset{\boldsymbol{p}\rightarrow \boldsymbol{p}_0}{\lim}f\left( \boldsymbol{p} \right) =f\left( \boldsymbol{p}_0 \right) $。
\end{enumerate}

~

\begin{example}
设函数$f\left( \boldsymbol{p} \right) =\left| x-y \right|g\left( \boldsymbol{p} \right) $,其中$g\left( \boldsymbol{p} \right) $在原点的某邻域内连续,问为$g\left( \mathbf{0} \right) $何值时$f\left( \boldsymbol{p} \right) $在原点处可微。
\end{example}

解一,从全微分定理入手。

首先考察偏导$f_x\left( \mathbf{0} \right) $对$g\left( \mathbf{0} \right) $的要求。
根据定义:
\[
\left. \frac{\partial f}{\partial x} \right|_{\mathbf{0}}=\underset{x\rightarrow 0}{\lim}\frac{\left| x \right|g\left( x,0 \right) -0\cdot g\left( x,0 \right)}{x-0}=\underset{x\rightarrow 0}{\lim}\frac{\left| x \right|g\left( x,0 \right)}{x}
\]
要使偏导存在,必须该极限存在,即左右极限存在且相等:
\[
\underset{x\rightarrow 0^+}{\lim}\frac{xg\left( x,0 \right)}{x}=\underset{x\rightarrow 0^-}{\lim}\frac{-xg\left( x,0 \right)}{x}
\]
由于$g\left( \boldsymbol{p} \right) $在原点的某邻域内连续,得到$g\left( \mathbf{0} \right) =0$,且有$\left. \frac{\partial f}{\partial x} \right|_{\mathbf{0}}=0$。对$f_y\left( \mathbf{0} \right) $分析,可得到同样结论。
考虑第二个条件——连续,根据定义,需要下式成立:
\[
\underset{x,y\rightarrow 0}{\lim}\left| x-y \right|g\left( x,y \right) =0
\]
由于刚才分析要求$g\left( \mathbf{0} \right) =0$,所以上式必然成立。
综合得到,当$g\left( \mathbf{0} \right) =0$时, 在原点处可微。

解二,从微分的定义出发证明无穷小量,即证明:
\begin{align*}
&\underset{\left\| \Delta \boldsymbol{p} \right\| \rightarrow 0}{\lim}\frac{o\left( \left\| \Delta \boldsymbol{p} \right\| \right)}{\left\| \Delta \boldsymbol{p} \right\|} \\
&=\underset{\left\| \Delta \boldsymbol{p} \right\| \rightarrow 0}{\lim}\frac{\left| \Delta x-\Delta y \right|g\left( \Delta x,\Delta y \right) -0\cdot g\left( 0,0 \right) -\frac{\partial f}{\partial x}\Delta x-\frac{\partial f}{\partial y}\Delta y}{\left\| \Delta \boldsymbol{p} \right\|}
\\
&=\underset{\Delta x,\Delta y\rightarrow 0}{\lim}\frac{\left| \Delta x-\Delta y \right|g\left( \Delta x,\Delta y \right)}{\sqrt{\Delta x^2+\Delta y^2}} \\
&\leqslant \underset{\Delta x,\Delta y\rightarrow 0}{\lim}\frac{\left| \Delta x \right|+\left| \Delta y \right|}{\sqrt{\Delta x^2+\Delta y^2}}g\left( \Delta x,\Delta y \right) \\
&\leqslant \sqrt{2}\cdot \underset{\Delta x,\Delta y\rightarrow 0}{\lim}g\left( \Delta x,\Delta y \right) \\
&=\sqrt{2}\cdot g\left( 0,0 \right) =0
\end{align*}
其中:
\begin{align*}
&\because \frac{\left| \Delta x \right|^2+\left| \Delta y \right|^2+2\left| \Delta x \right|\left| \Delta y \right|}{\left| \Delta x \right|^2+\left| \Delta y \right|^2}\leqslant \frac{\left| \Delta x \right|^2+\left| \Delta y \right|^2+\left| \Delta x \right|^2+\left| \Delta y \right|^2}{\left| \Delta x \right|^2+\left| \Delta y \right|^2}=2 \\
&\therefore \frac{\left| \Delta x \right|+\left| \Delta y \right|}{\sqrt{\left| \Delta x \right|^2+\left| \Delta y \right|^2}}\leqslant \sqrt{2}
\end{align*}

%============================================================
\subsection{全微分的应用——估值及误差}

同一元函数一样,用全微分我们可以近似计算一些工程值,不仅如此,我们还可以知道什么时候能做近似计算,以及误差是多少。
在做估算时,我们用$dz$代替$\Delta z$。
所以估算的关键是:
\begin{itemize}
    \item 构造合适的$f\left( \boldsymbol{p} \right) $,使得偏微分易于计算。
    \item 选定合适的$f\left( \boldsymbol{p}_0 \right) $,使得$\boldsymbol{p}_0$易于计算且对应的$\Delta x,\Delta y$非常小。
\end{itemize}

对于误差分析,我们定义如下概念:
\begin{itemize}
    \item {\bf 真值}:$f\left( \boldsymbol{p}_0+\Delta \boldsymbol{p} \right) $
    \item {\bf 绝对误差}:$\delta _z=\left| f_x\left( \boldsymbol{p}_0 \right) \right|\cdot dx+\left| f_y\left( \boldsymbol{p}_0 \right) \right|\cdot dy$
    \item {\bf 相对误差}:$\frac{\delta _z}{\left| z \right|}=\frac{\left| f_x\left( \boldsymbol{p}_0 \right) \right|\cdot dx+\left| f_y\left( \boldsymbol{p}_0 \right) \right|\cdot dy}{\left| z \right|}$
\end{itemize}

~

\begin{example}
求$1.04^{2.02}$近似值。
\end{example}

解:

首先构造一个合适的函数$z=f\left( \boldsymbol{p} \right) =x^y$,再选定起点$\boldsymbol{p}_0=\left( 1\,\,2 \right) ^T$,得:
\begin{align*}
1.04^{2.02}&\approx f\left( \boldsymbol{p}_0 \right) +dz \\
&=f\left( \boldsymbol{p}_0 \right) +f_x\left( \boldsymbol{p}_0 \right) \cdot dx+f_y\left( \boldsymbol{p}_0 \right) \cdot dy \\
&=1^2+\left. yx^{y-1} \right|_{\boldsymbol{p}_0}\cdot 0.04+\left. x^y\ln x \right|_{\boldsymbol{p}_0}\cdot 0.02 \\
&=1+0.08=1.08
\end{align*}
实际值是1.0824...,还是非常准的。


~

\begin{example}
如果我们通过电压电流测量计算电阻,已知电压表相对误差1\%,电流表相对误差0.5\%,如果测得电阻两端电压和电流分别为32V和8A,分析电阻的误差。
\end{example}

解:

电阻的绝对误差和相对误差:
\begin{align*}
&\delta _R=\left| f_V\left( V,I \right) \right|\cdot dV+\left| f_I\left( V,I \right) \right|\cdot dI=\left| \frac{1}{V} \right|dV+\left| -\frac{V}{I^2} \right|dI \\
&\frac{\delta _R}{\left| R \right|}=\frac{\delta _R}{\left| \frac{V}{I} \right|}
\end{align*}
电压电流的相对误差已知为1\%和0.5\%,可以计算它们的绝对误差:
\begin{align*}
&dV=32\cdot 1\% \\
&dI=8\cdot 0.5\%
\end{align*}
于是:
\begin{align*}
&\delta _R=\left| \frac{1}{V} \right|dV+\left| -\frac{V}{I^2} \right|dI=\frac{1}{8}\cdot \left( 32\cdot 1\% \right) +\frac{32}{8\cdot 8}\cdot \left( 8\cdot 0.5\% \right) =0.06\Omega \\
&\frac{\delta _R}{\left| R \right|}=\frac{\delta _R}{\left| \frac{V}{I} \right|}=\frac{0.06}{\left| \frac{32}{8} \right|}=15\%
\end{align*}




