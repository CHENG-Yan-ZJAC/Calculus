\section{二元函数的泰勒展开}

之前在一元函数中,我们讨论了泰勒展开,这里我们进一步讨论二元函数的泰勒展开。

本节要点:
\begin{itemize}
    \item 掌握二元函数的泰勒展开;
    \item 了解黑塞矩阵。
\end{itemize}

%============================================================
\subsection{二元函数的泰勒展开}

\begin{theorem}[泰勒定理(拉格朗日型)]
若函数$f\left( x,y \right) $在$\left( x_0,y_0 \right) $的某邻域$N$内具有n+1阶连续偏导数,$\left( x_0+h,y_0+k \right) \in N$,则有:
\begin{align*}
f\left( x_0+h,y_0+k \right) =&f\left( x_0,y_0 \right) + \\
&\left( h\frac{\partial}{\partial x}+k\frac{\partial}{\partial y} \right) f\left( x_0,y_0 \right) + \\
&\frac{1}{2!}\left( h\frac{\partial}{\partial x}+k\frac{\partial}{\partial y} \right) ^2f\left( x_0,y_0 \right) + \\
&\cdots \\
&\frac{1}{n!}\left( h\frac{\partial}{\partial x}+k\frac{\partial}{\partial y} \right) ^nf\left( x_0,y_0 \right) + \\
&R_n
\end{align*}
其中:
\begin{itemize}
    \item $R_n=\frac{1}{\left( n+1 \right) !}\left( h\frac{\partial}{\partial x}+k\frac{\partial}{\partial y} \right) ^{n+1}f\left( x_0+\theta h,y_0+\theta k \right) ,\theta \in \left( 0,1 \right) $:{\bf 拉格朗日余项};
\end{itemize}
该展开称为{\bf $f\left( x,y \right) $在$\left( x_0,y_0 \right) $的$n$阶具有拉格朗日余项的泰勒公式}。
上式中的记号:
\begin{align*}
&\left( h\frac{\partial}{\partial x}+k\frac{\partial}{\partial y} \right) f\left( x_0,y_0 \right) =hf_x\left( x_0,y_0 \right) +kf_y\left( x_0,y_0 \right) \\
&\left( h\frac{\partial}{\partial x}+k\frac{\partial}{\partial y} \right) ^2f\left( x_0,y_0 \right) =h^2f_{xx}\left( x_0,y_0 \right) +2hkf_{xy}\left( x_0,y_0 \right) +k^2f_{yy}\left( x_0,y_0 \right) \\
&\left( h\frac{\partial}{\partial x}+k\frac{\partial}{\partial y} \right) ^nf\left( x_0,y_0 \right) =\sum_{i=0}^n{\left[ C_{n}^{i}h^ik^{n-i}\frac{\partial ^nf\left( x_0,y_0 \right)}{\partial x^i\partial y^{n-i}} \right]}
\end{align*}
\end{theorem}

证明略,大致是一元泰勒展开+复合函数的链式求导,详见“教材\cite{book1}”。

\begin{theorem}[泰勒定理(佩亚诺型)]
若函数$f\left( x,y \right) $在$\left( x_0,y_0 \right) $的某邻域$N$内具有n阶连续偏导数,$\left( x_0+h,y_0+k \right) \in N$,则有:
\begin{align*}
f\left( x_0+h,y_0+k \right) =&f\left( x_0,y_0 \right) + \\
&\left( h\frac{\partial}{\partial x}+k\frac{\partial}{\partial y} \right) f\left( x_0,y_0 \right) + \\
&\frac{1}{2!}\left( h\frac{\partial}{\partial x}+k\frac{\partial}{\partial y} \right) ^2f\left( x_0,y_0 \right) + \\
&\cdots \\
&\frac{1}{n!}\left( h\frac{\partial}{\partial x}+k\frac{\partial}{\partial y} \right) ^nf\left( x_0,y_0 \right) + \\
&o\left( \rho ^n \right)
\end{align*}
其中:
\begin{itemize}
    \item $\rho =\sqrt{h^2+k^2}$:{\bf 佩亚诺余项};
\end{itemize}
该展开称为{\bf $f\left( x,y \right) $在$\left( x_0,y_0 \right) $的$n$阶具有佩亚诺余项的泰勒公式}。
\end{theorem}

%============================================================
\subsection{黑塞Hessian矩阵}

当我们考虑二阶拉格朗日余项的泰勒展开:
\begin{align*}
f\left( x_0+h,y_0+k \right) =&f\left( x_0,y_0 \right) + \\
&\left( h\frac{\partial}{\partial x}+k\frac{\partial}{\partial y} \right) f\left( x_0,y_0 \right) + \\
&\frac{1}{2!}\left( h\frac{\partial}{\partial x}+k\frac{\partial}{\partial y} \right) ^2f\left( x_0,y_0 \right) + \\
&R_3
\end{align*}
其中,第二项写成向量形式:
\[
\left( h\frac{\partial}{\partial x}+k\frac{\partial}{\partial y} \right) f\left( x_0,y_0 \right) =\left( \begin{array}{c}
	h\\
	k\\
\end{array} \right) ^T\left( \begin{array}{c}
	\left. \frac{\partial f}{\partial x} \right|_{\boldsymbol{p}_0}\\
	\left. \frac{\partial f}{\partial y} \right|_{\boldsymbol{p}_0}\\
\end{array} \right) =\left( \begin{array}{c}
	h\\
	k\\
\end{array} \right) ^T\nabla f_{\boldsymbol{p}_0}
\]
第三项写成向量形式:
\begin{align*}
\left( h\frac{\partial}{\partial x}+k\frac{\partial}{\partial y} \right) ^2f\left( x_0,y_0 \right) &=h^2f_{xx}\left( \boldsymbol{p}_0 \right) +2hkf_{xy}\left( \boldsymbol{p}_0 \right) +k^2f_{yy}\left( \boldsymbol{p}_0 \right) \\
&=\left( \begin{array}{c}
	h\\
	k\\
\end{array} \right) ^T\left[ \begin{matrix}
	\left. f_{xx} \right|_{\boldsymbol{p}_0}&		\left. f_{xy} \right|_{\boldsymbol{p}_0}\\
	\left. f_{yx} \right|_{\boldsymbol{p}_0}&		\left. f_{yy} \right|_{\boldsymbol{p}_0}\\
\end{matrix} \right] \left( \begin{array}{c}
	h\\
	k\\
\end{array} \right)
\end{align*}
我们将矩阵
\[
H_f=\left[ \begin{matrix}
	f_{xx}&		f_{xy}\\
	f_{yx}&		f_{yy}\\
\end{matrix} \right]
\]
称为{\bf 函数$f\left( x,y \right) $的黑塞(Hessian)矩阵},记作$H_f$,特别地,当$f\left( x,y \right) $具有二阶连续偏导时,有$f_{xy}=f_{yx}$,于是Hessian矩阵为对称矩阵,即$H_f={H_f}^T$。

于是,二阶拉格朗日余项的泰勒展开用矩阵形式可以写成矩阵形式:
\begin{align*}
f\left( \boldsymbol{p} \right) =&f\left( \boldsymbol{p}_0 \right) + \\
&\left( \boldsymbol{p}-\boldsymbol{p}_0 \right) ^T\nabla f_{\boldsymbol{p}_0}+ \\
&\frac{1}{2!}\left( \boldsymbol{p}-\boldsymbol{p}_0 \right) ^TH_f\left( \boldsymbol{p}_0 \right) \left( \boldsymbol{p}-\boldsymbol{p}_0 \right) + \\
&R_3
\end{align*}

一元函数是标量,所以其一阶导数和二阶导数都是标量。
二元函数虽然是标量,但一阶导数(也即梯度)就是一个矢量,二阶导数就是一个$2\times 2$矩阵,这就是黑塞矩阵。
所以黑塞矩阵相当于一元函数中的二阶导数,可用于判断函数的极值。




