\section{应用}

本节讨论多元函数微分学的两个实例。

%============================================================
\subsection{电场和电势}

设空间原点处有一电荷$q$,讨论电场和电势的关系。
三维空间某点的电势和电场表达式:
\begin{align*}
&U\left( \boldsymbol{p} \right) =\frac{q}{4\pi \varepsilon \left\| \boldsymbol{p} \right\|} \\
&\boldsymbol{E}\left( \boldsymbol{p} \right) =\frac{q}{4\pi \varepsilon \left\| \boldsymbol{p} \right\| ^2}\cdot \frac{\boldsymbol{p}}{\left\| \boldsymbol{p} \right\|}
\end{align*}
其中,$\boldsymbol{p}=\left( x\,\,y\,\,z \right) ^T$三维空间某点的坐标。

对电势求梯度,得:
\begin{align*}
&\begin{aligned}
	\because \nabla U\left( \boldsymbol{p} \right) &=\left( \begin{array}{c}
	\frac{\partial}{\partial x}\\
	\frac{\partial}{\partial y}\\
	\frac{\partial}{\partial z}\\
\end{array} \right) \frac{q}{4\pi \varepsilon \left\| \boldsymbol{p} \right\|}=\frac{q}{4\pi \varepsilon}\left( \begin{array}{c}
	\frac{\partial}{\partial x}\frac{1}{\sqrt{x^2+y^2+z^2}}\\
	\frac{\partial}{\partial y}\frac{1}{\sqrt{x^2+y^2+z^2}}\\
	\frac{\partial}{\partial z}\frac{1}{\sqrt{x^2+y^2+z^2}}\\
\end{array} \right)\\
	&=-\frac{q}{4\pi \varepsilon \left\| \boldsymbol{p} \right\| ^3}\left( \begin{array}{c}
	x\\
	y\\
	z\\
\end{array} \right) =-\frac{q}{4\pi \varepsilon \left\| \boldsymbol{p} \right\| ^3}\cdot \boldsymbol{p}\\
\end{aligned} \\
&\therefore \nabla U\left( \boldsymbol{p} \right) =-\boldsymbol{E}\left( \boldsymbol{p} \right)
\end{align*}
电荷在空间任意一点的电场为该点电势的负梯度。
即电场表示电势的变化率,且是反向的变化率。

可以想象,电势描述的是静电荷产生的静电势能,是一个数量场,就像一座山产生了重力势能。
而其变化率,或理解为这个势的“坡”,是有方向有大小的,即是一个矢量场,该矢量场就是电场。
梯度的方向是电势升高的方向为正,而电场方向是正电荷发出的方向,所以两者差了一个负号。

%============================================================
\subsection{一维波动方程}

物理上的波,指的是空间上一点的振动引起相隔的点的振动,而且这种振动以自有的速度向其他方向传播的现象。
一维空间上,假设原点有振动$y=f\left( t \right) $,该振动以速度$v$向{\it x}轴正向转播,则在$t_0$时刻,$x_0$处的质点有:
\[
y=f\left( t_0-\frac{x_0}{v} \right)
\]
可得{\it x}轴上任意一点在任意时刻的振动满足:
\[
y\left( x,t \right) =f\left( t-\frac{x}{v} \right)
\]
该振动就是{\bf 波},并会以速度$v$沿{\it x}轴方向传播。
符合该特征的函数称为{\bf 波函数}。

对波函数求偏导:
\begin{align*}
&\begin{cases}
	\frac{\partial y}{\partial x}=\frac{df}{du}\cdot \frac{\partial}{\partial x}\left( t-\frac{x}{v} \right) =-\frac{1}{v}\frac{df}{du}\\
	\frac{\partial ^2y}{\partial x^2}=-\frac{1}{v}\cdot \frac{\partial}{\partial x}\left( \frac{df}{du} \right) =-\frac{1}{v}\cdot \frac{d^2f}{du^2}\cdot \frac{\partial}{\partial x}\left( t-\frac{x}{v} \right) =\frac{1}{v^2}\frac{d^2f}{du^2}\\
\end{cases} \\
&\begin{cases}
	\frac{\partial y}{\partial t}=\frac{df}{du}\cdot \frac{\partial}{\partial t}\left( t-\frac{x}{v} \right) =\frac{df}{du}\\
	\frac{\partial ^2y}{\partial t^2}=\frac{\partial}{\partial t}\left( \frac{df}{du} \right) =\frac{d^2f}{du^2}\cdot \frac{\partial}{\partial t}\left( t-\frac{x}{v} \right) =\frac{d^2f}{du^2}\\
\end{cases}
\end{align*}
可见波函数满足如下偏微分方程:
\[
\frac{\partial ^2y}{\partial t^2}=v^2\cdot \frac{\partial ^2y}{\partial x^2}
\]
该偏微分方程称为{\bf 波动方程},同时,我们称该方程的解为{\bf 波函数}。




