\section{本章小结}

至此我们讨论完了二元函数微分学。
二元函数微分学主要讨论曲面的光滑性,我们在上一章给出了光滑曲面的第一个要求——不断,本章讨论了第二个要求——不折。
我们定义了偏导数、方向导数,并给出了二元函数微分学的一个核心概念——全微分。
如果曲面在一点满足全微分要求,几何上就代表在该点有切平面。
最后我们讨论了梯度,我们将一个曲面的变化率按照自变量坐标系分解得到一个矢量形式的变化率。
我们还举例了微分的实际用处——估值。

学习本章特别要和一元函数微分学中的定义对比。
同样是“以直代曲”,一元函数微分学中较为简单,而由于二元函数中$z$的变化量不光取决于$x,y$各自变化,还有它们联合产生的效果。

下面的积分章节,我们要讨论光滑在“累积”方面有哪些特征,可以帮助我们解决什么问题。




