\section{隐函数的求导}

对于隐函数,本节给出隐函数的求偏导的方法,该方法去除了链式求导步骤,大大简化计算过程。
隐函数求导法在曲线积分和曲面积分中有重要使用,需要熟练掌握。

本节要点:
\begin{itemize}
    \item 掌握一元隐函数和二元隐函数的求导法则。
\end{itemize}

%============================================================
\subsection{一元隐函数的求导}

\begin{theorem}[一元隐函数求导定理]
将方程$F\left( x,y \right) =0$的左边看成二元函数$z=F\left( \boldsymbol{p} \right) $,若满足:
\begin{itemize}
    \item 在点$\boldsymbol{p}_0$的某一邻域内有连续的偏导数,且$\left. \frac{\partial z}{\partial y} \right|_{\boldsymbol{p}_0}\ne 0$,
    \item 点$\boldsymbol{p}_0$满足方程$F\left( \boldsymbol{p}_0 \right) =0$,
\end{itemize}
则点$\boldsymbol{p}_0$的某一邻域内,方程$F\left( x,y \right) =0$将唯一确定单值连续函数$y=y\left( x \right) $,使得$F\left( x,y\left( x \right) \right) =0$,且$y_0=y\left( x_0 \right) $,并有:
\[
\frac{dy}{dx}=-\frac{\frac{\partial z}{\partial x}}{\frac{\partial z}{\partial y}}=-\frac{F_x}{F_y}
\]
\end{theorem}

%============================================================
\subsection{二元隐函数的求导}

\begin{theorem}[二元隐函数求导定理]
将方程$F\left( x,y,z \right) =0$的左边看成三元函数$u=F\left( \boldsymbol{p} \right) ,\boldsymbol{p}=\left( x\,\,y\,\,z \right) ^T$,若满足:
\begin{itemize}
    \item 在点$\boldsymbol{p}_0$的某一邻域内有连续的偏导数,且$\left. \frac{\partial u}{\partial z} \right|_{\boldsymbol{p}_0}\ne 0$,
    \item 点$\boldsymbol{p}_0$满足方程$F\left( \boldsymbol{p}_0 \right) =0$,
\end{itemize}
则点$\boldsymbol{p}_0$的某一邻域内,方程$F\left( x,y,z \right) =0$将唯一确定单值连续函数$z=z\left( x,y \right) $,使得$F\left( x,y,z\left( x,y \right) \right) =0$,且$z_0=z\left( x_0,y_0 \right) $,并有:
\begin{align*}
&\frac{\partial z}{\partial x}=-\frac{\frac{\partial u}{\partial x}}{\frac{\partial u}{\partial z}}=-\frac{F_x}{F_z} \\
&\frac{\partial z}{\partial y}=-\frac{\frac{\partial u}{\partial y}}{\frac{\partial u}{\partial z}}=-\frac{F_y}{F_z}
\end{align*}
\end{theorem}

使用隐函数的求导方法时,首先创建一个函数$u=F\left( \boldsymbol{p} \right) $,此时,$z$再不是应变量,而是自变量。

~

\begin{example}
设$x^2+y^2+z^2=1$,求$\frac{\partial ^2z}{\partial x^2}$。
\end{example}

解:

令$u=F\left( x,y,z \right) =x^2+y^2+z^2-1$,可得:
\[
\frac{\partial z}{\partial x}=-\frac{F_x}{F_z}=-\frac{x}{z}
\]
继续求偏导:
\[
\frac{\partial ^2z}{\partial x^2}=\frac{\partial}{\partial x}\left( -\frac{x}{z} \right) =-\frac{z-x\frac{\partial z}{\partial x}}{z^2}=-\frac{z+x\frac{x}{z}}{z^2}
\]

~

\begin{example}
    设$x^2+y^2+z^2=yf\left( \frac{z}{y} \right) $,其中$f\left( u \right) $对$u$可微,求$\frac{\partial z}{\partial y}$。
\end{example}

解:

令$u=F\left( x,y,z \right) =x^2+y^2+z^2-yf\left( \frac{z}{y} \right) $,可得:
\[
\frac{\partial z}{\partial y}=-\frac{F_y}{F_z}=-\frac{2y-f\left( \frac{z}{y} \right) -yf'\left( \frac{z}{y} \right) \left( -\frac{z}{y^2} \right)}{2z-yf'\left( \frac{z}{y} \right) \frac{1}{y}}=-\frac{2y-f+\frac{z}{y}f'}{2z-f'}
\]

%============================================================
\subsection{方程组的求导}

上面提到的是一个三元方程$F\left( x,y,z \right) =0$确定的一个二元函数。
如果是两个四元方程组成的方程组,如下:
\[
\left\{ \begin{array}{c}
	F\left( x,y,u,v \right) =0\\
	G\left( x,y,u,v \right) =0\\
\end{array} \right.
\]
也可以确定两个二元函数$u=u\left( x,y \right) ,v=v\left( x,y \right) $。
以求$\frac{\partial u}{\partial x},\frac{\partial v}{\partial x}$为例,将两个方程两边分别对$x$求偏导:
\[
\begin{cases}
	\frac{\partial F}{\partial x}+\frac{\partial F}{\partial u}\cdot \frac{\partial u}{\partial x}+\frac{\partial F}{\partial v}\cdot \frac{\partial v}{\partial x}=0\\
	\frac{\partial G}{\partial x}+\frac{\partial G}{\partial u}\cdot \frac{\partial u}{\partial x}+\frac{\partial G}{\partial v}\cdot \frac{\partial v}{\partial x}=0\\
\end{cases}
\]
其实就是解一个二元一次方程组,化为矩阵形式:
\[
\left[ \begin{matrix}
	\frac{\partial F}{\partial u}&		\frac{\partial F}{\partial v}\\
	\frac{\partial G}{\partial u}&		\frac{\partial G}{\partial v}\\
\end{matrix} \right] \cdot \left( \begin{array}{c}
	\frac{\partial u}{\partial x}\\
	\frac{\partial u}{\partial y}\\
\end{array} \right) =-\left( \begin{array}{c}
	\frac{\partial F}{\partial x}\\
	\frac{\partial G}{\partial y}\\
\end{array} \right)
\]
解用雅可比行列式可以写成:
\[
\begin{cases}
	\frac{\partial u}{\partial x}=-\frac{\partial \left( F,G \right)}{\partial \left( x,v \right)}/\frac{\partial \left( F,G \right)}{\partial \left( u,v \right)}\\
	\frac{\partial v}{\partial x}=-\frac{\partial \left( F,G \right)}{\partial \left( u,x \right)}/\frac{\partial \left( F,G \right)}{\partial \left( u,v \right)}\\
\end{cases}
\]

~

\begin{example}
若方程组
\[
\begin{cases}
	u^3+xv-y=0\\
	v^3+yu-x=0\\
\end{cases}
\]
可确定的函数$u=u\left( x,y \right) ,v=v\left( x,y \right) $,求偏导数$\frac{\partial u}{\partial x},\frac{\partial v}{\partial x}$。
\end{example}

解:

用上面的方法,直接写出解的公式:
\begin{align*}
&\frac{\partial \left( F,G \right)}{\partial \left( x,v \right)}=\left| \begin{matrix}
	\frac{\partial F}{\partial x}&		\frac{\partial F}{\partial v}\\
	\frac{\partial G}{\partial x}&		\frac{\partial G}{\partial v}\\
\end{matrix} \right|=v\cdot 3v^2-x\cdot \left( -1 \right) =x+3v^3 \\
&\frac{\partial \left( F,G \right)}{\partial \left( u,x \right)}=\left| \begin{matrix}
	\frac{\partial F}{\partial u}&		\frac{\partial F}{\partial x}\\
	\frac{\partial G}{\partial u}&		\frac{\partial G}{\partial x}\\
\end{matrix} \right|=3u^2\cdot \left( -1 \right) -v\cdot y=-vy-3u^2 \\
&\frac{\partial \left( F,G \right)}{\partial \left( u,v \right)}=\left| \begin{matrix}
	\frac{\partial F}{\partial u}&		\frac{\partial F}{\partial v}\\
	\frac{\partial G}{\partial u}&		\frac{\partial G}{\partial v}\\
\end{matrix} \right|=3u^2\cdot 3v^2-x\cdot y=9u^2v^2-xy
\end{align*}
解得:
\[
\begin{cases}
	\frac{\partial u}{\partial x}=-\frac{\partial \left( F,G \right)}{\partial \left( x,v \right)}/\frac{\partial \left( F,G \right)}{\partial \left( u,v \right)}=-\frac{x+3v^3}{9u^2v^2-xy}\\
	\frac{\partial v}{\partial x}=-\frac{\partial \left( F,G \right)}{\partial \left( u,x \right)}/\frac{\partial \left( F,G \right)}{\partial \left( u,v \right)}=-\frac{-vy-3u^2}{9u^2v^2-xy}\\
\end{cases}
\]

