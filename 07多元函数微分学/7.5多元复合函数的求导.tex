\section{多元复合函数的求导}

对于多元复合函数,本节给出链式求导法则。

本节要点:
\begin{itemize}
    \item 掌握链式求导法则;
    \item 掌握引入中间变量转化为复合函数求解非初等多元函数的偏导;
    \item 掌握使用链式求导求解多元函数的高阶偏导;
    \item 了解雅可比矩阵和雅可比行列式。
\end{itemize}

%============================================================
\subsection{多元复合函数的概念}

\begin{definition}[复合函数]
设二元函数$z=z\left( u,v \right) $在某开集$D$有定义,又设$u=u\left( x,y \right) ,v=v\left( x,y \right) $在开集$E$内有定义,于是由$z,u,v$可构成一个{\bf 复合函数}:
\[
z=z\left( u\left( x,y \right) ,v\left( x,y \right) \right) \qquad \left( x,y \right) \in E
\]
\end{definition}

%============================================================
\subsection{多元复合函数的链式求导法则}

\begin{theorem}[链式求导法则]
对于复合函数$z=z\left( u\left( x,y \right) ,v\left( x,y \right) \right) $,若:
\begin{itemize}
    \item $u=u\left( x,y \right) ,v=v\left( x,y \right) $在点$\left( x,y \right) $处偏导存在;
    \item $z=z\left( u,v \right) $对应点$\left( u,v \right) $可微,
\end{itemize}
则$z=z\left( u\left( x,y \right) ,v\left( x,y \right) \right) $在点$\left( x,y \right) $处的偏导数存在,且有链式法则:
\begin{align*}
&\frac{\partial z}{\partial x}=\frac{\partial z}{\partial u}\cdot \frac{\partial u}{\partial x}+\frac{\partial z}{\partial v}\cdot \frac{\partial v}{\partial x} \\
&\frac{\partial z}{\partial y}=\frac{\partial z}{\partial u}\cdot \frac{\partial u}{\partial y}+\frac{\partial z}{\partial v}\cdot \frac{\partial v}{\partial y}
\end{align*}
\end{theorem}

证明略,关键是条件2不能减弱为$z=z\left( u,v \right) $在点$\left( u,v \right) $存在偏导。

~

三种特殊情况。

1、当有三个中间变量时,即$z=z\left( u,v,w \right) $,$u=u\left( x,y \right) $,$v=v\left( x,y \right) $,$w=w\left( x,y \right) $:
\begin{align*}
&\frac{\partial z}{\partial x}=\frac{\partial z}{\partial u}\cdot \frac{\partial u}{\partial x}+\frac{\partial z}{\partial v}\cdot \frac{\partial v}{\partial x}+\frac{\partial z}{\partial w}\cdot \frac{\partial w}{\partial x} \\
&\frac{\partial z}{\partial y}=\frac{\partial z}{\partial u}\cdot \frac{\partial u}{\partial y}+\frac{\partial z}{\partial v}\cdot \frac{\partial v}{\partial y}+\frac{\partial z}{\partial w}\cdot \frac{\partial w}{\partial y}
\end{align*}

2、当只有一个中间变量时,即$z=z\left( u \right) ,u=u\left( x,y \right) $:
\begin{align*}
&\frac{\partial z}{\partial x}=\frac{dz}{du}\cdot \frac{\partial u}{\partial x} \\
&\frac{\partial z}{\partial y}=\frac{dz}{du}\cdot \frac{\partial u}{\partial y}
\end{align*}

3、当只有一个自变量时,即$z=z\left( u,v \right) ,u=u\left( x \right) ,v=v\left( x \right) $,此时的$z$称为{\bf 一元复合函数}:
\[
\frac{dz}{dx}=\frac{\partial z}{\partial u}\cdot \frac{du}{dx}+\frac{\partial z}{\partial v}\cdot \frac{dv}{dx}
\]
该导数称为{\bf $z$对$x$的全导数}。

~

\begin{example}
设$z=\left( x^2+y^2 \right) ^{\sin \left( 2x+y \right)}$,求$\frac{\partial z}{\partial x}$。
\end{example}

解:

对于这类非初等函数,可以引入中间变量,令
\[
\begin{cases}
	u=x^2+y^2\\
	v=\sin \left( 2x+y \right)\\
\end{cases}
\]
则$z=u^v$,有:
\begin{align*}
\frac{\partial z}{\partial x}&=\frac{\partial z}{\partial u}\cdot \frac{\partial u}{\partial x}+\frac{\partial z}{\partial v}\cdot \frac{\partial v}{\partial x} \\
&=vu^{v-1}\cdot 2x+u^v\ln u\cdot 2\cos \left( 2x+y \right) \\
&=2x\cdot \sin \left( 2x+y \right) \cdot \left( x^2+y^2 \right) ^{\sin \left( 2x+y \right) -1} \\
& \quad +2\left( x^2+y^2 \right) ^{\sin \left( 2x+y \right)}\cdot \ln \left( x^2+y^2 \right) \cdot \cos \left( 2x+y \right)
\end{align*}

%============================================================
\subsection{雅可比矩阵和雅可比行列式}

对比一元函数的链式求导,复合函数$y=y\left( u\left( x \right) \right) $对$x$的求导:
\[
\frac{dy}{dx}=\frac{dy}{du}\cdot \frac{du}{dx}
\]
二元函数的链式求导能不能采用这种写法?

考察二元函数的链式求导,二元函数$z=z\left( u\left( x,y \right) ,v\left( x,y \right) \right) $在对$x,y$的偏导数为:
\begin{align*}
&\frac{\partial z}{\partial x}=\frac{\partial z}{\partial u}\cdot \frac{\partial u}{\partial x}+\frac{\partial z}{\partial v}\cdot \frac{\partial v}{\partial x} \\
&\frac{\partial z}{\partial y}=\frac{\partial z}{\partial u}\cdot \frac{\partial u}{\partial y}+\frac{\partial z}{\partial v}\cdot \frac{\partial v}{\partial y}
\end{align*}
采用矩阵方程的形式,可以写成:
\[
\left( \begin{array}{c}
	\frac{\partial z}{\partial x}\\
	\frac{\partial z}{\partial y}\\
\end{array} \right) =\left[ \begin{matrix}
	\frac{\partial u}{\partial x}&		\frac{\partial v}{\partial x}\\
	\frac{\partial u}{\partial y}&		\frac{\partial v}{\partial y}\\
\end{matrix} \right] \left( \begin{array}{c}
	\frac{\partial z}{\partial u}\\
	\frac{\partial z}{\partial v}\\
\end{array} \right)
\]

~

\begin{definition}
我们称由两个二元函数$u\left( x,y \right) ,v\left( x,y \right) $的四个一阶偏导数组成的矩阵称为这两个二元函数的{\bf 雅可比矩阵},即:
\[
\left[ \begin{matrix}
	\frac{\partial u}{\partial x}&		\frac{\partial u}{\partial y}\\
	\frac{\partial v}{\partial x}&		\frac{\partial v}{\partial y}\\
\end{matrix} \right]
\]
其对应的行列式称为{\bf 雅可比行列式},记为$\frac{\partial \left( u,v \right)}{\partial \left( x,y \right)}$,即;
\[
\frac{\partial \left( u,v \right)}{\partial \left( x,y \right)}=\left| \begin{matrix}
	\frac{\partial u}{\partial x}&		\frac{\partial u}{\partial y}\\
	\frac{\partial v}{\partial x}&		\frac{\partial v}{\partial y}\\
\end{matrix} \right|
\]
于是上式可以写为:
\[
\left( \begin{array}{c}
	\frac{\partial z}{\partial x}\\
	\frac{\partial z}{\partial y}\\
\end{array} \right) =\left[ \begin{matrix}
	\frac{\partial u}{\partial x}&		\frac{\partial u}{\partial y}\\
	\frac{\partial v}{\partial x}&		\frac{\partial v}{\partial y}\\
\end{matrix} \right] ^T\left( \begin{array}{c}
	\frac{\partial z}{\partial u}\\
	\frac{\partial z}{\partial v}\\
\end{array} \right)
\]
\end{definition}

同样可扩展三个三元函数$u\left( x,y,z \right) ,v\left( x,y,z \right) ,w\left( x,y,z \right) $的九个一阶偏导数组成的雅可比矩阵:
\[
\left[ \begin{matrix}
	\frac{\partial u}{\partial x}&		\frac{\partial u}{\partial y}&		\frac{\partial u}{\partial z}\\
	\frac{\partial v}{\partial x}&		\frac{\partial v}{\partial y}&		\frac{\partial v}{\partial z}\\
	\frac{\partial w}{\partial x}&		\frac{\partial w}{\partial y}&		\frac{\partial w}{\partial z}\\
\end{matrix} \right]
\]

对于雅可比行列式,有如下法则:
\begin{align*}
&\frac{\partial \left( u,v \right)}{\partial \left( x,y \right)}=\frac{\partial \left( u,v \right)}{\partial \left( s,t \right)}\cdot \frac{\partial \left( s,t \right)}{\partial \left( x,y \right)} \\
&\frac{\partial \left( u,v \right)}{\partial \left( x,y \right)}\cdot \frac{\partial \left( x,y \right)}{\partial \left( u,v \right)}=1 \\
&\frac{\partial \left( u,v \right)}{\partial \left( x,y \right)}=-\frac{\partial \left( v,u \right)}{\partial \left( x,y \right)} \\
&\frac{\partial \left( u,u \right)}{\partial \left( x,y \right)}=0
\end{align*}




