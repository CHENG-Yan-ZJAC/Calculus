\section{光滑曲面的第二个要求}

光滑曲面的第二个要求是不折。

这里要捋清几个对应关系。
一元函数对应二维平面,对应{\it xy}直角坐标系,在二维平面中,光滑曲线的要求——不断不折,体现为连续和可导,由于可导和可微的等价性,也可以表现为连续和可微。
二元函数对应三维空间,对应{\it xyz}直角坐标系,在三维空间中,同样对光滑曲面有不折不断的要求,体现为连续和可微,这里的连续和可微都要求各个方向。

定义二元函数的导数是一件麻烦的事情。
首先,二元函数有两个变量,所以变化率会有沿着两个方向。
其次,如果我们考虑一个“总的变化率”,它的大小将随着方向的变化而变化,即总变化率是一个有方向的量,我们用矢量描述它将会比较合适。
综合下来,二元函数中没有“导数”这个概念,与一元函数导数对等的是一个叫“梯度”的概念。
梯度和可微是等价概念。更准确地讲,是一维空间中“导数”这个概念对应着多维空间中的“梯度”和“散度”两个概念。

\begin{table}[h]
\centering
\begin{tabular}{lll}
    \toprule
    概念 & 一元函数中的定义 & 二元函数中的定义\\
    \midrule
    极限        & $\underset{x\rightarrow x_0}{\lim}f\left( x \right) =A$ & $\underset{\boldsymbol{p}\rightarrow \boldsymbol{p}_0}{\lim}f\left( \boldsymbol{p} \right) =A$\\
    连续        & $\underset{x\rightarrow x_0}{\lim}f\left( x \right) =f\left( x_0 \right) $ & $\underset{\boldsymbol{p}\rightarrow \boldsymbol{p}_0}{\lim}f\left( \boldsymbol{p} \right) =f\left( \boldsymbol{p}_0 \right) $\\
    导数/梯度   & $\frac{dy}{dx}$ & $\nabla z=\left( \frac{\partial z}{\partial x}\,\,\frac{\partial z}{\partial y} \right) ^T$\\
    微分/全微分 & $dy=y'dx$ & $dz=\frac{\partial z}{\partial x}dx+\frac{\partial z}{\partial y}dy$\\
    偏导数      &  & $\frac{\partial z}{\partial x},\frac{\partial z}{\partial y}$\\
    方向导数    &  & $\frac{\partial z}{\partial n}=\frac{\partial z}{\partial x}\cos \alpha +\frac{\partial z}{\partial y}\cos \beta $\\
    \bottomrule
\end{tabular}
\end{table}

二元函数中,梯度和全微分都是建立在偏导数的基础上,所以本章先从偏导入手,再介绍全微分和梯度。




