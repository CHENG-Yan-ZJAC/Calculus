\section{偏导数和方向导数}

和一元函数一样,二元函数中我们也考察因变量对于自变量的变化率。
我们首先着重在一个维度(一个元)上考察函数的变化率——偏导,该概念用于引出后续的全微分、方向导数、梯度等概念,然后介绍方向导数,关于方向导数更多内容等讨论完全微分后继续深入。
本节着重理解偏导数的概念。

本节要点:
\begin{itemize}
    \item 掌握偏导数的定义;
    \item 了解二阶混偏相等定理;
    \item 了解方向导数的概念。
\end{itemize}

%============================================================
\subsection{偏导数的概念}

\begin{definition}[偏导数]
设函数$z=f\left( \boldsymbol{p} \right) $在点$\boldsymbol{p}_0$的某邻域内有定义,当$\boldsymbol{p}$有偏增量$\Delta _x\boldsymbol{p}$,相应地$z$也有偏增量$\Delta _xz$,如果$\Delta x\rightarrow 0$时,$\Delta _xz/\Delta x$的极限存在,则称此极限为{\bf 函数$z=f\left( \boldsymbol{p} \right) $在$\boldsymbol{p}_0$处对$x$的偏导数},记为$\left. \frac{\partial z}{\partial x} \right|_{\boldsymbol{p}_0}$,即:
\[
\left. \frac{\partial z}{\partial x} \right|_{\boldsymbol{p}_0}:=\underset{\Delta x\rightarrow 0}{\lim}\frac{\Delta _xz}{\Delta x}=\underset{\Delta x\rightarrow 0}{\lim}\frac{f\left( \boldsymbol{p}_0+\Delta _x\boldsymbol{p} \right) -f\left( \boldsymbol{p}_0 \right)}{\Delta x}
\]
也可记为$f_x\left( \boldsymbol{p}_0 \right) $。
同样定义$f_y\left( \boldsymbol{p}_0 \right) $,略。
\end{definition}

需要特别注意的是,偏导数是一个固定的实数。

\begin{definition}[偏导函数]
如果函数$z=f\left( \boldsymbol{p} \right) $在区域$D$内每一点,$\frac{\partial z}{\partial x},\frac{\partial z}{\partial y}
$存在,则称它们为{\bf $z=f\left( \boldsymbol{p} \right) $在区域$D$上的偏导函数},也可记为$f_x\left( \boldsymbol{p} \right) ,f_y\left( \boldsymbol{p} \right) $。
\end{definition}

除非特别指明,一般我们将偏导函数简称为偏导数。
需要特别注意的是,偏导函数依然是一个二元数量值函数。

由于偏导数只考虑一个变量,固定了其余所有变量,即把多元函数“退化”成一元函数,所以本质上偏导数还是一元函数中的导数。
偏导基于偏增量的概念——“固定一个,变动一个”,只要涉及偏导的定理,其证明过程都离不开偏增量概念。

二元函数极限的要求是“任意性”和“唯一性”。
而偏导数是固定了一个变量,变化另一个变化,缺乏了“任意性”,其次,二元函数在某点的各个偏导数不一定相同,缺乏“唯一性”,所以,函数在$\boldsymbol{p}_0$存在偏导数,并不能由此断定在该点存在极限,更不能断定在该点处连续。

%============================================================
\subsection{偏导数的几何意义}

对于二元函数$z=f\left( \boldsymbol{p} \right) $的偏导数$\frac{\partial z}{\partial x}$,由于固定了$y=y_0$,该函数变成了一元函数$z=f\left( x,y_0 \right) $。
几何上,是曲面$z=f\left( \boldsymbol{p} \right) $和平面$y=y_0$的交线。
偏导数的几何意义是曲线$z=f\left( x,y_0 \right) $在点$\boldsymbol{p}_0=\left( x_0\,\,y_0 \right) ^T$处的切线的斜率。

这里要注意,偏导的存在不能说明曲面是不折的,因为偏导相当于将曲面“切片”。
只是说明曲面在这个切片上投影出来的曲线是不折的。

%============================================================
\subsection{高阶偏导数的概念}

\begin{definition}[二阶偏导数]
设函数$z=f\left( \boldsymbol{p} \right) $在区域$D$内的两个偏导数$\frac{\partial z}{\partial x},\frac{\partial z}{\partial y}$都存在,进一步如果它们的偏导数也存在,则称这些偏导数是{\bf $z=f\left( \boldsymbol{p} \right) $的二阶偏导数},记为(二元函数一共4个二阶偏导):
\begin{align*}
&\frac{\partial}{\partial x}\left( \frac{\partial z}{\partial x} \right) =\frac{\partial ^2z}{\partial x^2}=f_{xx}\left( \boldsymbol{p} \right) \\
&\frac{\partial}{\partial y}\left( \frac{\partial z}{\partial x} \right) =\frac{\partial ^2z}{\partial x\partial y}=f_{xy}\left( \boldsymbol{p} \right) \\
&\frac{\partial}{\partial x}\left( \frac{\partial z}{\partial y} \right) =\frac{\partial ^2z}{\partial y\partial x}=f_{yx}\left( \boldsymbol{p} \right) \\
&\frac{\partial}{\partial y}\left( \frac{\partial z}{\partial y} \right) =\frac{\partial ^2z}{\partial y^2}=f_{yy}\left( \boldsymbol{p} \right)
\end{align*}
\end{definition}

\begin{theorem}[二阶混偏相等定理]
设函数$z=f\left( \boldsymbol{p} \right) $的两个二阶混合偏导数$\frac{\partial ^2z}{\partial x\partial y},\frac{\partial ^2z}{\partial y\partial x}$都存在,且这两个二阶混合偏导数都连续,则有
\[
\frac{\partial ^2z}{\partial x\partial y}=\frac{\partial ^2z}{\partial y\partial x}
\]
即:二阶混合偏导存在且均连续则有二阶偏导顺序无关。
\end{theorem}

%============================================================
\subsection{方向导数的概念}

\begin{definition}[方向导数]
设函数$z=f\left( \boldsymbol{p} \right) $在点$\boldsymbol{p}_0$的某邻域内有定义,由该点出发的一个方向$n$,在该方向上$\boldsymbol{p}$有方向增量$\Delta _n\boldsymbol{p}$,进而引起$z$产生方向增量$\Delta _nz$,若当$\left\| \Delta _n\boldsymbol{p} \right\| \rightarrow 0$时,$\frac{\Delta _nz}{\left\| \Delta _n\boldsymbol{p} \right\|}$存在极限,则称此极限为{\bf 函数$z=f\left( \boldsymbol{p} \right) $在点$\boldsymbol{p}_0$处沿方向$n$的方向偏导数},记作 ,即:
\[
\left. \frac{\partial z}{\partial n} \right|_{\boldsymbol{p}_0}:=\lim_{\left\| \Delta _n\boldsymbol{p} \right\| \rightarrow 0} \frac{\Delta _nz}{\left\| \Delta _n\boldsymbol{p} \right\|}=\lim_{\left\| \Delta _n\boldsymbol{p} \right\| \rightarrow 0} \frac{f\left( \boldsymbol{p}_0+\Delta _n\boldsymbol{p} \right) -f\left( \boldsymbol{p}_0 \right)}{\left\| \Delta _n\boldsymbol{p} \right\|}
\]
同样也有{\bf 方向导函数},记作$\frac{\partial z}{\partial n}$。
\end{definition}




