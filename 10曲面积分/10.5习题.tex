\section{习题}

\begin{exercise}
已知球面$x^2+y^2+z^2=R^2$,求包含在柱面$x^2+y^2=Rx$中的部分的面积。
\end{exercise}

解:

基本思路是求解一个区域内曲面的面积。
由于对称性,只计算第一卦限。
首先分析积分区域,得:
\[
S=\left\{ \left( x,y,z \right) \middle| 0\leqslant x\leqslant R,0\leqslant y\leqslant \sqrt{Rx-x^2},z=\sqrt{R^2-x^2-y^2} \right\}
\]
于是面积:
\begin{align*}
&A=4\iint\limits_S{ds}=4\iint\limits_D{\sqrt{1+{z_x}^2+{z_y}^2}dxdy}=4R\iint\limits_D{\frac{1}{\sqrt{R^2-x^2-y^2}}dxdy} \\
&D=\left\{ \left( x,y \right) \middle| 0\leqslant x\leqslant R,0\leqslant y\leqslant \sqrt{Rx-x^2} \right\}
\end{align*}
为计算简单,化为极坐标:
\begin{align*}
&A=4R\iint\limits_D{\frac{r}{\sqrt{R^2-r^2}}drd\theta} \\
&D=\left\{ \left( r,\theta \right) \middle| 0\leqslant \theta \leqslant \frac{\pi}{2},0\leqslant r\leqslant R\cos \theta \right\}
\end{align*}
解得:
\begin{align*}
A&=4R\int_0^{\frac{\pi}{2}}{d\theta \int_0^{R\cos \theta}{\frac{r}{\sqrt{R^2-r^2}}dr}} \\
&=4R\int_0^{\frac{\pi}{2}}{\left( R-R\sin \theta \right) d\theta}=\left( 2\pi -4 \right) R^2
\end{align*}


~

\begin{exercise}
设有流速场$\boldsymbol{v}\left( \boldsymbol{p} \right) =\left( x\,\,y\,\,z \right) $,求通过曲面$S:x^2+y^2+z^2=R^2,z\geqslant 0$的流量。
\end{exercise}

解:

曲面是一个位于原点的球形的上半部分。
大致思路就是构建第二类曲面积分方程,主要是被积函数和积分曲面,然后求解。

首先构建积分方程:
\[
\varPhi =\iint\limits_S{\boldsymbol{v}^T\boldsymbol{ds}}=\iint\limits_S{\left( \begin{array}{c}
	x\\
	y\\
	z\\
\end{array} \right) ^T\left( \begin{array}{c}
	-z_x\\
	-z_y\\
	1\\
\end{array} \right) dxdy}=\iint\limits_S{\left( -xz_x-yz_y+z \right) dxdy}
\]
求解$z_x,z_y$,需要用到隐函数偏导方法,构建隐函数$F\left( x,y,z \right) =x^2+y^2+z^2-R^2$:
\begin{align*}
&z_x=-\frac{F_x}{F_z}=-\frac{x}{z} \\
&z_y=-\frac{F_y}{F_z}=-\frac{y}{z}
\end{align*}
代入得:
\begin{align*}
&\varPhi =\iint\limits_S{\left( \frac{x^2}{z}+\frac{y^2}{z}+z \right) dxdy}=\iint\limits_S{\frac{R^2}{z}dxdy}=\iint\limits_D{\frac{R^2}{\sqrt{R^2-x^2-y^2}}dxdy} \\
&D=\left\{ \left( x,y \right) \middle| -R\leqslant x\leqslant R,-\sqrt{R^2-x^2}\leqslant y\leqslant \sqrt{R^2-x^2} \right\}
\end{align*}
转换到极坐标求解:
\begin{align*}
\varPhi &=\iint\limits_D{\frac{R^2}{\sqrt{R^2-r^2}}rdrd\theta}=R^2\int_0^{2\pi}{d\theta \int_0^R{\frac{r}{\sqrt{R^2-r^2}}dr}} \\
&=R^2\int_0^{2\pi}{Rd\theta}=2\pi R^3
\end{align*}




