\section{多元积分总结}

由于多元能产生矢量,所以多元积分分数量值积分和向量值积分,但无论哪个积分,最终结果都是一个标量值。

数量值积分的被积函数是数量值函数,表示数量值对于被积区域的累积,可分为重积分、线积分、面积分,向量值积分的被积函数是向量值函数,表示向量值在有向被积区域内的累积,可分为线积分和面积分。

对于数量值函数:
\begin{table}[h]
\centering
\begin{tabular}{lll}
    \toprule
    名称 & 表达式 & 物理意义\\
    \midrule
    重积分   & $\iiint_V{f\left( \boldsymbol{p} \right) dv}$ & 体质量\\
    曲线积分 & $\int_L{f\left( \boldsymbol{p} \right) dl}$   & 曲线质量\\
    曲面积分 & $\iint_S{f\left( \boldsymbol{p} \right) ds}$  & 曲面质量\\
    \bottomrule
\end{tabular}
\end{table}

对于向量值函数:
\begin{table}[h]
\centering
\begin{tabular}{lll}
    \toprule
    名称 & 表达式 & 物理意义\\
    \midrule
    曲线积分 & $\int_L{\boldsymbol{f}\left( \boldsymbol{p} \right) ^T\boldsymbol{dl}}$  & 矢量场的环流量\\
    曲面积分 & $\iint_S{\boldsymbol{f}\left( \boldsymbol{p} \right) ^T\boldsymbol{ds}}$ & 矢量场的通量\\
    \bottomrule
\end{tabular}
\end{table}




