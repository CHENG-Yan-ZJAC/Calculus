\section{第二类曲面积分}

本节讨论第二类曲面积分。
所谓“第二类曲面积分”是向量值函数对曲面的积分。

本节要点:
\begin{itemize}
    \item 掌握第二类曲面积分的概念;
    \item 掌握第二类曲面积分的计算,重点掌握积分表达式的内积计算。
\end{itemize}

%============================================================
\subsection{第二类曲面积分的概念}

\begin{definition}[第二类曲面积分]
若三维空间中有曲线$S$,$\mathbf{n}\left( \boldsymbol{p} \right) $为曲面上任一点处的单位法向量,$\boldsymbol{f}\left( \boldsymbol{p} \right) $为定义在该空间上的三元向量值函数,若内积$\boldsymbol{f}\left( \boldsymbol{p} \right) ^T\mathbf{n}\left( \boldsymbol{p} \right) $在$S$上的第一类曲面积分存在,则称此积分值为{\bf $\boldsymbol{f}\left( \boldsymbol{p} \right) $在$S$上的第二类曲面积分},记为$\iint_S{\left[ \boldsymbol{f}\left( \boldsymbol{p} \right) ^T\mathbf{n}\left( \boldsymbol{p} \right) \right] \cdot ds}$,由于$\mathbf{n}\left( \boldsymbol{p} \right) ds=\boldsymbol{ds}$为曲面微元在三个坐标平面的投影,所以更普遍地记作$\iint_S{\boldsymbol{f}\left( \boldsymbol{p} \right) ^T\boldsymbol{ds}}$,即:
\begin{align*}
&\iint\limits_S{\left[ \boldsymbol{f}\left( \boldsymbol{p} \right) ^T\mathbf{n}\left( \boldsymbol{p} \right) \right] \cdot ds}=\iint\limits_S{\boldsymbol{f}\left( \boldsymbol{p} \right) ^T\boldsymbol{ds}} \\
&:=\underset{\lambda \rightarrow 0}{\lim}\sum_{i=1}^n{\left[ \boldsymbol{f}\left( \xi _i,\eta _i,\zeta _i \right) ^T\mathbf{n}\left( \xi _i,\eta _i,\zeta _i \right) \cdot \Delta s_i \right]}
\end{align*}
其中:
\begin{itemize}
    \item $\boldsymbol{f}\left( \boldsymbol{p} \right) $:{\bf 被积函数};
    \item $\mathbf{n}\left( \boldsymbol{p} \right) $:曲面$S$的{\bf 单位法向量};
    \item $\boldsymbol{ds}=\mathbf{n}\left( \boldsymbol{p} \right) ds$:{\bf 有向曲面微元};
    \item $S$:{\bf 积分曲面}。
\end{itemize}
特别地,当$S$为封闭曲面时,记为$\oiint_S{\boldsymbol{f}\left( \boldsymbol{p} \right) ^T\boldsymbol{ds}}$。
\end{definition}

这里要注意,第二类曲面积分是有方向性的,默认方向是曲面的法方向,如果取反方向,则写为:
\[
\iint\limits_{S^-}{\boldsymbol{f}\left( \boldsymbol{p} \right) ^T\boldsymbol{ds}}
\]

%============================================================
\subsection{第二类曲面积分的计算}

\begin{theorem}[第二类曲面积分的计算公式]
我们知道有向曲面微元有:
\begin{align*}
\boldsymbol{ds}&=\mathbf{n}\left( \boldsymbol{p} \right) ds=\frac{\left( -z_x\,\,-z_y\,\,1 \right) ^T}{\sqrt{\left( z_x \right) ^2+\left( z_y \right) ^2+1}} \left[ \sqrt{\left( z_x \right) ^2+\left( z_y \right) ^2+1}\cdot dxdy \right] \\
&=\left( \begin{array}{c}
	-z_x\\
	-z_y\\
	1\\
\end{array} \right) dxdy
\end{align*}
所以向量值函数$\boldsymbol{f}\left( \boldsymbol{p} \right) =\left( P\,\,Q\,\,R \right) ^T$的第二类曲面积分的计算有:
\begin{align*}
\iint\limits_S{\boldsymbol{f}\left( \boldsymbol{p} \right) ^T\boldsymbol{ds}}&=\iint\limits_S{\left( \begin{array}{c}
	P\\
	Q\\
	R\\
\end{array} \right) ^T\left( \begin{array}{c}
	-z_x\\
	-z_y\\
	1\\
\end{array} \right) dxdy} \\
&=\iint\limits_S{\left( -Pz_x-Qz_y+R \right) dxdy}
\end{align*}
\end{theorem}

这里,第二类曲面积分的计算是将有向曲面微元投影到{\it xOy}平面,也可以投影到其他平面,看曲面以什么形式给出。
和第二类曲线积分一样,重点理解被积表达式中的内积。

曲面$S$的单位法向量可写成方向余弦形式$\mathbf{n}=\left( \cos \alpha \,\,\cos \beta \,\,\cos \gamma \right) ^T$,于是积分:
\[
\iint\limits_S{\boldsymbol{f}\left( \boldsymbol{p} \right) ^T\boldsymbol{ds}}=\iint\limits_S{\left( \begin{array}{c}
	P\\
	Q\\
	R\\
\end{array} \right) ^T\left( \begin{array}{c}
	\cos \alpha\\
	\cos \beta\\
	\cos \gamma\\
\end{array} \right) ds}
\]
而又有投影关系:
\begin{align*}
&\cos \alpha \cdot ds=dydz \\
&\cos \beta \cdot ds=dzdx \\
&\cos \gamma \cdot ds=dxdy
\end{align*}
所以第二类曲面积分还可以写为:
\[
\iint\limits_S{\boldsymbol{f}\left( \boldsymbol{p} \right) ^T\boldsymbol{ds}}=\iint\limits_S{Pdydz+Qdzdx+Rdxdy}
\]
这表明$\boldsymbol{f}\left( \boldsymbol{p} \right) $在$S$上的积分可以分解为其三个坐标分量($P,Q,R$)分别对坐标平面({\it yOz}、{\it zOx}和{\it xOy})积分的代数和。




