\section{泰勒展开}

\begin{tcolorbox}
从逻辑概念的归属来讲,泰勒定理属于级数的范畴,应该和三角级数编排在一起。
而从三角级数到傅里叶变换需要一点多元函数积分的概念,所以泰勒展开会在级数部分详细介绍。
但如果把泰勒展开完全编排在最后,等学完所有的微积分内容(特别是多元函数微积分)再学级数,那对很多工程应用来说就学了一堆不需要的东西(电子电路就不需要多元函数微积分的概念),而且大大增加学习难度。
所以一般教材的编排将泰勒定理放在这里讲述。
\end{tcolorbox}

本节讨论导数一个很重要、很典型的应用——多项式展开。

本节要点:
\begin{itemize}
    \item 掌握泰勒展开的概念;
    \item 从线性代数角度理解泰勒展开;
    \item 掌握麦克劳林公式;
    \item 熟悉泰勒展开的工程使用。
\end{itemize}

%============================================================
\subsection{泰勒定理}

之前在讨论微分时,我们将任意可导函数$f\left( x_0 \right) $在$x_0$的一个微小局部用一个线性表达式$f\left( x \right) \approx f\left( x_0 \right) +f'\left( x_0 \right) \left( x-x_0 \right) $描述。
进一步地,我们希望在$x_0$附近找到一个关于$x$的多项式$P\left( x \right) $,能够表示$x\left( x \right) $,要求很简单:
\begin{itemize}
    \item $P\left( x \right) $和$x\left( x \right) $在$x_0$有相同的函数值和1~n阶导数;
    \item $P\left( x \right) $和$x\left( x \right) $的误差能定量描述。
\end{itemize}

~

\begin{theorem}[泰勒(Taylor)定理]
若$f\left( x \right) $在$x_0$处满足n阶可导,则$f\left( x \right) $可以展开成:
\begin{align*}
f\left( x \right) &=f\left( x_0 \right) +f'\left( x_0 \right) \left( x-x_0 \right) +\frac{f''\left( x_0 \right)}{2!}\left( x-x_0 \right) ^2+\cdots \\
&\quad +\frac{f^{\left( n \right)}\left( x_0 \right)}{n!}\left( x-x_0 \right) ^n+o\left( \left( x-x_0 \right) ^n \right) \\
&=\left[ \sum_{i=0}^n{\frac{f^{\left( i \right)}\left( x_0 \right)}{i!}\left( x-x_0 \right) ^i} \right] +o\left( \left( x-x_0 \right) ^n \right)
\end{align*}
其中$R\left( x \right) :=o\left( \left( x-x_0 \right) ^n \right) $称为{\bf 佩亚诺(G. Peano)型余项}。
\end{theorem}

当$n=1$时,泰勒展开就是一阶微分形式:
\begin{align*}
&\because f\left( x \right) =f\left( x_0 \right) +f'\left( x_0 \right) \left( x-x_0 \right) +o\left( x-x_0 \right) \\
&\therefore \Delta y=f\left( x \right) -f\left( x_0 \right) =f'\left( x_0 \right) \left( x-x_0 \right) +o\left( x-x_0 \right)
\end{align*}

\begin{theorem}[泰勒(Taylor)定理]
若$f\left( x \right) $在区间$D$上满足n+1阶可导,则$\forall x,x_0\in D$,至少存在$\xi \in \left[ x,x_0 \right] $或$\xi \in \left[ x_0,x \right] $,使得:
\begin{align*}
f\left( x \right) &=f\left( x_0 \right) +f'\left( x_0 \right) \left( x-x_0 \right) +\frac{f''\left( x_0 \right)}{2!}\left( x-x_0 \right) ^2+\cdots \\
&\quad +\frac{f^{\left( n \right)}\left( x_0 \right)}{n!}\left( x-x_0 \right) ^n+\frac{f^{\left( n+1 \right)}\left( \xi \right)}{\left( n+1 \right) !}\left( x-x_0 \right) ^{n+1} \\
&=\left[ \sum_{i=0}^n{\frac{f^{\left( i \right)}\left( x_0 \right)}{i!}\left( x-x_0 \right) ^i} \right] +\frac{f^{\left( n+1 \right)}\left( \xi \right)}{\left( n+1 \right) !}\left( x-x_0 \right) ^{n+1}
\end{align*}
其中$R_n\left( x \right) :=\frac{f^{\left( n+1 \right)}\left( \xi \right)}{\left( n+1 \right) !}\left( x-x_0 \right) ^{n+1}$称为{\bf 朗格兰日(Lagrange)型余项}。
\end{theorem}

泰勒定理告诉我们,任何n阶可导函数,都可以展开成一个二项式。
虽然两个都是泰勒定理,但第二个泰勒定理的要求更严格,好处是给出了余项的表达式,可以用来估算误差。
泰勒定理其实就是一个函数的幂级数展开。

%============================================================
\subsection{从线性代数看泰勒展开}

从线性代数角度看来,泰勒展开就将函数将基变换为多项式。
原来$f\left( x \right) $的基为实数集$D$,变换后为:
\[
B=\left[ \begin{array}{c}
	1\\
	\left( x-x_0 \right)\\
	\left( x-x_0 \right) ^2\\
	\vdots\\
	\left( x-x_0 \right) ^n\\
\end{array} \right]
\]
相应地坐标为:
\[
\left[ f \right] _B=\left[ \begin{array}{c}
	f\left( x_0 \right)\\
	f'\left( x_0 \right)\\
	\frac{f''\left( x_0 \right)}{2!}\\
	\vdots\\
	\frac{f^{\left( n \right)}\left( x_0 \right)}{n!}\\
\end{array} \right]
\]
于是函数可以描述为:
\[
f\left( x \right) =B^T\left[ f \right] _B+R_n\left( x \right)
\]

%============================================================
\subsection{麦克劳林公式}

\begin{definition}[n阶麦克劳林(Maclaurin)公式]
{\bf 麦克劳林公式}是函数在0点处的泰勒展开,取$x_0=0$,泰勒展开公式变为:
\begin{align*}
f\left( x \right) &=f\left( 0 \right) +f'\left( 0 \right) \cdot x+\frac{f''\left( 0 \right)}{2!}\cdot x^2+\cdots +\frac{f^{\left( n \right)}\left( 0 \right)}{n!}\cdot x^n+o\left( x \right) \\
&=\left[ \sum_{i=0}^n{\frac{f^{\left( i \right)}\left( 0 \right)}{i!}\cdot x^i} \right] +o\left( x \right)
\end{align*}
\end{definition}

~

几个常用函数的麦克劳林公式:
\begin{align*}
&\left( 1+x \right) ^a=1+ax+\frac{a\left( a-1 \right)}{2!}x^2+\cdots +\frac{a\left( a-1 \right) \cdots \left( a-n \right)}{n!}x^n+o\left( x \right) \\
&e^x=1+x+\frac{x^2}{2!}+\cdots +\frac{x^n}{n!}+o\left( x \right) \\
&\sin x=x-\frac{x^3}{3!}+\frac{x^5}{5!}-\cdots +\left( -1 \right) ^{m-1}\frac{x^{2m-1}}{\left( 2m-1 \right) !}+o\left( x \right) \\
&\cos x=1-\frac{x^2}{2!}+\frac{x^4}{4!}-\cdots +\left( -1 \right) ^m\frac{x^{2m}}{\left( 2m \right) !}+o\left( x \right)
\end{align*}

特别地,一般在工程中常用:
\begin{align*}
&\left( 1+x \right) ^a=1+ax \\
&e^x=1+x \\
&\sin x=x \\
&\cos x=1-\frac{x^2}{2!}
\end{align*}

麦克劳林公式常用于工程上的公式推导,通过麦克劳林公式将超越方程简化成一次或二次方程,进而求解解析式。
如求解方程$e^x=\cos x-1$,在0附近可以用方程$1+x=-x^2/2!$代替求解。

%============================================================
\subsection{泰勒展开的工程意义}

要估算一个复杂的函数,通常的做法是用另一个简单的函数代替,即“化繁为简”。
这种方法(或思想)称为级数。
代替的函数必须简单,最好是初等函数的组合。
数学上有三种可行的初等函数代替展开:
\begin{itemize}
    \item 幂函数;
    \item 三角函数;
    \item 指数函数。
\end{itemize}
但由于欧拉公式,所以实际只剩下幂函数和三角函数两种方法。

幂函数展开的方法之一就是泰勒展开,三角函数展开就是傅里叶级数。
泰勒展开由于采用幂函数将超越方程化为二次方程,所以多用于手解超越方程或判断两个超越方程的大小关系。
傅里叶级数采用三角函数,使得原函数由一系列不同角频的正弦函数组成,所以通常用于考察一个函数的特征。




