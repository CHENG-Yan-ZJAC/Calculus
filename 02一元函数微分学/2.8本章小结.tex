\section{本章小结}

形而下来讲,本章引入了微积分里的灵魂概念——导数。
如果将微积分定位为工程上的数学工具,则导数是这个工具中的数学基础,微分是作为数学的导数的第一个工程应用,如何用一个微小量代替全部量。
形而上来讲,本章解决了光滑的第二个要求。
至此,光滑的好函数——不断不折——在数学上有了严格的定义。

我们要将“光滑”这个概念作为微积分的出发点和落脚点,深入理解极限、连续、导数、微分这些概念。
我们喜欢的是光滑曲线,我们要考察的是光滑曲线,我们要寻找使用的也是光滑曲线。
所以我们首先定义光滑——不断不折。
通过连续定义什么是不断,通过可导定义什么是不折。
接着,我们考察光滑曲线的特征。
光滑的特征体现在连续上是最值、有界、介值三个定理。
由于可导等价于可微,所以不折的特征体现在可导上是微分中值定理,特别是拉格朗日定理。
充分领悟光滑这个概念,这些定理将会是无比自然。
最后,我们从“微”的角度考察光滑的好处,即光滑的实际价值。
我们提出了微分的概念,用一个线性小量代替绝对差量,当然,这得首先要求是光滑的。

至此,我们解决了光滑的问题,讨论了光滑在“微”方面的意义。
下一章积分,我们要讨论光滑在“累积”方面有哪些特征,可以帮助我们解决什么问题。




