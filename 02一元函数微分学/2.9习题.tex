\section{习题}

\begin{exercise}
求曲线$y=\cos x$在点$\left( \frac{\pi}{3},\frac{1}{2} \right) $处的切线和法线方程。
\end{exercise}

解:

使用曲线的切线和法线的公式即可。
\begin{align*}
&y=f'\left( x_0 \right) x+\left[ f\left( x_0 \right) -f'\left( x_0 \right) x_0 \right] \\
&y=-\frac{1}{f'\left( x_0 \right)}x+\left[ f\left( x_0 \right) +\frac{1}{f'\left( x_0 \right)}x_0 \right]
\end{align*}
得到切线和法线方程:
\begin{align*}
&\because f'\left( \frac{\pi}{3} \right) =-\sin \left( \frac{\pi}{3} \right) =-\frac{\sqrt{3}}{2} \\
&\therefore \begin{cases}
	y=-\frac{\sqrt{3}}{2}x+\left( \frac{1}{2}+\frac{\sqrt{3}}{2}\frac{\pi}{3} \right)\\
	y=\frac{1}{\frac{\sqrt{3}}{2}}x+\left( \frac{1}{2}+\frac{1}{-\frac{\sqrt{3}}{2}}\frac{\pi}{3} \right) =\frac{2}{\sqrt{3}}x+\left( \frac{1}{2}-\frac{2}{\sqrt{3}}\frac{\pi}{3} \right)\\
\end{cases}
\end{align*}

~

\begin{exercise}
求曲线$e^{xy}-2x-y=3$上点$\left( -1,0 \right) $所对应的切线方程。
\end{exercise}

解:

大体思路一样,只是需要用到隐函数求导:
\begin{align*}
&\because \frac{dy}{dx}=-\frac{F_x}{F_y}=-\frac{ye^{xy}-2}{xe^{xy}-1} \\
&\therefore f'\left( x_0 \right) =-\frac{-2}{-1-1}=-1 \\
&\therefore y=f'\left( x_0 \right) x+\left[ f\left( x_0 \right) -f'\left( x_0 \right) x_0 \right] =-x-1
\end{align*}

~

\begin{exercise}
求过点$\left( 3,0 \right) $与曲线$y=x^2$相切的直线方程。
\end{exercise}

解:

注意点$\left( 3,0 \right) $并不是曲线上的点,而是切线过的点。
假设切线切与点$x_0$,易得切线方程:
\[
y=2x_0\cdot x+\left( {x_0}^2-2x_0x_0 \right) =2x_0\cdot x-{x_0}^2
\]
带入点$\left( 3,0 \right) $求得$x_0$:
\begin{align*}
&\because 0=6x_0-{x_0}^2 \\
&\therefore x_0=0\,\,\mathrm{or}\,\,6
\end{align*}
于是得到2条切线:
\begin{align*}
&y=0 \\
&y=12x-36
\end{align*}

~

\begin{exercise}
在抛物线$y=1-x^2$上求两点,使得过这两点的切线与{\it x}轴形成一个等边三角形。
\end{exercise}

解:

首先假设切点为$x_0$,获得切线方程:
\[
y=\left( -2x_0 \right) x+\left[ \left( 1-{x_0}^2 \right) -\left( -2x_0 \right) x_0 \right] =\left( -2x_0 \right) x+\left( 1+{x_0}^2 \right)
\]
若与{\it x}轴形成等边三角形,则这两条直线的斜率必为$\pm \sqrt{3}$,于是:
\begin{align*}
&\because -2x_0=\pm \sqrt{3}\Rightarrow x_0=\pm \frac{\sqrt{3}}{2} \\
&\therefore \left( \pm \frac{\sqrt{3}}{2},\frac{1}{4} \right)
\end{align*}

~

\begin{exercise}
求和$y=x^2,y=-x^2+6x-5$两条曲线都相切的直线方程。
\end{exercise}

解:

假设和两条曲线分别切于点$x_1,x_2$,分别根据两条曲线易得切线:
\begin{align*}
y&=2x_1x+\left( {x_1}^2-2x_1x_1 \right) =2x_1x-{x_1}^2 \\
y&=\left( -2x_2+6 \right) x+\left[ \left( -{x_2}^2+6x_2-5 \right) -\left( -2x_2+6 \right) x_2 \right] \\
&=\left( -2x_2+6 \right) x+\left( {x_2}^2-5 \right)
\end{align*}
由于这两条切线是同一条,于是可以得到:
\[
\begin{cases}
	2x_1=-2x_2+6\\
	-{x_1}^2={x_2}^2-5\\
\end{cases}\Rightarrow \begin{cases}
	x_1=1\\
	x_2=2\\
\end{cases} \quad \mathrm{or} \quad \begin{cases}
	x_1=2\\
	x_2=1\\
\end{cases}
\]
即有两条切线都可以和$y=x^2,y=-x^2+6x-5$相切:
\begin{align*}
&y=2x-1 \\
&y=4x-4
\end{align*}

~

\begin{exercise}
给定椭圆$4x^2+y^2=5$,求与此椭圆相切于$\left( \pm 1,-1 \right) $的抛物线方程。
\end{exercise}

解:

椭圆和抛物线相切,则必有一切线,同时切于这两个曲线。
先求和椭圆的切于$\left( \pm 1,-1 \right) $的切线方程,椭圆方程的隐函数为$F=4x^2+y^2-5$,则切线:
\begin{align*}
&\because y'\left( x_0 \right) =-\frac{F_x}{F_y}=-\frac{8x_0}{2y_0} \\
&\therefore y=f'\left( x_0 \right) x+\left[ f\left( x_0 \right) -f'\left( x_0 \right) x_0 \right] =-\frac{8x_0}{2y_0}x+\left[ y_0+\frac{8x_0}{2y_0}x_0 \right] \\
&\therefore \begin{cases}
	y=-\frac{8x_0}{2y_0}x+\left[ y_0+\frac{8x_0}{2y_0}x_0 \right] =4x-5\\
	y=-\frac{8x_0}{2y_0}x+\left[ y_0+\frac{8x_0}{2y_0}x_0 \right] =-4x-5\\
\end{cases}
\end{align*}
假设抛物线方程$y=ax^2+bx+c$,则有切线:
\begin{align*}
&\because y'\left( x_0 \right) =2ax_0+b \\
&\therefore y=f'\left( x_0 \right) x+\left[ f\left( x_0 \right) -f'\left( x_0 \right) x_0 \right] =\left( 2ax_0+b \right) x+\left[ y_0-\left( 2ax_0+b \right) x_0 \right] \\
&\therefore \left\{ \begin{aligned}
	y&=\left( 2ax_0+b \right) x+\left[ y_0-\left( 2ax_0+b \right) x_0 \right]\\
	&=\left( 2a+b \right) x+\left[ -1-\left( 2a+b \right) \right]\\
	y&=\left( 2ax_0+b \right) x+\left[ y_0-\left( 2ax_0+b \right) x_0 \right]\\
	&=\left( -2a+b \right) x+\left[ -1+\left( -2a+b \right) x_0 \right]\\
\end{aligned} \right.
\end{align*}
于是得到:
\[
\begin{cases}
	2a+b=4\\
	-1-\left( 2a+b \right) =-5\\
	-2a+b=-4\\
	-1+\left( -2a+b \right) =-5\\
\end{cases}
\]
解得$a=2,b=0$后得到抛物线方程$y=2x^2+c$,再将$\left( \pm 1,-1 \right) $代入求得$c=-3$,于是最终得到抛物线方程:
\[
y=2x^2-3
\]

~

\begin{exercise}
若$f\left( x \right) $对任意实数$x_1,x_2$有$f\left( x_1+x_2 \right) =f\left( x_1 \right) f\left( x_2 \right) $,且$f'\left( 0 \right) =1$,证明$f'\left( x \right) =f\left( x \right) $。
\end{exercise}

解:

总体思路求解$f'\left( x \right) $。

首先我们可以根据$f\left( x_1+x_2 \right) =f\left( x_1 \right) f\left( x_2 \right) $得到:
\begin{align*}
&\because f\left( x_1+x_2 \right) =f\left( x_1 \right) f\left( x_2 \right) \\
&\therefore f\left( x \right) =f\left( 0 \right) f\left( x \right) \\
&\therefore f\left( 0 \right) =0\,\,\mathrm{or}\,\,1
\end{align*}
其次,将$f\left( x_1+x_2 \right) =f\left( x_1 \right) f\left( x_2 \right) $两边求导得到:
\begin{align*}
&\because f'\left( x_1+x_2 \right) =\left[ f\left( x_1 \right) f\left( x_2 \right) \right] '=f'\left( x_1 \right) f\left( x_2 \right) +f\left( x_1 \right) f'\left( x_2 \right) \\
&\therefore f'\left( x \right) =f'\left( x+0 \right) =f'\left( x \right) f\left( 0 \right) +f\left( x \right) f'\left( 0 \right) =f'\left( x \right) f\left( 0 \right) +f\left( x \right)
\end{align*}
当$f\left( 0 \right) =0$时,有
\[
f'\left( x \right) =f'\left( x \right) f\left( 0 \right) +f\left( x \right) =f\left( x \right)
\]
当$f\left( 0 \right) =1$时,有
\[
f'\left( x \right) =f'\left( x \right) f\left( 0 \right) +f\left( x \right) =f'\left( x \right) +f\left( x \right)
\]
此时解得函数表达式为$f\left( x \right) =0$,也即$f'\left( x \right) =f\left( x \right) $,证毕。




