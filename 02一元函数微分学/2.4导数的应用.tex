\section{导数的应用}

导数是描述一个函数的变化率,所以通过导数我们可以分析一个函数的变化趋势、形态、走向,进而用多项式近似该函数。
微分中值定理解决前者,泰勒展开和麦克劳林公式解决后者。

本节要点:
\begin{itemize}
    \item 掌握各个定理。
\end{itemize}

%============================================================
\subsection{微分中值定理}

\begin{theorem}[费马定理]
设$f\left( x \right) $在点$x_0$处可导,若$x_0$处是$f\left( x \right) $的一个极值点,则$f'\left( x_0 \right) =0$。
\end{theorem}

\begin{theorem}[罗尔定理]
设$f\left( x \right) $在$\left[ a,b \right] $连续,在$\left( a,b \right) $可导,若$f\left( a \right) =f\left( b \right) $,则必有$\xi \in \left( a,b \right) $,使得$f'\left( \xi \right) =0$。
即两端相等的光滑曲线必有极值点。
\end{theorem}

\begin{theorem}[拉格朗日定理]
设$f\left( x \right) $在$\left[ a,b \right] $连续,在$\left( a,b \right) $可导,则必有$\xi \in \left( a,b \right) $,使得$f'\left( \xi \right) =\frac{f\left( b \right) -f\left( a \right)}{b-a}$。
即可作切线和割线平行。
\end{theorem}

\begin{theorem}[柯西定理]
设$f\left( x \right) $在$\left[ a,b \right] $连续,在$\left( a,b \right) $可导,若$g'\left( x_0 \right) \ne 0$,则必有$\xi \in \left( a,b \right) $,使得$\frac{f'\left( \xi \right)}{g'\left( \xi \right)}=\frac{f\left( b \right) -f\left( a \right)}{g\left( b \right) -g\left( a \right)}$。
\end{theorem}

微分中值定理建立了函数在考察区间内的变化形态。
这四个微分中值定理,前提都是要求可导,说明被考察对象都是光滑曲线。
而光滑,则直觉上必然会有这些定理描述的现象。
充分理解光滑的含义,这些定理也将是不证自明。

费马定理可用于求解区间内的最大值最小值问题。

罗尔定理可用于判断一个方程有无根,考察它的原函数在区间内连续可导,如果区间端点值相等,则必有根。

拉格朗日定理和后面的积分中值定理是对应关系,以不同的角度描述了一个事情,一个从微分角度,一个从积分角度。

柯西定理是拉格朗日定理的推广。

%============================================================
\subsection{洛必达法则}

\begin{theorem}[洛必达法则(L'Hopital's rule)]
若$f\left( x \right) ,g\left( x \right) $满足:
\begin{itemize}
    \item $\underset{x\rightarrow x_0}{\lim}f\left( x \right) =0,\underset{x\rightarrow x_0}{\lim}g\left( x \right) =0$;
    \item 在点$x_0$的去心邻域内$f'\left( x \right) ,g'\left( x \right) $均存在,且$g'\left( x \right) \ne 0$;
    \item $\underset{x\rightarrow x_0}{\lim}\frac{f'\left( x \right)}{g'\left( x \right)}$存在(或为$\infty $);
\end{itemize}
则有:
\[
\underset{x\rightarrow x_0}{\lim}\frac{f\left( x \right)}{g\left( x \right)}=\underset{x\rightarrow x_0}{\lim}\frac{f'\left( x \right)}{g'\left( x \right)}
\]
\end{theorem}

洛必达法则用于求解$\frac{0}{0}$和$\frac{\infty}{\infty}$之类的不定型极限。
注意,其他类型的极限不能用洛必达法则。




