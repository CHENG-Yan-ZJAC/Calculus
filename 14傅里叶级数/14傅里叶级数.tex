\chapter{傅里叶级数}

本章讨论另一种函数代替的方法,傅里叶级数。

首先介绍傅里叶级数的两个形式,三角形式和复数形式,讨论三角函数系的正交性和介绍级数收敛条件,然后简单介绍傅里叶变换。

法国数学家傅里叶(J. Fourier,1768~1830)于1822年在研究热传导理论时提出并证明了周期函数的正弦级数展开,之后泊松(Poisson)、高斯(Gauss)等人将这一成功运用到电学,后又经过多年发展形成了一个系统的分析方法。
现在傅里叶变换已广泛应用于信号分析、系统设计等各个工程领域。

本章要点:
\begin{itemize}
    \item 傅里叶级数的三角形式和复数形式。
    \item 傅里叶积分、傅里叶级数变换。
\end{itemize}

~

\newpage
\section{傅里叶级数的三角形式}

本节讨论傅里叶级数的三角级数形式。

本节要点:
\begin{itemize}
    \item 理解三角函数系的正交性;
    \item 掌握傅里叶三角级数的概念;
    \item 掌握傅里叶三角级数的展开;
    \item 深刻理解三角级数的向量空间意义。
\end{itemize}

%============================================================
\subsection{三角函数系的正交性}

\begin{theorem}[三角函数系正交性定理]
设所有在区间$\left[ -\pi ,\pi \right] $上连续的函数构成一个向量空间,记作$C\left[ -\pi ,\pi \right] $,则下列三角函数系:
\[
1,\sin x,\cos x,\sin 2x,\cos 2x,\cdots ,\sin nx,\cos nx,\cdots \qquad n=1,2,\cdots
\]
构成$C\left[ -\pi ,\pi \right] $上的一个正交基。
\end{theorem}

完整证明过程略,这里验证它们的正交性:
\begin{align*}
\left< 1,\sin nx \right> &=\int_{-\pi}^{\pi}{\left( 1\cdot \sin nx \right) \cdot dx}=\left. \frac{-1}{n}\cos nx \right|_{-\pi}^{\pi}=0 \\
\left< 1,\cos nx \right> &=\int_{-\pi}^{\pi}{\left( 1\cdot \cos nx \right) \cdot dx}=\left. \frac{1}{n}\sin nx \right|_{-\pi}^{\pi}=0 \\
\left< \sin mx,\cos nx \right> &=\int_{-\pi}^{\pi}{\left( \sin mx\cdot \cos nx \right) \cdot dx} \\
&=\frac{1}{2}\int_{-\pi}^{\pi}{\left[ \sin \left( m+n \right) x+\sin \left( m-n \right) x \right] \cdot dx}=0 \\
\left< \sin mx,\sin nx \right> &=\int_{-\pi}^{\pi}{\left( \sin mx\cdot \sin nx \right) \cdot dx} \\
&=\frac{1}{2}\int_{-\pi}^{\pi}{\left[ -\cos \left( m+n \right) x+\cos \left( m-n \right) x \right] \cdot dx}=0 \\
\left< \cos mx,\cos nx \right> &=\int_{-\pi}^{\pi}{\left( \cos mx\cdot \cos nx \right) \cdot dx} \\
&=\frac{1}{2}\int_{-\pi}^{\pi}{\left[ \cos \left( m+n \right) x+\cos \left( m-n \right) x \right] \cdot dx}=0
\end{align*}
值得注意的是,它们本身并不规范,基并也不等长,但不影响讨论:
\begin{align*}
\left\| 1 \right\| ^2&=\left< 1,1 \right> =\int_{-\pi}^{\pi}{\left( 1\cdot 1 \right) \cdot dx}=\left. \frac{1}{2}x \right|_{-\pi}^{\pi}=2\pi \\
\left\| \sin nx \right\| ^2&=\left< \sin nx,\sin nx \right> =\int_{-\pi}^{\pi}{\left( \sin nx\cdot \sin nx \right) \cdot dx} \\
&=\frac{1}{2}\int_{-\pi}^{\pi}{\left( -\cos 2nx+1 \right) \cdot dx}=\pi \\
\left\| \cos nx \right\| ^2&=\left< \cos nx,\cos nx \right> =\int_{-\pi}^{\pi}{\left( \cos nx\cdot \cos nx \right) \cdot dx} \\
&=\frac{1}{2}\int_{-\pi}^{\pi}{\left( \cos 2nx+1 \right) \cdot dx}=\pi
\end{align*}

%============================================================
\subsection{狄利克雷充分条件}

\begin{definition}[狄利克雷充分条件]
对于周期函数$f\left( x \right) ,T=2\pi $,下述两个条件:
\begin{itemize}
    \item 满足在$\left[ -\pi ,\pi \right] $上连续;
    \item 只有有限个一类间断点,而且只有有限个极值点,
\end{itemize}
称为{\bf 狄利克雷充分条件},又称{\bf 狄利克雷收敛条件}。
\end{definition}

%============================================================
\subsection{傅里叶三角级数的概念}

\begin{definition}[傅里叶三角级数]
若周期函数$f\left( x \right) $满足狄利克雷收敛条件,则函数在$\left[ -\pi ,\pi \right] $上可以展开为三角级数,且该级数收敛于$f\left( x \right) $,即:
\begin{align*}
&f\left( x \right) =a_0+\sum_{n=1}^{\infty}{a_n\cos nx}+\sum_{n=1}^{\infty}{b_n\sin nx} \qquad x\in \left[ -\pi ,\pi \right]\\
&\begin{cases}
	a_0=\frac{1}{2\pi}\int_{-\pi}^{\pi}{\left[ f\left( x \right) \cdot 1 \right] \cdot dx}\\
	a_n=\frac{1}{\pi}\int_{-\pi}^{\pi}{\left[ f\left( x \right) \cdot \cos nx \right] \cdot dx}\\
	b_n=\frac{1}{\pi}\int_{-\pi}^{\pi}{\left[ f\left( x \right) \cdot \sin nx \right] \cdot dx}\\
\end{cases} \qquad n=1,2,\cdots
\end{align*}
该展开式称为{\bf $f\left( x \right) $的傅里叶三角级数},$a_0,a_n,b_n$称为{\bf 傅里叶三角系数},且该级数的和函数有:
\begin{itemize}
    \item 当$f\left( x \right) $在$x_0$处连续时,$S\left( x_0 \right) =f\left( x_0 \right) $;
    \item 当$f\left( x \right) $在$x_0$处不连续时,$S\left( x_0 \right) =\frac{1}{2}\left[ f\left( {x_0}^+ \right) +f\left( {x_0}^- \right) \right] $。
\end{itemize}
\end{definition}

需要特别注意,狄利克雷收敛条件只要求$f\left( x \right) $在$\left[ -\pi ,\pi \right] $有定义,至于出了这个范围有没有定义没有要求。
从线性代数角度看,由于三角函数系构成了$\left[ -\pi ,\pi \right] $上的一组正交基,所以$f\left( x \right) $能在$C\left[ -\pi ,\pi \right] $进行级数展开的前提条件就是$f\left( x \right) \in C\left[ -\pi ,\pi \right] $。

\begin{tcolorbox}
为区分之后的傅里叶级数的复数形式,本笔记特称为“傅里叶三角级数”。
一般教材中称为“傅里叶级数”。
\end{tcolorbox}

简单证明,为求$a_0$,等式两边乘以$1$后作积分,由于三角函数的正交性,易得:
\begin{align*}
&\begin{aligned}
	\because \int_{-\pi}^{\pi}{1\cdot f\left( x \right) dx}&=\int_{-\pi}^{\pi}{\left[ \left( a_0+\sum_{n=1}^{\infty}{a_n\cos nx}+\sum_{n=1}^{\infty}{b_n\sin nx} \right) \cdot 1 \right] dx}\\
	&=\int_{-\pi}^{\pi}{a_0dx}=2\pi a_0\\
\end{aligned} \\
&\therefore a_0=\frac{1}{2\pi}\int_{-\pi}^{\pi}{\left[ f\left( x \right) \cdot 1 \right] \cdot dx}
\end{align*}
同理,等式两边乘以$\cos nx$后作积分,由于正交性易得:
\begin{align*}
&\begin{aligned}
	\because &\int_{-\pi}^{\pi}{\left[ f\left( x \right) \cdot \cos nx \right] \cdot dx} \\
    &=\int_{-\pi}^{\pi}{\left[ \left( a_0+\sum_{n=1}^{\infty}{a_n\cos nx}+\sum_{n=1}^{\infty}{b_n\sin nx} \right) \cdot \cos nx \right] \cdot dx}\\
	&=\int_{-\pi}^{\pi}{\left[ a_n\cos nx\cdot \cos nx \right] \cdot dx}=\pi a_n\\
\end{aligned} \\
&\therefore a_n=\frac{1}{\pi}\int_{-\pi}^{\pi}{f\left( x \right) \cos nxdx}
\end{align*}
同理可得$b_n$,略。

注意,由于我们选取的基$1$和$\sin nx,\cos nx$并不的长,所以$a_0$和$a_n,b_n$差了一个$1/2$。

%============================================================
\subsection{傅里叶三角级数的向量空间意义}

从线性代数的角度看,傅里叶级数的本质是空间变换,将原本在以{\it x}轴为基的空间的函数映射到以三角函数为基的空间。
变换过程在微积分看来是用积分求解傅里叶系数,但在线性代数看来就是坐标系变换。

关于狄利克雷充分条件,$f\left( x \right) $能在$\left[ -\pi ,\pi \right] $进行级数展开的前提条件就是$f\left( x \right) $为属于该向量空间的向量,也即$f\left( x \right) \in C\left[ -\pi ,\pi \right] $。
但我们在讨论时可以放宽一些,$f\left( x \right) $可以有有限个不连续点,但这类间断点必须都是一类,也即在整体效果上$f\left( x \right) $必须可积。

关于正交基,我们也可以把负的三角函数包括进来:
\begin{align*}
&\cdots ,\sin \left( -nx \right) ,\cdots ,\sin \left( -2x \right) ,\sin \left( -x \right) ,1,\sin x,\sin 2x,\cdots ,\sin nx,\cdots \\
&\cdots ,\cos \left( -nx \right) ,\cdots ,\cos \left( -2x \right) ,\cos \left( -x \right) ,1,\cos x,\cos 2x,\cdots ,\cos nx,\cdots
\end{align*}
同样也是正交的,证明略。
但是多出来的共轭部分构成了非“基”的向量,也就是这组函数系虽然是正交集但不是正交基,除了把坐标系复杂化,没有任何其他意义。

%============================================================
\subsection{《信号与系统》中的傅里叶三角级数}

《信号与系统》介绍傅里叶三角级数通常以这样的方程开始:
\[
f\left( t \right) =A_0+\sum_{n=1}^{\infty}{A_n\sin \left( n\omega _0t+\varphi _n \right)}
\]
也即选择正弦函数作为正交基。
我们不重新开始推导,而是从已有傅里叶三角级数开始,令$c_n=\sqrt{{a_n}^2+{b_n}^2},\tan \varphi _n=\frac{a_n}{b_n}$,则有$\sin \varphi _n=\frac{a_n}{c_n},\cos \varphi _n=\frac{b_n}{c_n}$,于是傅里叶三角级数为:
\begin{align*}
&\begin{aligned}
	f\left( x \right) &=a_0+\sum_{n=1}^{\infty}{\left( a_n\cos nx+b_n\sin nx \right)}\\
	&=a_0+\sum_{n=1}^{\infty}{c_n\left( \sin \varphi _n\cdot \cos nx+\cos \varphi _n\cdot \sin nx \right)}\\
	&=a_0+\sum_{n=1}^{\infty}{c_n\sin \left( nx+\varphi _n \right)}\\
\end{aligned} \\
&\begin{cases}
	a_0=\frac{1}{2\pi}\int_{-\pi}^{\pi}{\left[ f\left( x \right) \cdot 1 \right] \cdot dx}\\
	c_n=\sqrt{\left( \frac{1}{\pi}\int_{-\pi}^{\pi}{\left[ f\left( x \right) \cdot \cos nx \right] \cdot dx} \right) ^2+\left( \frac{1}{\pi}\int_{-\pi}^{\pi}{\left[ f\left( x \right) \cdot \sin nx \right] \cdot dx} \right) ^2}\\
	\varphi _n=\mathrm{arc}\tan \frac{\frac{1}{\pi}\int_{-\pi}^{\pi}{\left[ f\left( x \right) \cdot \cos nx \right] \cdot dx}}{\frac{1}{\pi}\int_{-\pi}^{\pi}{\left[ f\left( x \right) \cdot \sin nx \right] \cdot dx}}\\
\end{cases}
\end{align*}
令$x=\omega _0t,dx=\omega _0dt$,整理后:
\begin{align*}
&f\left( \omega _0t \right) =\frac{\omega _0}{\pi}a_0+\frac{\omega _0}{\pi}\sum_{n=1}^{\infty}{c_n\sin \left( n\omega _0t+\varphi _n \right)} \\
&\begin{cases}
	a_0=\frac{1}{2}\int_{-\pi}^{\pi}{\left[ f\left( \omega _0t \right) \cdot 1 \right] \cdot dt}\\
	c_n=\sqrt{\left( \int_{-\pi}^{\pi}{\left[ f\left( \omega _0t \right) \cdot \cos n\omega _0t \right] \cdot dt} \right) ^2+\left( \int_{-\pi}^{\pi}{\left[ f\left( \omega _0t \right) \cdot \sin n\omega _0t \right] \cdot dt} \right) ^2}\\
	\varphi _n=\mathrm{arc}\tan \frac{\int_{-\pi}^{\pi}{\left[ f\left( \omega _0t \right) \cdot \cos n\omega _0t \right] \cdot dt}}{\int_{-\pi}^{\pi}{\left[ f\left( \omega _0t \right) \cdot \sin n\omega _0t \right] \cdot dt}}\\
\end{cases}
\end{align*}
通常在微积分里,我们忽略相位$\varphi _n$,专心讨论空间变换。
在信号与系统中,我们会一直保留$f\left( t \right) =A_0+\sum_{n=1}^{\infty}{A_n\sin \left( n\omega _0t+\varphi _n \right)}$这个形式直到傅里叶级数的复数形式。

%============================================================
\subsection{正弦级数和余弦级数}

\begin{definition}
设周期函数$f\left( x \right) ,T=2\pi $,当其展开成傅里叶级数时,有:
\begin{itemize}
    \item 若$f\left( x \right) $是奇函数,此时的展开称为{\bf 正弦级数},傅里叶系数为:
    \[
    \begin{cases}
        a_0=0\\
        a_n=0\\
        b_n=\frac{1}{\pi}\int_{-\pi}^{\pi}{f\left( x \right) \sin nxdx}\\
    \end{cases} \quad n=1,2,\cdots
    \]
    \item 若$f\left( x \right) $是偶函数,此时的展开称为{\bf 余弦级数},傅里叶系数为:
    \[
    \begin{cases}
        a_0=\frac{2}{\pi}\int_0^{\pi}{f\left( x \right) dx}\\
        a_n=\frac{2}{\pi}\int_0^{\pi}{f\left( x \right) \cos nxdx}\\
        b_n=0\\
    \end{cases} \quad n=1,2,\cdots
    \]
\end{itemize}
\end{definition}

%============================================================
\subsection{任意周期的傅里叶三角级数}

\begin{theorem}
设周期函数$f\left( x \right) $,当在$\left[ -T/2,T/2 \right] $上满足狄利克雷充分条件时,有傅里叶级数:
\begin{align*}
&f\left( x \right) =a_0+\sum_{n=1}^{\infty}{\left( a_n\cos \frac{2n\pi}{T}x+b_n\sin \frac{2n\pi}{T}x \right)} \quad x\in \left[ -\frac{T}{2},\frac{T}{2} \right] \\
&\begin{cases}
	a_0=\frac{1}{T}\int_{-\frac{T}{2}}^{\frac{T}{2}}{\left[ f\left( x \right) \cdot 1 \right] \cdot dx}\\
	a_n=\frac{2}{T}\int_{-\frac{T}{2}}^{\frac{T}{2}}{\left[ f\left( x \right) \cdot \cos \frac{2n\pi}{T}x \right] \cdot dx}\\
	b_n=\frac{2}{T}\int_{-\frac{T}{2}}^{\frac{T}{2}}{\left[ f\left( x \right) \cdot \sin \frac{2n\pi}{T}x \right] \cdot dx}\\
\end{cases} \quad n=1,2,\cdots
\end{align*}
\end{theorem}

该定理将傅里叶级数从一个三角函数周期扩展到任意周期。

%============================================================
\subsection{例}

\begin{example}
设周期函数$f\left( x \right) =x,x\in \left[ -\pi ,\pi \right) $的周期为$2\pi $,求其傅里叶级数。
\end{example}

解:

求解傅里叶系数:
\begin{align*}
&a_0=\frac{1}{2\pi}\int_{-\pi}^{\pi}{\left[ f\left( x \right) \cdot 1 \right] \cdot dx}=\frac{1}{2\pi}\int_{-\pi}^{\pi}{xdx}=0 \\
&a_n=\frac{1}{\pi}\int_{-\pi}^{\pi}{\left[ f\left( x \right) \cdot \cos nx \right] \cdot dx}=\frac{1}{\pi}\int_{-\pi}^{\pi}{x\cos nxdx}=0 \\
&b_n=\frac{1}{\pi}\int_{-\pi}^{\pi}{\left[ f\left( x \right) \cdot \sin nx \right] \cdot dx}=\frac{1}{\pi}\int_{-\pi}^{\pi}{x\sin nxdx}=\left( -1 \right) ^{n-1}\frac{2}{n}
\end{align*}
最终得$f\left( x \right) =x,x\in \left[ -\pi ,\pi \right) $的傅里叶级数:
\[
f\left( x \right) \backsim \sum_{n=1}^{\infty}{\left( -1 \right) ^{n-1}\frac{2}{n}\sin nx}
\]

~

\begin{example}
设函数
\[
f\left( x \right) =\begin{cases}
	0 & -\pi \leqslant x<0\\
	x & 0\leqslant x<\pi\\
\end{cases}
\]
求其$\left( -\pi ,\pi \right) $上的傅里叶级数。
\end{example}

解:
\begin{align*}
&a_0=\frac{1}{2\pi}\int_{-\pi}^{\pi}{\left[ f\left( x \right) \cdot 1 \right] \cdot dx}=\frac{1}{2\pi}\int_0^{\pi}{xdx}=\frac{\pi}{4} \\
&a_n=\frac{1}{\pi}\int_{-\pi}^{\pi}{\left[ f\left( x \right) \cdot \cos nx \right] \cdot dx}=\frac{1}{\pi}\int_0^{\pi}{x\cos nxdx}=\frac{1}{n^2\pi}\left[ \left( -1 \right) ^n-1 \right] \\
&b_n=\frac{1}{\pi}\int_{-\pi}^{\pi}{\left[ f\left( x \right) \cdot \sin nx \right] \cdot dx}=\frac{1}{\pi}\int_0^{\pi}{x\sin nxdx}=\left( -1 \right) ^{n-1}\frac{1}{n} \\
&f\left( x \right) \backsim \frac{\pi}{4}+\sum_{n=1}^{\infty}{\left[ \frac{\left( -1 \right) ^n-1}{n^2\pi}\cos nx+\frac{\left( -1 \right) ^{n-1}}{n}\sin nx \right]}
\end{align*}

~

\begin{example}
设周期函数$f\left( x \right) ,T=2\pi $在区间$\left[ -\pi ,\pi \right) $内的表达式为
\[
f\left( x \right) =\begin{cases}
	x^2 & -\pi \leqslant x<0\\
	0 & \leqslant x<\pi\\
\end{cases}
\]
求其在$\left[ -\pi ,2\pi \right] $上的傅里叶级数的和函数。
\end{example}

解:

可以直接用狄利克雷充分条件获得其傅里叶级数的和函数,但是需要分区间讨论。
对于题目要求$\left[ -\pi ,2\pi \right] $,由于$f\left( x \right) $周期性,先求$\left[ -\pi ,\pi \right) $内的和函数,再进行对$\left[ -\pi ,2\pi \right] $上的扩展即可。
在$\left[ -\pi ,\pi \right) $上求解和函数时,需要注意一个间断点$-\pi $,和函数有:
\[
S\left( x \right) =\begin{cases}
	\frac{1}{2}\pi ^2 & x=-\pi\\
	f\left( x \right) & x\in \left( -\pi ,\pi \right)\\
\end{cases}
\]
在$\left[ -\pi ,2\pi \right] $上求解和函数时,同样需要注意间断点$-\pi $,和函数有:
\[
S\left( x \right) =\begin{cases}
	\frac{1}{2}\pi ^2 & x=\pi\\
	f\left( x-2\pi \right) & x\in \left( \pi ,2\pi \right]\\
\end{cases}
\]
综上所述,$f\left( x \right) $在$\left[ -\pi ,2\pi \right] $上的傅里叶级数的和函数为:
\[
S\left( x \right) =\begin{cases}
	\frac{1}{2}\pi ^2 & x=\pm \pi\\
	x^2 & x\in \left( -\pi ,0 \right)\\
	0 & x\in \left[ 0,\pi \right)\\
	\left( x-2\pi \right) ^2 & x\in \left( \pi ,2\pi \right]\\
\end{cases}
\]






\newpage
\section{傅里叶级数的复数形式}

本节讨论傅里叶级数的另一种形式,复数形式。

本节要点:
\begin{itemize}
    \item 理解复数级数的概念;
    \item 了解通过欧拉公式的转换过程;
    \item 理解复数级数的数学意义。
\end{itemize}

%============================================================
\subsection{傅里叶复数级数的概念}

利用欧拉公式,可以联系傅里叶级数的三角形式和复数形式:
\begin{align*}
&e^{ix}=\cos x+i\sin x \\
&e^{-ix}=\cos x-i\sin x
\end{align*}
三角展开如下:
\begin{align*}
&f\left( x \right) =a_0+\sum_{n=1}^{\infty}{a_n\cos nx}+\sum_{n=1}^{\infty}{b_n\sin nx} \qquad x\in \left[ -\pi ,\pi \right] \\
&\begin{cases}
	a_0=\frac{1}{2\pi}\int_{-\pi}^{\pi}{\left[ f\left( x \right) \cdot 1 \right] \cdot dx}\\
	a_n=\frac{1}{\pi}\int_{-\pi}^{\pi}{\left[ f\left( x \right) \cdot \cos nx \right] \cdot dx}\\
	b_n=\frac{1}{\pi}\int_{-\pi}^{\pi}{\left[ f\left( x \right) \cdot \sin nx \right] \cdot dx}\\
\end{cases} \quad n=1,2,\cdots
\end{align*}
将傅里叶展开里的三角函数化成指数函数:
\begin{align*}
f\left( x \right) &=a_0+\sum_{n=1}^{+\infty}{\left( a_n\cos nx+b_n\sin nx \right)} \qquad x\in \left[ -\pi ,\pi \right] \\
&=a_0+\sum_{n=1}^{+\infty}{\left[ a_n\frac{e^{inx}+e^{-inx}}{2}+b_n\frac{-i\left( e^{inx}-e^{-inx} \right)}{2} \right]} \\
&=a_0+\sum_{n=1}^{+\infty}{\left( \frac{a_n-ib_n}{2}e^{inx}+\frac{a_n+ib_n}{2}e^{-inx} \right)}
\end{align*}
记:
\begin{align*}
&\left\{ \begin{aligned}
	c_0&=a_0=\frac{1}{2\pi}\int_{-\pi}^{\pi}{\left[ f\left( x \right) \cdot 1 \right] \cdot dx}\\
	c_n&=\frac{a_n-ib_n}{2}=\frac{1}{2\pi}\int_{-\pi}^{\pi}{f\left( x \right) \cdot \left( \cos nx-i\sin nx \right) \cdot dx}\\
	&=\frac{1}{2\pi}\int_{-\pi}^{\pi}{f\left( x \right) \cdot e^{-inx}\cdot dx}\\
	c_{-n}&=\frac{a_n+ib_n}{2}=\frac{1}{2\pi}\int_{-\pi}^{\pi}{f\left( x \right) \cdot \left( \cos nx+i\sin nx \right) \cdot dx}\\
	&=\frac{1}{2\pi}\int_{-\pi}^{\pi}{f\left( x \right) \cdot e^{inx}\cdot dx}\\
\end{aligned} \right. \\
&n=1,2,\cdots
\end{align*}
则可以写成:
\begin{align*}
&f\left( x \right) =c_0+\sum_{n=1}^{+\infty}{\left( c_ne^{inx}+c_{-n}e^{-inx} \right)}=\sum_{n=-\infty}^{+\infty}{c_ne^{inx}} \quad x\in \left[ -\pi ,\pi \right] \\
&c_n=\frac{1}{2\pi}\int_{-\pi}^{\pi}{f\left( x \right) \cdot e^{-inx}\cdot dx} \qquad \qquad \qquad \qquad \qquad \quad n=1,2,\cdots
\end{align*}

\begin{definition}[傅里叶复数级数]
若周期函数$f\left( x \right) $满足狄利克雷收敛条件,则函数在$\left[ -\pi ,\pi \right] $上可以展开为复指数级数,且该级数收敛于$f\left( x \right) $,即:
\begin{align*}
&f\left( x \right) =\sum_{n=-\infty}^{+\infty}{c_ne^{inx}} \qquad \qquad \quad x\in \left[ -\pi ,\pi \right] \\
&c_n=\frac{1}{2\pi}\int_{-\pi}^{\pi}{f\left( x \right) \cdot e^{-inx}\cdot dx} \quad n=0,\pm 1,\pm 2,\cdots
\end{align*}
该展开式称为{\bf $f\left( x \right) $的傅里叶级数的复数形式},$c_n$称为{\bf 傅里叶系数的复数形式}。
\end{definition}

%============================================================
\subsection{傅里叶复数级数的向量空间意义}

从线性代数的角度看,傅里叶级数的复数形式的本质依然是空间变换,将原本在实数$x$空间的函数映射到虚数$c_n$空间。
\begin{align*}
&c_n=\frac{1}{2\pi}\int_{-\pi}^{\pi}{f\left( x \right) e^{-inx}\cdot dx} \\
f\left( x \right) \gets &------------\rightarrow c_n
\end{align*}
其中狄利克雷充分条件保证了哪些函数可以变换。
寻找变换的过程就是求解傅里叶系数$c_n$,在线性代数角度看,就是求解一个非常“稠密的”线性方程组。

%============================================================
\subsection{任意周期的傅里叶复数级数}

同样,若考虑任意周期$T$,则有:
\begin{align*}
&f\left( x \right) =\sum_{n=-\infty}^{+\infty}{c_ne^{i\frac{2n\pi}{T}x}} \qquad \qquad \quad x\in \left[ -\frac{T}{2},\frac{T}{2} \right] \\
&c_n=\frac{1}{T}\int_{-T/2}^{T/2}{f\left( x \right) \cdot e^{-i\frac{2n\pi}{T}x}\cdot dx} \quad n=0,\pm 1,\pm 2,\cdots
\end{align*}






\newpage
\section{三角形式和复数形式的对比}

本节将两种形式的傅里叶级数并在一起,以便对比。
设周期为$2\pi $的函数$f\left( x \right) ,x\in \left[ -\pi ,\pi \right] $。

傅里叶级数的三角形式:
\begin{align*}
&f\left( x \right) =\frac{a_0}{2}+\sum_{n=1}^{\infty}{\left( a_n\cos nx+b_n\sin nx \right)} \\
&\begin{cases}
	a_0=\frac{1}{\pi}\int_{-\pi}^{\pi}{f\left( x \right) dx}\\
	a_n=\frac{1}{\pi}\int_{-\pi}^{\pi}{f\left( x \right) \cos nxdx}\\
	b_n=\frac{1}{\pi}\int_{-\pi}^{\pi}{f\left( x \right) \sin nxdx}\\
\end{cases} \quad n=1,2,\cdots
\end{align*}

傅里叶级数的复数形式:
\begin{align*}
&f\left( x \right) =\sum_{n=-\infty}^{+\infty}{c_ne^{inx}} \\
&c_n=\frac{1}{2\pi}\int_{-\pi}^{\pi}{f\left( x \right) e^{-inx}dx} \quad n=0,\pm 1,\pm 2,\cdots
\end{align*}






\newpage
\section{复指函数系}

之前讲到的傅里叶复数级数的推导过程是一般微积分教材中使用的方法。
本节介绍从正交函数系开始推导。

%============================================================
\subsection{正交的复指数函数系}

构建基于复指数的正交函数系是一个不直观的过程,单个$e^{inx}$的基长度总是为0。
参考欧拉公式的推导过程,采用如下复指数函数系:
\begin{align*}
&1 \\
&\left( e^{ix}+e^{-ix} \right) ,\left( e^{i2x}+e^{-i2x} \right) ,\cdots ,\left( e^{inx}+e^{-inx} \right) ,\cdots \\
&i\left( e^{ix}-e^{-ix} \right) ,i\left( e^{i2x}-e^{-i2x} \right) ,\cdots ,i\left( e^{inx}-e^{-inx} \right) ,\cdots
\end{align*}

\begin{tcolorbox}
寻找过程颇为繁琐,反复试了好几组。
单个$e^{inx}$由于长度为0是不行的,单纯的共轭对$e^{inx}+e^{-inx}$张不成$C\left[ -\pi ,\pi \right] $。
\end{tcolorbox}

验证正交性:
\begin{align*}
&\left< 1,\left( e^{inx}+e^{-inx} \right) \right> =\int_{-\pi}^{\pi}{\left( e^{inx}+e^{-inx} \right) \cdot dx}=0 \\
&\left< 1,i\left( e^{inx}-e^{-inx} \right) \right> =\int_{-\pi}^{\pi}{i\left( e^{inx}-e^{-inx} \right) \cdot dx}=0 \\
&\left< \left( e^{inx}+e^{-inx} \right) ,\left( e^{imx}+e^{-imx} \right) \right> = \\
& \qquad \qquad \qquad \int_{-\pi}^{\pi}{\left( e^{inx}+e^{-inx} \right) \cdot \left( e^{imx}+e^{-imx} \right) \cdot dx}=0 \\
&\left< i\left( e^{inx}-e^{-inx} \right) ,i\left( e^{imx}-e^{-imx} \right) \right> = \\
& \qquad \qquad \qquad \int_{-\pi}^{\pi}{i\left( e^{inx}-e^{-inx} \right) \cdot i\left( e^{imx}-e^{-imx} \right) \cdot dx}=0 \\
&\left< \left( e^{inx}+e^{-inx} \right) ,i\left( e^{imx}-e^{-imx} \right) \right> = \\
& \qquad \qquad \qquad \int_{-\pi}^{\pi}{\left( e^{inx}+e^{-inx} \right) \cdot i\left( e^{imx}-e^{-imx} \right) \cdot dx}=0
\end{align*}
这里注意,中间两个内积必须$n\ne m$,最后这个内积没有要求,即便$n\ne m$也是为0。

计算基的长度:
\begin{align*}
\left\| 1 \right\| ^2=\left< 1,1 \right> &=\int_{-\pi}^{\pi}{1\cdot dx}=2\pi \\
\left\| \left( e^{inx}+e^{-inx} \right) \right\| ^2&=\left< \left( e^{inx}+e^{-inx} \right) ,\left( e^{inx}+e^{-inx} \right) \right> \\
&=\int_{-\pi}^{\pi}{\left( e^{i2nx}+e^{-i2nx}+2 \right) \cdot dx}=4\pi \\
\left\| i\left( e^{inx}-e^{-inx} \right) \right\| ^2&=\left< i\left( e^{inx}-e^{-inx} \right) ,i\left( e^{inx}-e^{-inx} \right) \right> \\
&=\int_{-\pi}^{\pi}{i^2\left( e^{i2nx}+e^{-i2nx}-2 \right) \cdot dx}=4\pi
\end{align*}

%============================================================
\subsection{傅里叶复数级数}

于是,满足狄利克雷收敛条件的周期函数$f\left( x \right) $在$\left[ -\pi ,\pi \right] $的展开式为:
\begin{align*}
&f\left( x \right) =a_0+\sum_{n=1}^{\infty}{b_n\left( e^{inx}+e^{-inx} \right)}+\sum_{n=1}^{\infty}{c_ni\left( e^{inx}-e^{-inx} \right)} \\
&a_0=\frac{1}{2\pi}\int_{-\pi}^{\pi}{\left[ f\left( x \right) \cdot 1 \right] \cdot dx} \\
&b_n=\frac{1}{4\pi}\int_{-\pi}^{\pi}{\left[ f\left( x \right) \cdot \left( e^{inx}+e^{-inx} \right) \right] \cdot dx}=\frac{1}{2\pi}\int_{-\pi}^{\pi}{\left[ f\left( x \right) \cdot \cos nx \right] \cdot dx} \\
&c_n=\frac{1}{4\pi}\int_{-\pi}^{\pi}{\left[ f\left( x \right) \cdot i\left( e^{inx}-e^{-inx} \right) \right] \cdot dx}=-\frac{1}{2\pi}\int_{-\pi}^{\pi}{\left[ f\left( x \right) \cdot \sin nx \right] \cdot dx}
\end{align*}

傅里叶系数的计算过程略,基本就是使用正交性和长度。

%============================================================
\subsection{化简}

上述展开式并不是我们想要的,需要化简。

先考察:
\[
\sum_{n=1}^{\infty}{b_n\left( e^{inx}+e^{-inx} \right)}=\sum_{n=1}^{\infty}{b_ne^{inx}}+\sum_{n=1}^{\infty}{b_ne^{-inx}}
\]
由于
\[
b_{-n}=\frac{1}{2\pi}\int_{-\pi}^{\pi}{\left[ f\left( x \right) \cdot \cos \left( -n \right) x \right] \cdot dx}=\frac{1}{2\pi}\int_{-\pi}^{\pi}{\left[ f\left( x \right) \cdot \cos nx \right] \cdot dx}=b_n
\]
对于$\sum_{n=1}^{\infty}{b_ne^{-inx}}$,我们令$m=-n$后得到:
\[
\sum_{n=1}^{\infty}{b_ne^{-inx}}=\sum_{m=-1}^{-\infty}{b_{-m}e^{imx}}=\sum_{m=-1}^{-\infty}{b_me^{imx}}
\]
于是可以将两项和并得到:
\begin{align*}
&\sum_{n=1}^{\infty}{b_n\left( e^{inx}+e^{-inx} \right)}=\sum_{n=1}^{+\infty}{b_ne^{inx}}+\sum_{n=-1}^{-\infty}{b_ne^{inx}}=\sum_{n=-\infty}^{+\infty}{b_ne^{inx}} \\
&b_n=\frac{1}{2\pi}\int_{-\pi}^{\pi}{\left[ f\left( x \right) \cdot \cos nx \right] \cdot dx}
\end{align*}

再考察:
\[
\sum_{n=1}^{\infty}{c_ni\left( e^{inx}-e^{-inx} \right)}=\sum_{n=1}^{\infty}{c_nie^{inx}}+\sum_{n=1}^{\infty}{-c_nie^{-inx}}
\]
同样
\[
c_{-n}=-\frac{1}{2\pi}\int_{-\pi}^{\pi}{\left[ f\left( x \right) \cdot \sin \left( -n \right) x \right] \cdot dx}=\frac{1}{2\pi}\int_{-\pi}^{\pi}{\left[ f\left( x \right) \cdot \sin nx \right] \cdot dx}=-c_n
\]
对于$\sum_{n=1}^{\infty}{-c_nie^{-inx}}$,我们令$m=-n$后得到:
\[
\sum_{n=1}^{\infty}{-c_nie^{-inx}}=\sum_{m=-1}^{-\infty}{c_mie^{imx}}
\]
于是可以将两项和并得到:
\begin{align*}
&\sum_{n=1}^{\infty}{c_ni\left( e^{inx}-e^{-inx} \right)}=\sum_{n=+1}^{+\infty}{c_nie^{inx}}+\sum_{n=-1}^{-\infty}{c_nie^{inx}}=\sum_{n=-\infty}^{+\infty}{c_nie^{inx}} \\
&c_n=-\frac{1}{2\pi}\int_{-\pi}^{\pi}{\left[ f\left( x \right) \cdot \sin nx \right] \cdot dx}
\end{align*}

此时傅里叶级数展开式为:
\begin{align*}
&f\left( x \right) =a_0+\sum_{n=-\infty}^{+\infty}{b_ne^{inx}}+\sum_{n=-\infty}^{+\infty}{c_nie^{inx}} \\
&a_0=\frac{1}{2\pi}\int_{-\pi}^{\pi}{\left[ f\left( x \right) \cdot 1 \right] \cdot dx} \\
&b_n=\frac{1}{2\pi}\int_{-\pi}^{\pi}{\left[ f\left( x \right) \cdot \cos nx \right] \cdot dx} \\
&c_n=-\frac{1}{2\pi}\int_{-\pi}^{\pi}{\left[ f\left( x \right) \cdot \sin nx \right] \cdot dx}
\end{align*}

%============================================================
\subsection{进一步化简}

考察展开式后两项,是可以合并的,如下:
\begin{align*}
&\sum_{n=-\infty}^{+\infty}{b_ne^{inx}}+\sum_{n=-\infty}^{+\infty}{c_nie^{inx}}=\sum_{n=-\infty}^{+\infty}{\left( b_n+ic_n \right) e^{inx}} \\
&b_n+ic_n=\frac{1}{2\pi}\int_{-\pi}^{\pi}{f\left( x \right) \cdot \left[ \cos nx-i\sin nx \right] \cdot dx}=\frac{1}{2\pi}\int_{-\pi}^{\pi}{f\left( x \right) \cdot e^{-inx}\cdot dx}
\end{align*}
于是令$c_n=b_n+ic_n$得到:
\begin{align*}
&f\left( x \right) =a_0+\sum_{n=-\infty}^{+\infty}{c_ne^{inx}} \\
&a_0=\frac{1}{2\pi}\int_{-\pi}^{\pi}{\left[ f\left( x \right) \cdot 1 \right] \cdot dx} \\
&c_n=\frac{1}{2\pi}\int_{-\pi}^{\pi}{f\left( x \right) \cdot e^{-inx}\cdot dx}
\end{align*}
至此,还剩$a_0$。
由于$n=0$时,$e^{-inx}=1$,于是$a_0$也可以统一进来:
\[
c_0=a_0=\frac{1}{2\pi}\int_{-\pi}^{\pi}{\left[ f\left( x \right) \cdot e^{-inx} \right] \cdot dx}
\]

得到最终傅里叶级数的复指数展开式:
\begin{align*}
&f\left( x \right) =\sum_{n=-\infty}^{+\infty}{c_ne^{inx}} \\
&c_n=\frac{1}{2\pi}\int_{-\pi}^{\pi}{f\left( x \right) \cdot e^{-inx}\cdot dx}
\end{align*}






\newpage
\section{傅里叶积分和傅里叶变换}

本节继续考察任意周期函数。
如果周期为无穷大,就相当于任意函数。

本节要点:
\begin{itemize}
    \item 理解傅里叶积分的推导过程;
    \item 理解傅里叶变换。
\end{itemize}

\begin{definition}[傅里叶积分]
将非周期函数$f\left( x \right) $依照$T$的周期函数在$\left[ -T/2,T/2 \right] $内的傅里叶展开:
\begin{align*}
f\left( x \right) &=\sum_{n=-\infty}^{+\infty}{c_ne^{i\frac{2n\pi}{T}x}}=\sum_{n=-\infty}^{+\infty}{\left[ \frac{1}{T}\int_{-\frac{T}{2}}^{\frac{T}{2}}{f\left( \xi \right) e^{-i\frac{2n\pi}{T}\xi}\cdot d\xi} \right] e^{i\frac{2n\pi}{T}x}} \\
&=\sum_{n=-\infty}^{+\infty}{\left[ \frac{1}{T}\int_{-\frac{T}{2}}^{\frac{T}{2}}{f\left( \xi \right) e^{-i\omega \xi}\cdot d\xi} \right] e^{i\omega x}} \\
&=\sum_{n=-\infty}^{+\infty}{\left[ \frac{\Delta \omega}{2\pi}\int_{-\frac{T}{2}}^{\frac{T}{2}}{f\left( \xi \right) e^{-i\omega \xi}\cdot d\xi} \right] e^{i\omega x}} \\
&=\frac{1}{2\pi}\sum_{n=-\infty}^{+\infty}{\left[ \int_{-\frac{T}{2}}^{\frac{T}{2}}{f\left( \xi \right) e^{-i\omega \xi}\cdot d\xi} \right] e^{i\omega x}\Delta \omega}
\end{align*}
其中$\omega =2n\pi /T,\Delta \omega =2\pi /T$,如果$T\rightarrow \infty $时,极限
\begin{align*}
f\left( x \right) &=\underset{T\rightarrow +\infty}{\lim}\sum_{n=-\infty}^{+\infty}{c_ne^{i\frac{2n\pi}{T}x}} \\
&=\underset{T\rightarrow +\infty}{\lim}\frac{1}{2\pi}\sum_{n=-\infty}^{+\infty}{\left[ \int_{-\frac{T}{2}}^{\frac{T}{2}}{f\left( \xi \right) e^{-i\omega \xi}\cdot d\xi} \right] e^{i\omega x}\Delta \omega}
\end{align*}
存在且唯一,则称该极限为{\bf $f\left( x \right) $的傅里叶积分},此时写为:
\[
f\left( x \right) =\frac{1}{2\pi}\int_{-\infty}^{+\infty}{\left[ \int_{-\infty}^{+\infty}{f\left( x \right) e^{-i\omega x}dx} \right] e^{i\omega x}d\omega}
\]
同时,我们称$\int_{-\infty}^{+\infty}{f\left( x \right) e^{-i\omega x}dx}$部分为{\bf $f\left( x \right) $的傅里叶变换},记为$F\left( \omega \right) $,即:
\[
F\left( \omega \right) =\int_{-\infty}^{+\infty}{f\left( x \right) e^{-i\omega x}dx}
\]
相应地,称:
\[
f\left( x \right) =\frac{1}{2\pi}\int_{-\infty}^{+\infty}{F\left( \omega \right) e^{i\omega x}d\omega}
\]
为{\bf 傅里叶逆变换}。
\end{definition}

\begin{tcolorbox}
微积分的角度,傅里叶积分是一个加权和,这点和定积分一样。
\end{tcolorbox}

对比傅里叶级数和傅里叶变换:
\begin{align*}
&\begin{cases}
	f\left( x \right) =\sum_{n=-\infty}^{+\infty}{c_ne^{inx}} \qquad x\in \left[ -\pi ,\pi \right]\\
	c_n=\frac{1}{2\pi}\int_{-\pi}^{\pi}{f\left( x \right) e^{-inx}dx} \quad n=0,\pm 1,\pm 2,...\\
\end{cases} \\
&\begin{cases}
	F\left( \omega \right) =\int_{-\infty}^{+\infty}{f\left( x \right) e^{-i\omega x}dx}\\
	f\left( x \right) =\frac{1}{2\pi}\int_{-\infty}^{+\infty}{F\left( \omega \right) e^{i\omega x}d\omega}\\
\end{cases}
\end{align*}
根据$f\left( x \right) =\frac{1}{2\pi}\int_{-\infty}^{+\infty}{\left[ \int_{-\infty}^{+\infty}{f\left( x \right) e^{-i\omega x}dx} \right] e^{i\omega x}d\omega}$,其实这个$\frac{1}{2\pi}$放在哪里都可以,只是习惯上放在逆变换中。
$\frac{1}{2\pi}$来自于我们使用的函数基并不是规范的,即长度并不是1。

\begin{theorem}
若函数$f\left( x \right) $在整个实数域满足:
\begin{itemize}
    \item 任一有限区域上满足狄利克雷收敛条件;
    \item 在$\mathbb{R} $上绝对可积,即$\int_{-\infty}^{+\infty}{\left| f\left( x \right) \right|dx}$收敛;
\end{itemize}
则有:
\[
f\left( x \right) =\frac{1}{2\pi}\int_{-\infty}^{+\infty}{\left[ \int_{-\infty}^{+\infty}{f\left( x \right) e^{-i\omega x}dx} \right] e^{i\omega x}d\omega}
\]
对$f\left( x \right) $所有的连续点成立,在间断点$x_0$处,可以用$\frac{1}{2}\left[ f\left( {x_0}^+ \right) +f\left( {x_0}^- \right) \right] $代替。
\end{theorem}





\newpage
\section{本章小结}

综合上一章,我们一共讨论了两种函数级数,幂级数和傅里叶级数。

整个微积分可以说是定义和分析什么是“好函数”。
于是,对于“坏函数”,我们希望能用好函数去描述。
这就是级数的基本思想。

基本初等函数都是好的函数,有三大类非常好:1)幂函数;2)三角函数;3)指数函数。
它们都可以用来构建“好函数级数”以代替坏函数。
但由于欧拉公式,所以指数函数构成的级数其实就是三角函数构成的级数。
所以微积分中,函数级数只有幂级数和傅里叶级数两种。









