\section{傅里叶积分和傅里叶变换}

本节继续考察任意周期函数。
如果周期为无穷大,就相当于任意函数。

本节要点:
\begin{itemize}
    \item 理解傅里叶积分的推导过程;
    \item 理解傅里叶变换。
\end{itemize}

\begin{definition}[傅里叶积分]
将非周期函数$f\left( x \right) $依照$T$的周期函数在$\left[ -T/2,T/2 \right] $内的傅里叶展开:
\begin{align*}
f\left( x \right) &=\sum_{n=-\infty}^{+\infty}{c_ne^{i\frac{2n\pi}{T}x}}=\sum_{n=-\infty}^{+\infty}{\left[ \frac{1}{T}\int_{-\frac{T}{2}}^{\frac{T}{2}}{f\left( \xi \right) e^{-i\frac{2n\pi}{T}\xi}\cdot d\xi} \right] e^{i\frac{2n\pi}{T}x}} \\
&=\sum_{n=-\infty}^{+\infty}{\left[ \frac{1}{T}\int_{-\frac{T}{2}}^{\frac{T}{2}}{f\left( \xi \right) e^{-i\omega \xi}\cdot d\xi} \right] e^{i\omega x}} \\
&=\sum_{n=-\infty}^{+\infty}{\left[ \frac{\Delta \omega}{2\pi}\int_{-\frac{T}{2}}^{\frac{T}{2}}{f\left( \xi \right) e^{-i\omega \xi}\cdot d\xi} \right] e^{i\omega x}} \\
&=\frac{1}{2\pi}\sum_{n=-\infty}^{+\infty}{\left[ \int_{-\frac{T}{2}}^{\frac{T}{2}}{f\left( \xi \right) e^{-i\omega \xi}\cdot d\xi} \right] e^{i\omega x}\Delta \omega}
\end{align*}
其中$\omega =2n\pi /T,\Delta \omega =2\pi /T$,如果$T\rightarrow \infty $时,极限
\begin{align*}
f\left( x \right) &=\underset{T\rightarrow +\infty}{\lim}\sum_{n=-\infty}^{+\infty}{c_ne^{i\frac{2n\pi}{T}x}} \\
&=\underset{T\rightarrow +\infty}{\lim}\frac{1}{2\pi}\sum_{n=-\infty}^{+\infty}{\left[ \int_{-\frac{T}{2}}^{\frac{T}{2}}{f\left( \xi \right) e^{-i\omega \xi}\cdot d\xi} \right] e^{i\omega x}\Delta \omega}
\end{align*}
存在且唯一,则称该极限为{\bf $f\left( x \right) $的傅里叶积分},此时写为:
\[
f\left( x \right) =\frac{1}{2\pi}\int_{-\infty}^{+\infty}{\left[ \int_{-\infty}^{+\infty}{f\left( x \right) e^{-i\omega x}dx} \right] e^{i\omega x}d\omega}
\]
同时,我们称$\int_{-\infty}^{+\infty}{f\left( x \right) e^{-i\omega x}dx}$部分为{\bf $f\left( x \right) $的傅里叶变换},记为$F\left( \omega \right) $,即:
\[
F\left( \omega \right) =\int_{-\infty}^{+\infty}{f\left( x \right) e^{-i\omega x}dx}
\]
相应地,称:
\[
f\left( x \right) =\frac{1}{2\pi}\int_{-\infty}^{+\infty}{F\left( \omega \right) e^{i\omega x}d\omega}
\]
为{\bf 傅里叶逆变换}。
\end{definition}

\begin{tcolorbox}
微积分的角度,傅里叶积分是一个加权和,这点和定积分一样。
\end{tcolorbox}

对比傅里叶级数和傅里叶变换:
\begin{align*}
&\begin{cases}
	f\left( x \right) =\sum_{n=-\infty}^{+\infty}{c_ne^{inx}} \qquad x\in \left[ -\pi ,\pi \right]\\
	c_n=\frac{1}{2\pi}\int_{-\pi}^{\pi}{f\left( x \right) e^{-inx}dx} \quad n=0,\pm 1,\pm 2,...\\
\end{cases} \\
&\begin{cases}
	F\left( \omega \right) =\int_{-\infty}^{+\infty}{f\left( x \right) e^{-i\omega x}dx}\\
	f\left( x \right) =\frac{1}{2\pi}\int_{-\infty}^{+\infty}{F\left( \omega \right) e^{i\omega x}d\omega}\\
\end{cases}
\end{align*}
根据$f\left( x \right) =\frac{1}{2\pi}\int_{-\infty}^{+\infty}{\left[ \int_{-\infty}^{+\infty}{f\left( x \right) e^{-i\omega x}dx} \right] e^{i\omega x}d\omega}$,其实这个$\frac{1}{2\pi}$放在哪里都可以,只是习惯上放在逆变换中。
$\frac{1}{2\pi}$来自于我们使用的函数基并不是规范的,即长度并不是1。

\begin{theorem}
若函数$f\left( x \right) $在整个实数域满足:
\begin{itemize}
    \item 任一有限区域上满足狄利克雷收敛条件;
    \item 在$\mathbb{R} $上绝对可积,即$\int_{-\infty}^{+\infty}{\left| f\left( x \right) \right|dx}$收敛;
\end{itemize}
则有:
\[
f\left( x \right) =\frac{1}{2\pi}\int_{-\infty}^{+\infty}{\left[ \int_{-\infty}^{+\infty}{f\left( x \right) e^{-i\omega x}dx} \right] e^{i\omega x}d\omega}
\]
对$f\left( x \right) $所有的连续点成立,在间断点$x_0$处,可以用$\frac{1}{2}\left[ f\left( {x_0}^+ \right) +f\left( {x_0}^- \right) \right] $代替。
\end{theorem}



