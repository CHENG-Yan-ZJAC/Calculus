\section{常微分方程}

函数描述量和量之间的关系,但在很多实际问题中这个函数关系无法直接找出,而是根据条件建立起量及其变化率之间的关系式,这就是微分方程。
通过求解微分方程得到函数关系。

本节要点:
\begin{itemize}
    \item 理解常微分方程的概念;
    \item 理解常微分方程的物理意义;
    \item 掌握一阶微分方程的求解思路和方法。
\end{itemize}

%============================================================
\subsection{常微分方程的概念}

\begin{definition}[微分方程]
我们将含有未知函数及其导数(或微分)的方程称为{\bf 微分方程}。
未知函数是一元函数的微分方程称为{\bf 常微分方程}。
未知函数的最高阶导数的阶数称为微分方程的{\bf 阶}。
如果微分方程是未知函数及其各阶导数的一次有理整式,则称之为{\bf n阶线性常微分方程}(简称{\bf n阶线性方程}),记作:
\[
F\left( x,y,y',y'',...,y^{\left( n \right)} \right) =0
\]
如果函数$y=f\left( x \right) $代入微分方程后等式成立,则称$y=f\left( x \right) $为该微分方程的{\bf 解}。
如果函数含有常数,且常数的个数和微分方程的阶数相同,这样的函数称为{\bf 通解},否则称为{\bf 特解}。
\end{definition}

“阶”和“线性”都是对未知函数及其导数而言,不是对自变量的要求。
“线性”只是未知函数及其各阶导数的一次有理,如$\sin y$、$\left( y' \right) ^2$、$yy'$、$e^{y''}$都不是线性。

%============================================================
\subsection{求解思路}

求解一阶微分方程的大体思路是将方程变成分离形式,即:
\[
f\left( y \right) dy=g\left( x \right) dx
\]
然后两边分别积分:
\[
\int{f\left( y \right) dy}=\int{g\left( x \right) dx}
\]
再分别求不定积分,最后得到:
\[
F\left( y \right) =G\left( x \right) +C
\]

%============================================================
\subsection{微分方程的物理意义}

通常一个物理量和其变化率有关,所以可以用一个微分方程描述,那么微分方程的通解就是这个物理量的数学表达式。

可以毫不夸张地说,宇宙里所有系统都可以用微分方程描述,这是有一个深层次原因的。
对于一个输入输出系统,输入的量并不是直接作用产生输出,往往会在系统中振荡几下再输出。
在系统论的角度,当刻的输出是当刻及之前的输入和之前的输出一起作用的结果,这种作用体现在数学模型上就是微分方程。

\begin{tcolorbox}
微分方程可以和《信号与系统》课程相互学习。
\end{tcolorbox}




