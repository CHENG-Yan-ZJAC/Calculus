\section{一阶线性常微分方程}

全称“一阶线性常微分方程”:
\begin{itemize}
    \item 一阶:未知函数的最高阶导数为一阶导数;
    \item 线性:微分方程是未知函数及其导数的一次有理整式;
    \item 常:未知函数是一元函数。
\end{itemize}
由于本章所述均为一元函数,所以省去“常微分”,简称“一阶线性方程”。

一阶线性方程解法的总体思路是分离变量,通过把$dx$和$dy$分离到方程的两边,再求解积分化掉微分项$dx$和$dy$。
一阶线性方程的通解可以描述为:
\[
\text{一阶线性方程的通解} = \text{齐次方程的通解} + \text{非齐次方程的一个特解}
\]
为了得到这个方法,本节先从分离变量开始,讲到齐次方程,最后推导出一阶线性方程的求解公式。

本节要点:
\begin{itemize}
    \item 熟悉一阶线性方程的形式;
    \item 推导一阶线性方程的通解;
    \item 体会一阶线性方程的物理意义。
\end{itemize}

%============================================================
\subsection{一阶齐次线性方程}

\begin{definition}[一阶齐次线性方程]
形如
\[
\frac{dy}{dx}+P\left( x \right) y=0
\]
的微分方程称为{\bf 一阶齐次线性方程},其中$y=f\left( x \right) $。
\end{definition}

\begin{tcolorbox}
“齐次”指的是没有和$y$无关的项。
\end{tcolorbox}

求解过程如下:
\begin{enumerate}
    \item 分离变量:
    \[
    \frac{1}{y}dy=-P\left( x \right) dx
    \]
    \item 两边分别求不定积分:
    \[
    \ln \left| y \right|=-\int{P\left( x \right) dx}+C
    \]
    \item 两边取$e$的指数,得到通解:
    \[
    y=Ce^{-\int{P\left( x \right) dx}}
    \]
\end{enumerate}

特别地,当$P\left( x \right) \equiv P\ne 0$时,方程中无自变量$x$,方程变成:
\[
\frac{dy}{dx}+Py=0
\]
它是一阶齐次线性方程的特殊形式——{\bf 一阶常系数齐次线性方程},即$y$的系数不是一个关于$x$的函数,而是一个固定值,用通解公式易得通解:
\[
y=Ce^{-Px}
\]
是一个很典型的指数函数,讨论$x\in \left( 0,+\infty \right) $的时候:
\begin{itemize}
    \item 当$P>0$,$y$单调递减并从$C$趋向0,相当于引入一个负反馈;
    \item 当$P<0$,$y$单调递增并从$C$趋向$+\infty $,相当于引入一个正反馈;
    \item 当$P=0$,方程变成$\frac{dy}{dx}=0$,解为$y=C$。
\end{itemize}

\begin{tcolorbox}
在物理角度看方程$\frac{dy}{dx}+Py=0$的解,如果一个系统没有输入,仅靠存量作用产生输出(《信号与系统》理论中称为零输入响应),往往会以指数形式衰减,比如放射性物质衰变公式。
\end{tcolorbox}

%============================================================
\subsection{一阶非齐次线性方程}

\begin{definition}[一阶非齐次线性方程]
形如
\[
\frac{dy}{dx}+P\left( x \right) y=Q\left( x \right)
\]
的微分方程称为{\bf 一阶非齐次线性方程},其中$y=f\left( x \right) $。
\end{definition}

\begin{tcolorbox}
“非齐次”指的是有了一个和$y$无关的项$Q\left( x \right) $。
\end{tcolorbox}

求解过程如下:
\begin{enumerate}
    \item 求得对应的齐次方程的通解:
    \[
    y=Ce^{-\int{P\left( x \right) dx}}
    \]
    \item 将常数$C$变为待定函数$C=\left( x \right) $,代入上式:
    \[
    y=C\left( x \right) e^{-\int{P\left( x \right) dx}}
    \]
    \item 将该函数代入微分方程求得:
    \[
    C\left( x \right) =\int{Q\left( x \right) e^{\int{P\left( x \right) dx}}dx}+C
    \]
    \item 得到通解:
    \[
    y=e^{-\int{P\left( x \right) dx}}\cdot \left[ \int{Q\left( x \right) e^{\int{P\left( x \right) dx}}dx}+C \right]
    \]
\end{enumerate}

特别地,当$P\left( x \right) \equiv P,Q\left( x \right) \equiv Q$时,方程中无自变量$x$,方程变成:
\[
\frac{dy}{dx}+Py=Q
\]
它是一阶非齐次线性方程的特殊形式——{\bf 一阶常系数非齐次线性方程},即$y$的系数和无关项是一个固定值,用通解公式易得通解:
\[
y=Ce^{-Px}+\frac{Q}{P}
\]
在典型的指数函数的基础上上抬了$Q/P$,讨论$x\in \left( 0,+\infty \right) $的时候:
\begin{itemize}
    \item 当$P>0$,$y$单调递减并从$C+Q/P$趋向$Q/P$,相当于引入一个负反馈;
    \item 当$P<0$,$y$单调递增并从$C+Q/P$趋向$+\infty $,相当于引入一个正反馈。
\end{itemize}
和一阶常系数齐次线性方程对比看,整个系统有一个“固定能量”。

一阶线性常微分方程由于只包含一阶导数,函数的变化率单一,所以不会像二阶一样发生振荡。

%============================================================
\subsection{一阶线性方程及通解形式}

一阶齐次线性方程:
\begin{align*}
&\text{方程:} \quad \frac{dy}{dx}+P\left( x \right) y=0 \\
&\text{通解:} \quad y=Ce^{-\int{P\left( x \right) dx}}
\end{align*}

一阶常系数齐次线性方程:
\begin{align*}
&\text{方程:} \quad \frac{dy}{dx}+Py=0 \\
&\text{通解:} \quad y=Ce^{-Px}
\end{align*}

一阶非齐次线性方程:
\begin{align*}
&\text{方程:} \quad \frac{dy}{dx}+P\left( x \right) y=Q\left( x \right) \\
&\text{通解:} \quad y=e^{-\int{P\left( x \right) dx}}\cdot \left[ \int{Q\left( x \right) e^{\int{P\left( x \right) dx}}dx}+C \right]
\end{align*}

一阶常系数非齐次线性方程:
\begin{align*}
&\text{方程:} \quad \frac{dy}{dx}+Py=Q \\
&\text{通解:} \quad y=Ce^{-Px}+\frac{Q}{P}
\end{align*}




