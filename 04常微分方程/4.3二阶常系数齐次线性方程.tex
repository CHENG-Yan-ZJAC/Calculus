\section{二阶常系数齐次线性方程}

对于二阶线性方程,本节只讨论二阶常系数齐次线性方程。

本节要点:
\begin{itemize}
    \item 熟悉二阶线性方程的形式;
    \item 推导二阶线性方程的通解;
    \item 体会二阶线性方程的物理意义。
\end{itemize}

%============================================================
\subsection{二阶常系数齐次线性方程}

\begin{definition}[二阶常系数齐次线性方程]
形如
\[
\frac{d^2y}{dx^2}+P\frac{dy}{dx}+Qy=0
\]
的微分方程称为{\bf 二阶常系数齐次线性方程},其中$y=f\left( x \right) $。
\end{definition}

该微分方程的求解,全靠“凑”,求解过程如下:
\begin{enumerate}
    \item 假设函数形如:
    \[
    y=e^{rx}
    \]
    \item 将$y,y',y''$代入方程得到:
    \[
    r^2e^{rx}+Pre^{rx}+Qe^{rx}=0
    \]
    \item 由于$e^{rx}>0$,所以要使方程成立,即为求解特征方程:
    \[
    r^2+Pr+Q=0
    \]
    \item 得到两个根:
    \[
    r=\frac{-P\pm \sqrt{P^2-4Q}}{2}
    \]
\end{enumerate}

根据特征方程的根,当$Q\ne 0$时,通解分成以下4个情形:
\begin{itemize}
    \item $P^2>4Q$,$r$有两个不等实根$r_{1,2}=\frac{-P\pm \sqrt{P^2-4Q}}{2}$,于是
    \[
    y=C_1e^{r_1x}+C_2e^{r_2x}
    \]
    物理上对应大阻尼,物理量呈指数衰减趋于0,没有振荡。
    \item $P^2=4Q$,$r$有两个相等实根$r_{1,2}=-\frac{P}{2}$,于是
    \[
    y=\left( C_1+C_2 \right) e^{-Px/2}
    \]
    物理上对应临界阻尼,同上。
    \item $P^2<4Q$,$r$有两个共轭复根$r_{1,2}=\frac{-P\pm i\sqrt{4Q-P^2}}{2}$,于是
    \[
    y=e^{-\frac{P}{2}x}\cdot \left[ C_1\cos \frac{x\sqrt{4Q-P^2}}{2}+C_2\sin \frac{x\sqrt{4Q-P^2}}{2} \right]
    \]
    物理上对应小阻尼,物理量以指数衰减的方式振荡趋于0。
    \item $P=0$,$r$为纯虚根$r_{1,2}=\pm i\sqrt{Q}$,于是
    \[
    y=C_1\cos \left( x\sqrt{Q} \right) +C_2\sin \left( x\sqrt{Q} \right)
    \]
    物理上对应无阻尼振荡。
\end{itemize}

实部决定了{\it y}是否有衰减阻尼,虚部决定了{\it y}是否有振荡。
很多物理过程都可以用二阶常系数齐次线性方程描述,如弹簧震动、RLC电路等。

特别地,当$Q=0$时方程化为:
\[
\frac{d^2y}{dx^2}+P\frac{dy}{dx}=0
\]
求解特征方程$r^2+Pr=0$得到两个根$r=-P\mathrm{or}0$,于是解为:
\begin{align*}
&y=Ce^{-Px} \\
&y=C
\end{align*}

%============================================================
\subsection{PQ关系总结}

将$P,Q$及其关系总结如下:
\begin{itemize}
    \item 当$Q\ne 0,P\ne 0$时,方程为$\frac{d^2y}{dx^2}+P\frac{dy}{dx}+Qy=0$:
    \begin{itemize}
        \item 当$P^2>4Q$时,通解表现为衰减$y=C_1e^{r_1x}+C_2e^{r_2x}$;
        \item 当$P^2=4Q$时,通解表现为衰减$y=\left( C_1+C_2 \right) e^{-Px/2}$;
        \item 当$P^2<4Q$时,通解表现为振荡衰减
        \[
        y=e^{-\frac{P}{2}x}\cdot \left[ C_1\cos \frac{x\sqrt{4Q-P^2}}{2}+C_2\sin \frac{x\sqrt{4Q-P^2}}{2} \right]
        \]
    \end{itemize}
    \item 当$Q\ne 0,P=0$时,方程为$\frac{d^2y}{dx^2}+Qy=0$:
    \begin{itemize}
        \item 通解表现为振荡$y=C_1\cos \left( x\sqrt{Q} \right) +C_2\sin \left( x\sqrt{Q} \right) $。
    \end{itemize}
    \item 当$Q=0,P\ne 0$时,方程为$\frac{d^2y}{dx^2}+P\frac{dy}{dx}=0$:
    \begin{itemize}
        \item 通解表现为衰减$y=Ce^{-Px}$。
    \end{itemize}
    \item 当$Q=0,P=0$时,方程为$\frac{d^2y}{dx^2}=0$:
    \begin{itemize}
        \item 通解表现为恒定$y=C$。
    \end{itemize}
\end{itemize}

~

不难发现:
\begin{itemize}
    \item $P$表示阻尼,如果$P\ne 0$则衰减,如果$P=0$则恒定;
    \item $Q$表示振荡,如果$Q\ne 0$则振荡,如果$Q=0$则无振荡;
    \item 特别地,当$P^2\geqslant 4Q$,阻尼大过振荡,也振荡不起来。
\end{itemize}




