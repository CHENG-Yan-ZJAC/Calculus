\section{无穷小量和无穷大量}

本节给出无穷小量的概念,是微分的基础概念。

本节要点:
\begin{itemize}
    \item 理解无穷小量和无穷大量的概念;
    \item 理解本节最后一条定理。
\end{itemize}

%============================================================
\subsection{无穷量的概念}

\begin{definition}[无穷小量]
若$\forall \varepsilon >0$时存在$\delta >0$,当$x\in N\left( \hat{x}_0,\delta \right) $时有$\left| f\left( x \right) -0 \right|<\varepsilon $,则称$f\left( x \right) $为{\bf 当$x\rightarrow x_0$时的无穷小量},记作
\[
\underset{x\rightarrow x_0}{\lim}f\left( x \right) =0
\]
\end{definition}

注意:0是一个特殊的无穷小量,也是唯一可以作为无穷小量的常数。

\begin{definition}[无穷大量]
若$\forall M>0$时存在$\delta >0$,当$x\in N\left( \hat{x}_0,\delta \right) $时有$\left| f\left( x \right) -0 \right|>M$,则称$f\left( x \right) $为{\bf 当$x\rightarrow x_0$时的无穷大量},记作
\[
\underset{x\rightarrow x_0}{\lim}f\left( x \right) =\infty
\]
\end{definition}

本质上,无穷小量和无穷大量都是极限,是函数的趋势。

%============================================================
\subsection{无穷量的定理}

\begin{theorem}[互倒定理]
在自变量趋向一致下,无穷小量和无穷大量互为倒数。
\end{theorem}

\begin{theorem}[有限和定理]
有限个无穷小量之和(差)仍为无穷小量。
\end{theorem}

\begin{theorem}[有限积定理]
有限个无穷小量之积仍为无穷小量。
\end{theorem}

\begin{theorem}[有界积定理]
若$\underset{x\rightarrow x_0}{\lim}f\left( x \right) =0$,$g\left( x \right) $在$x_0$的某去心邻域内有界,则$\underset{x\rightarrow x_0}{\lim}\left[ f\left( x \right) \cdot g\left( x \right) \right] =0$。
\end{theorem}

\begin{theorem}
$\underset{x\rightarrow x_0}{\lim}f\left( x \right) =A\Leftrightarrow f\left( x \right) =A+o\left( x \right) $,其中$A$为常数,$o\left( x \right) $为无穷小量,即$\underset{x\rightarrow x_0}{\lim}o\left( x \right) =0$。
\end{theorem}

最后一条定理反映的是极限运算,也是后续微分和近似分析的基础。

%============================================================
\subsection{无穷小的比较}

\begin{definition}
设$\underset{x\rightarrow x_0}{\lim}\alpha \left( x \right) =0,\underset{x\rightarrow x_0}{\lim}\beta \left( x \right) =0,\beta \left( x \right) \ne 0$ ,则可有如下定义:
\begin{itemize}
    \item 若有$\underset{x\rightarrow x_0}{\lim}\frac{\alpha \left( x \right)}{\beta \left( x \right)}=0$,则称当$x\rightarrow x_0$时,{\bf $\alpha \left( x \right) $是$\beta \left( x \right) $的高阶无穷小量},或{\bf $\beta \left( x \right) $是$\alpha \left( x \right) $的低阶无穷小量},记作$\alpha \left( x \right) =o\left( \beta \left( x \right) \right) $,
    \item 若有$\underset{x\rightarrow x_0}{\lim}\frac{\alpha \left( x \right)}{\beta \left( x \right)}=C\ne 0$,则称当$x\rightarrow x_0$时,{\bf $\alpha \left( x \right) $和$\beta \left( x \right) $是同阶无穷小量},特别地,如$C=1$,则称{\bf $\alpha \left( x \right) $和$\beta \left( x \right) $是等价无穷小},记作$\alpha \left( x \right) \sim \beta \left( x \right) $,
    \item 若$\underset{x\rightarrow x_0}{\lim}\frac{\alpha \left( x \right)}{\left[ \beta \left( x \right) \right] ^k}=C\ne 0,k>0$则称当$x\rightarrow x_0$时,{\bf $\alpha \left( x \right) $是$\beta \left( x \right) $的$k$阶无穷小量}。
\end{itemize}
\end{definition}

无穷小量的比较的数学意义是描述谁的趋近速度快。




