\section{幂级数}

本节介绍幂级数的概念和性质。
本节是后续小节的概念性基础,所以只需要简单了解即可。

本节要点:
\begin{itemize}
    \item 了解幂级数的概念;
    \item 了解幂级数的性质。
\end{itemize}

%============================================================
\subsection{幂级数的概念}

\begin{definition}[幂级数]
我们称
\[
\sum_{n=0}^{\infty}{a_nx^n}:=a_0+a_1x+a_2x^2+a_3x^3+\cdots
\]
这样的函数项级数为{\bf 幂级数},其中常数$a_n$称为{\bf 幂级数的系数},更一般地,称:
\[
\sum_{n=0}^{\infty}{a_n\left( x-x_0 \right) ^n}:=a_0+a_1\left( x-x_0 \right) +a_2\left( x-x_0 \right) ^2+a_3\left( x-x_0 \right) ^3+\cdots
\]
为在{\bf 点$x_0$处的幂级数}。
\end{definition}

%============================================================
\subsection{幂级数的性质}

\begin{theorem}[阿贝尔(Abel)定理]
设有幂级数$\sum_{n=0}^{\infty}{a_nx^n}$,则:
\begin{itemize}
    \item 若幂级数在点$x_0$处收敛,则区间$\left( -\left| x_0 \right|,\left| x_0 \right| \right) $内幂级数绝对收敛;
    \item 若幂级数在点$x_0$处发散,则区间$\left( -\left| x_0 \right|,\left| x_0 \right| \right) $外幂级数发散。
\end{itemize}
\end{theorem}

\begin{corollary}
若幂级数$\sum_{n=0}^{\infty}{a_nx^n}$不是仅在0点这一点处收敛,但也不是在整个实数上收敛,则必有一个确定的正数$R$,使得:
\begin{itemize}
    \item 区间$\left( -R,R \right) $内,幂级数绝对收敛;
    \item 区间$\left[ -R,R \right] $外,幂级数发散;
    \item $\pm R$两点上,幂级数敛散性不定;
\end{itemize}
此时,我们称$R$为幂级数的{\bf 收敛半径},幂级数的收敛区间必然为$\left( -R,R \right) $、$\left( -R,R \right] $、$\left[ -R,R \right) $、$\left[ -R,R \right] $,四种情况之一。
\end{corollary}

\begin{theorem}[收敛半径定理]
设有幂级数$\sum_{n=0}^{\infty}{a_nx^n}$,若:
\[
\underset{n\rightarrow \infty}{\lim}\left| \frac{a_{n+1}}{a_n} \right|=\rho \quad \text{或} \quad \underset{n\rightarrow \infty}{\lim}\sqrt[n]{\left| a_n \right|}=\rho
\]
则有:
\begin{itemize}
    \item $\rho \ne 0$时,$R=\frac{1}{\rho}$;
    \item $\rho =0$时,$R=\pm \infty $;
    \item $\rho =+\infty $时,$R=0$。
\end{itemize}
\end{theorem}

%============================================================
\subsection{幂级数的运算}
若有两个幂级数,则它们的和差积商结果在公共收敛区间收敛。
商除外,其收敛区间可能会小很多。

%============================================================
\subsection{和函数的基本性质}

设幂级数$\sum_{n=0}^{\infty}{a_nx^n}$的和函数是$S\left( x \right) $,则具有性质:
\begin{itemize}
    \item 在收敛区间内连续;
    \item 在收敛区间$\left( -R,R \right) $内可导,且有:
    \[
    S'\left( x \right) =\sum_{n=0}^{\infty}{\left( a_nx^n \right) '}=\sum_{n=0}^{\infty}{na_nx^{n-1}}
    \]
    称为{\bf 逐项求导公式},公式的右端级数和原幂级数有相同收敛半径,但端点的敛散性可能改变;
    \item 在收敛区间$\left( -R,R \right) $内可积,且有:
    \[
    \int_0^x{S\left( \lambda \right) d\lambda}=\sum_{n=0}^{\infty}{\left( \int_0^x{a_n\lambda ^nd\lambda} \right)}=\sum_{n=0}^{\infty}{\frac{a_n}{n+1}x^{n+1}}
    \]
    称为{\bf 逐项积分公式},公式的右端级数和原幂级数有相同收敛半径,但端点的敛散性可能改变。
\end{itemize}




