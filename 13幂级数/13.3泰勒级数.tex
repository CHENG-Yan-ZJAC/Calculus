\section{泰勒级数}

泰勒级数是幂级数的一种,是利用目标函数自身构造的一个幂级数。

本节要点:
\begin{itemize}
    \item 掌握泰勒级数的概念;
    \item 熟悉常用的麦克劳林展开;
    \item 了解如何使用麦克劳林展开求解超越方程。
\end{itemize}

%============================================================
\subsection{泰勒级数的概念}

\begin{definition}[泰勒级数]
设函数$f\left( x \right) $在$x_0$的某个邻域内有任意阶导数,那么该函数能在该邻域内展开成幂级数:
\begin{align*}
&\begin{aligned}
	f\left( x \right) =&f\left( x_0 \right) +f'\left( x_0 \right) \cdot \left( x-x_0 \right) +\frac{f''\left( x_0 \right)}{2!}\cdot \left( x-x_0 \right) ^2+\cdots\\
	&+\frac{f^{\left( n \right)}\left( x_0 \right)}{n!}\cdot \left( x-x_0 \right) ^n+R_n\left( x \right)\\
\end{aligned} \\
&R_n\left( x \right) =\frac{f^{\left( n+1 \right)}\left( \xi \right)}{\left( n+1 \right) !}\cdot \left( x-x_0 \right) ^{n+1} \qquad \xi \in \left( x,x_0 \right)
\end{align*}
其中$R_n$称为{\bf 拉格朗日余项},开展式成立的充要条件是其拉格朗日余项满足:
\[
\underset{n\rightarrow \infty}{\lim}R\left( x \right) =0
\]
该幂级数称为{\bf 函数$f\left( x \right) $在$x_0$处的泰勒级数},且展开式唯一。
特别地,称0点处的泰勒展开为{\bf 麦克劳林展开},即:
\[
f\left( 0 \right) =f\left( 0 \right) +f'\left( 0 \right) \cdot x+\frac{f''\left( 0 \right)}{2!}\cdot x^2+\cdots +\frac{f^{\left( n \right)}\left( 0 \right)}{n!}\cdot x^n+\cdots
\]
\end{definition}

该定义的意思是,具有任意阶导数的函数,只要余项是收敛的,就都能展开成泰勒级数。

人们在工程应用中总结了几个常用函数的麦克劳林展开:
\begin{align*}
&\frac{1}{1-x}=1+\sum_{n=1}^{\infty}{x^i} \\
&\left( 1+x \right) ^a=1+\sum_{n=1}^{\infty}{\frac{a\left( a-1 \right) ...\left( a-n+1 \right)}{n!}x^n} \\
&e^x=1+\sum_{n=1}^{\infty}{\frac{x^n}{n!}} \\
&\ln \left( 1+x \right) =\sum_{n=1}^{\infty}{\left( -1 \right) ^{n-1}\frac{x^n}{n}} \\
&\sin x=\sum_{n=1}^{\infty}{\left( -1 \right) ^{n-1}\frac{x^{2n-1}}{\left( 2n-1 \right) !}} \\
&\cos x=1+\sum_{n=1}^{\infty}{\left( -1 \right) ^n\frac{x^{2n}}{\left( 2n \right) !}}
\end{align*}

%============================================================
\subsection{泰勒级数的应用}

由于泰勒级数在展开点能很好的逼近目标函数,所以工程上常用于求解超越方程。
下面各给出两个例子。

~

\begin{example}
用幂级数表示$\mathrm{Si}\left( x \right) =\int_0^x{\frac{\sin t}{t}dt}$。
\end{example}

解:

由于
\begin{align*}
\mathrm{Si}\left( x \right) &=\int_0^x{\sum_{n=1}^{\infty}{\left( -1 \right) ^{n-1}\frac{t^{2n-2}}{\left( 2n-1 \right) !}}dt} \\
&=\sum_{n=1}^{\infty}{\frac{\left( -1 \right) ^{n-1}}{\left( 2n-1 \right) !}\int_0^x{t^{2n-2}dt}}=\sum_{n=1}^{\infty}{\frac{\left( -1 \right) ^{n-1}\cdot x^{2n-1}}{\left( 2n-1 \right) !\cdot \left( 2n-1 \right)}} \\
&=x-\frac{x^3}{3\cdot 3!}+\frac{x^5}{5\cdot 5!}-\frac{x^7}{7\cdot 7!}+...
\end{align*}

~

\begin{example}
电路中,电容用于正弦波的滤波,设有解超越方程$\cos \left( \omega \Delta t \right) =e^{-\frac{\Delta t}{RC}}$,求解0点附近的值。
\end{example}

解:

该方程无法得到解析解,但可以用泰勒级数展开,求得近似解。
两边各用泰勒展开,取2级,得:
\begin{align*}
&\because \begin{cases}
	\cos \left( \omega \Delta t \right) \approx 1-\frac{\left( \omega \Delta t \right) ^2}{2!}\\
	e^{-\frac{\Delta t}{RC}}\approx 1+\left( -\frac{\Delta t}{RC} \right)\\
\end{cases} \\
&\therefore \frac{\left( \omega \Delta t \right) ^2}{2!}=\frac{\Delta t}{RC} \\
&\therefore \Delta t=\frac{2}{\omega ^2RC}
\end{align*}

%============================================================
\subsection{欧拉公式}

欧拉公式:
\[
e^{ix}=\cos x+i\sin x
\]

简单推导过程如下:
\begin{align*}
e^{ix}&=1+\sum_{n=1}^{\infty}{\frac{\left( ix \right) ^n}{n!}} \\
&=1+ix+\frac{\left( ix \right) ^2}{2!}+\frac{\left( ix \right) ^3}{3!}+\frac{\left( ix \right) ^4}{4!}+... \\
&=1+ix-\frac{x^2}{2!}-i\frac{x^3}{3!}+\frac{x^4}{4!}+i\frac{x^5}{5!}+...\\
&=\left( 1-\frac{x^2}{2!}+\frac{x^4}{4!}+... \right) +\left( ix-i\frac{x^3}{3!}+i\frac{x^5}{5!}+... \right) \\
&=\left[ 1+\sum_{n=1}^{\infty}{\left( -1 \right) ^n\frac{x^{2n}}{\left( 2n \right) !}} \right] +i\left[ \sum_{n=1}^{\infty}{\left( -1 \right) ^{n-1}\frac{x^{2n-1}}{\left( 2n-1 \right) !}} \right] \\
&=\cos x+i\sin x
\end{align*}

欧拉公式是泰勒展开的必然结果。
由于指数函数和三角函数的麦克劳林展开是在整个实数域成立,所以欧拉公式的定义域是整个实数域。
公式揭示了指数函数和三角函数的关系,在许多领域有广泛用途。
这里只是粗略地推导,严格推导需要证明复数项级数绝对收敛、实部绝对收敛、虚部绝对收敛,才可以这么转换。

\begin{tcolorbox}
泰勒展开是用目标函数的导数进行幂级数的构建,而只有指数函数、正弦函数和余弦函数的导数才会是自己,加上0点处的泰勒展开,所构建的幂级数的系数不会受到“污染”。
这就是欧拉公式成立的原因。
从线性代数的角度看,指数函数、正弦函数和余弦函数能够构成一组正交基。
\end{tcolorbox}




